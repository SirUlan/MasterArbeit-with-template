\chapter{Luminescence of semiconductors}
\label{ch:scintillation}
Under the term ``luminescence'' are classified phenomena, which concern the absorption of energy and the subsequent emission of light. This is a very broad topic as it is a property of different materials (gases, such as in gas-discharge lamps, organic and inorganic semiconductors, such as in (O)LEDs and scintillators, etc.) and are produced by different processes (material excitation by photons, called photoluminescence; photons from biochemical reactions known as bioluminescence; the release of absorbed energy on heating the material, thermoluminescence, etc.). This chapter focuses on a general description of the production of light in semiconductors after excitation by high-energetic particles, such as alpha particles and electrons, also known as scintillation.


\section{Basic mechanism of radiative transitions}
\label{sec:basicmechanism}
Essentially, the photon emission in a luminescence process stems from an electronic transition between an initial state of energy $E_i$ and a final state $E_f$, where $E_i>E_f$. Assuming an electron population $N_{\textrm{i}}$ in the state $i$, the electrons will spontaneously decay from the higher energy level to the lower state with the probability per unit time $A_{\textrm{i}\rightarrow\textrm{f}}$, thereby emitting a photon with an energy $h\nu =E_i-E_f$\footnote{Indeed, even in this simplified picture, the photons follow a narrow spectral linewidth due to the energy-time uncertainty principle.} with a rate expressed as
\begin{equation}
\label{eq:pop}
    \dot{N_{\textrm{i}}} = -A_{\textrm{i}\rightarrow\textrm{f}} \cdot N_{\textrm{i}} \hspace{1cm}\rightarrow \hspace{1cm} \frac{d N_{\textrm{i}}}{N_{\textrm{i}}} = -A_{\textrm{i}\rightarrow\textrm{f}} \cdot dt.
\end{equation}
The solution of equation \ref{eq:pop} is an exponential decay 
\begin{equation}
N_{\textrm{i}}(t) = N_{\textrm{i}}(0) \cdot \exp{(-A_{\textrm{i}\rightarrow\textrm{f}}\cdot t)} = N_{\textrm{i}}(0) \cdot \exp{(-t/\tau_{\textrm{i}\rightarrow\textrm{f}})},
\end{equation}
where $N_{\textrm{i}}(0) $ is the initial population and $\tau_{\textrm{i}\rightarrow\textrm{f}} = A_{\textrm{i}\rightarrow\textrm{f}}^{-1}$ is the lifetime of the transition \sloppy{\mbox{\cite[p.~210]{InorganicGlasses}}}. These are general relations for any two-level radiative transitions, such as in the case of atomic spectral lines, where the two energy levels are atomic orbitals.

\par In a crystal lattice, the electronic energy levels of neighbouring atoms overlap and split due to Pauli's exclusion principle, thereby being condensed into a band structure. The outermost occupied orbital of the atoms forms the valence band (VB), while the lowest unfilled energy band is the conduction band (CB). In semiconductors, these bands are separated by a bandgap without allowed electron states, with a width of $E_g < \SI{5}{eV}$ \cite[p.~366]{knoll2010radiation}. Hence, electrons in the valence band can be externally excited to the conduction band, e.g.\ by means of the absorption of a photon with energy $E\geq E_g$, leaving an unfilled position in the valence band also called a hole. The recombination of this electron-hole pair can be radiative, emitting a photon with a wavelength corresponding to the energy released $E\leq E_g$. The deepest level of the conduction band and the maximal-energy state of the valence band each corresponds to a certain momentum of the lattice (wave vector). If these wave vectors are the same, the band gap is called a direct gap; otherwise, it is referred as an indirect gap. In the case of the latter, the momentum of electrons and holes are not the same and a phonon-assisted transition is needed for the absorption and the emission of a photon. As this requires an extra particle, the probability of such a process is lower than in the case of direct transitions and thus the fundamental emission in indirect-gap semiconductors is relatively weak \sloppy{\mbox{\cite[p.~23]{Yacobi_1990}}}.


\par In the case of radioactive decays inside the glass, the excitation of electrons is caused by high-energy charged particles. The energy released by these particles ($\mathcal{O}(\textrm{keV})$-$\mathcal{O}(\textrm{MeV})$) is much larger than the binding energies of atomic electrons ($\sim \SI{10}{eV}$) and thus they lose their kinetic energy almost completely due to the excitation and ionisation of bound electrons in the material. During this process, the particles may also produce high-energy electrons (also called delta rays) that lose their energy exciting more electron-hole pairs. When the energy of the generated electrons is lower than the bound energy, they thermalise by intraband, radiative and nonradiative transitions. Let us first focus on the radiative processes inside semiconductors.


\begin{wrapfigure}{O}{0pt}
\centering
\includegraphics[scale=1.0]{Figures/theory/bandmodel.pdf}
\caption{Schematic diagram showing transitions between the conduction band (CB), the valence band (VB) and donor ($E_D$) and acceptor ($E_A$) levels. Based on \cite[p.~25]{Yacobi_1990}}
\label{fig:bandmodel}
\end{wrapfigure}
\par The emission spectra of a semiconductor can be divided between \textit{intrinsic} (fundamental or edge emission) and \textit{extrinsic}, also known as activated or characteristic emission \cite[p.~22]{Yacobi_1990}. The intrinsic emission is caused, as its name suggests, due to the inherent properties of the material, viz.\ the aforementioned recombination of electrons and holes across the conduction and the valence band of the lattice, which is the inverse mechanism responsible for the fundamental optical absorption edge. On the other hand, the emission activated by impurities in the crystal is extrinsic in nature since they originate from energy levels in the band gap. This can be made much more intense than the intrinsic case, depending on the kind of activator and its concentration, which is the base of luminescent materials and phosphor technology. Figure \ref{fig:bandmodel} summarises a simplified picture of radiative transitions found in semiconductors. The first process (a) describes the intrinsic emission of $h\nu \cong E_g$ by the recombination of states close to the band edges. Nevertheless, normally, it results in a broad emission spectrum owing to the thermal distribution of charge carriers (see section \ref{sec:abandem}). The next three processes (b) describe extrinsic emissions arising from decays that start and/or finish on energy levels of impurities (called donor, in the case of states near the CB and acceptors near the VB). As mentioned earlier, this can stem from impurity atoms, but also from lattice defects, such as dislocations, which can produce both shallow and deep energy levels. The last process (c) depicts the excitation and the radiative deexcitation through an impurity of incomplete inner shells, such as rare-earth ions (lanthanides). For these elements, the energy difference in the 5d and the 4f-shell is in favour of the 5d occupancy, thus leaving a vacant 4f-shell. The electrons of this shell are screened by the electrons in the outer orbitals and are therefore barely affected by the crystal lattice \cite[p.~221]{InorganicGlasses}. This gives rise to well-defined energy transitions that can be easily recognised. Since lanthanides are commonly found in nature, they may be a relevant trace luminescence centre for studies in this thesis.
\par As already denoted, the emission and the absorption of light in solids is a consequence of symmetric processes. This relation is illustrated in the next section, as it helps to explain luminescence properties exhibited in a material.



\section{Absorption and emission of photons}
\label{sec:abandem}
\begin{wrapfigure}{O}{0.4\textwidth}
\centering
\includegraphics[width=0.4\textwidth]{Figures/theory/Frankcandon.pdf}
\caption{Configurational coordinate diagram for a radiative recombination involving the emission of phonons. From \cite{Salh_2011} and modified.}
\label{fig:frankcandon}
\end{wrapfigure}
The process of absorption and emission from luminescent point defects can be elementary described using a configuration coordinate diagram (see figure \ref{fig:frankcandon}), which is a representation of the effects of the relaxation of the crystal following optical transitions \cite[p.~4]{LuminescenceTh2app}. Here, $Q_g$ and $Q_e$ represent the distances of the nucleus in the ground (lower curve) and excited state (upper curve), respectively, and $E_a$ and $E_e$ are the energies at which the absorption and the emission are the most intense. At a temperature of $\SI{0}{\kelvin}$ the electron occupies the lowest vibrational level of the electronic ground state ($n=0$) and any transition to the excited state will take place from this level. As the electron transitions are faster than the relaxation of the position of neighbouring atoms, the absorption occurs adiabatically represented by a vertical transition in the diagram (the Frank-Condon principle \cite[p.~4]{LuminescenceTh2app}). The absorption transition carries the electron to an excited vibrational state and vertically projecting the endpoints of the $n=0$ results in the width of $E_a$. After the absorption, the configuration relaxes through the emission of phonons to the zero vibration level of the excited state ($m=0$) and the electron returns to the ground state, in this example via a vertical luminescence emission corresponding to the energy $E_e$. Analogously, projecting the endpoints of the $m=0$ results in the width of the $E_e$ emission band. The energy difference between $E_a$ and $E_e$ is called the Stokes shift. The greater the number of phonons involved in the transition, the larger will be this shift. Conversely, with a considerable overlap of the lowest vibrational states $n=0$ and $m=0$, the transition can occur without emission of any phonon and $E_e = E_a$. This is known as the zero-phonon-line. The transitions involving phonons form the phonon sideband, which lies at higher energies than the zero-phonon-line for absorption processes and at lower energies for radiative transitions.
\par  As the temperature rises, higher vibrational states $n>0$ are populated and more transitions occur from these levels to higher vibrational levels of the excited electronic state, resulting in broader absorption bands. The distribution of intensity between the zero-phonon-line and the sideband is also strongly temperature dependent. For example, at room temperature the probability for a $n=0 \leftrightarrow m=0$ transition is negligible \cite[pp.~30-32]{Yacobi_1990} since the energy is high enough to emit a lot of phonons. 


\section{Nonradiative transitions}
So far, only a de-excitation via radiative transitions has been assumed. However, the absorbed energy can be dissipated to the crystal lattice through different processes without the emission of photons, which are categorised as nonradiative transitions. Depending on the material, there are many kinds of such transitions and thus only the two most important processes, which occur in all inorganic semiconductors, are going to be explained in this section.
\par The principal nonradiative relaxation involves multiphonon emission, also known as thermal quenching. This effect is best appreciated by constructing a configuration coordinate diagram (see figure \ref{fig:nonradiative}a). When the excited state and the ground state energy curves overlap at an energy thermally accessible, the electron can escape the excited state and return to the minimum energy generating phonons. The energy is therefore given up as heat to the lattice. The probability of thermal quenching $P_{\textrm{nr}}$ is temperature-dependent as the electron in the excited state must overcome the barrier $W$, and is given by
\begin{equation}
\label{eq:thque}
    P_{\textrm{nr}} = C \cdot \exp{(-\frac{W}{k_{\textrm{B}}T})},
\end{equation}
where $k_{\textrm{B}}$ is Boltzmann's constant, $T$ the temperature and C is a constant (units $\textrm{s}^{-1}$) \cite{Akselrod_1998}.
\par Another source of nonradiative transitions are surface recombinations, since a continuum of states may join the conduction with the valence band, dissipating the excitation energy through the emission of phonons. These energy levels arise due to the abrupt change of the band structure of the bulk of the crystal, in addition to impurity atoms and oxide layers present on the surface. Furthermore, localised state bridges across the energy gap can be produced by crystal defects, such as pores, dislocations and grain boundaries \sloppy{\mbox{\cite[p.~7]{Electroluminescence}}}. Although these recombination centres are very localized, the effect extends over the electron diffusion length. The configuration coordinate diagram of this effect is shown in figure \ref{fig:nonradiative}b.

\begin{figure}[b]
\centering
\begin{minipage}{.5\textwidth}
  \centering
  \includegraphics[scale = 1.0]{Figures/theory/thermalquenchingNR.pdf}
\end{minipage}%
\begin{minipage}{.5\textwidth}
  \centering
  \includegraphics[scale = 1.0]{Figures/theory/defectNR.pdf}
\end{minipage}  
\caption{Configuration coordinate diagram for two nonradiative transitions, \textbf{a)} thermal quenching \cite[p.~6]{Electroluminescence} and \textbf{b)} energy level crossing due to defects and bounderies effects \cite{Shu_2017}.}
\label{fig:nonradiative}
\end{figure}



\section{Quantum yield and lifetime}
\label{sec:QYLif}
As introduced in the beginning of section \ref{sec:basicmechanism}, the lifetime of a transition is $\tau = A^{-1}$, where $A$ is the probability of decay per unit time. If for a transition competitive radiative and nonradiative processes are present, the observable lifetime $\tau$ is given by
\begin{equation}
    \tau^{-1} = k_{\textrm{r}}+k_{\textrm{nr}} = \tau^{-1}_{\textrm{r}}+\tau^{-1}_{\textrm{nr}},
\end{equation}
where $k_{\textrm{r}}$ and $k_{\textrm{nr}}$ are the radiative and nonradiative rate constants respectively $A = k_{\textrm{r}}+k_{\textrm{nr}}$ and $\tau^{-1}_{\textrm{r}}$, $\tau^{-1}_{\textrm{nr}}$ are the respective lifetimes \cite[p.~31]{LuminescenceTh2app}. In general $\tau^{-1}_{\textrm{nr}}$ is the product of several nonradiative recombination processes and hence $\tau^{-1}_{\textrm{nr}} = \sum_i \tau^{-1}_{\textrm{nr,i}}$. Therefore the observable lifetime is dominated by the fastest time constant. 

\par For example, in case of one transition of lifetime $\tau_0$ and only considering thermal quenching, following equation \ref{eq:thque}, the total probability of decay per unit of time is given by
\begin{equation}
    P(T) = \frac{1}{\tau_0}+ C \cdot \exp{(-\frac{W}{k_{\textrm{B}}T})}
\end{equation}
and thus the lifetime of the transition will follow the temperature dependence:
\begin{equation}
\label{eq:tauTemp}
   \tau = \frac{\tau_0}{1+\tau_0\cdot C \cdot \exp{(-\frac{W}{k_{\textrm{B}}T})}}.
\end{equation}
Hence, the lifetime in this system decreases with higher temperatures. Furthermore, by measuring $\tau$ at different temperatures, it is possible to calculate the thermal quenching parameters $C$ and $W$. However, there may be other non-negligible processes that show some temperature dependence, such as phonon-assisted indirect transitions, where the probability is $\propto \textrm{coth}(\frac{E_{ph}}{k_{\textrm{B}}T})$ for phonons with energy $E_{ph}$ \cite{Akselrod_1998}.
\par In a heterogeneous sample, where many luminescence centres are available, the time distribution of the fluorescence decay $I(t)$ can be described with a multiexponential model
\begin{equation}
\label{eq:multiexponentialdecay}
    I(t) = \sum_i \alpha_i \exp{(-\frac{t}{\tau_i})},
\end{equation}
where $\tau_i$ are the lifetimes of the transitions present in the sample and $\alpha_i$, its relative intensity.
\par The internal quantum yield or radiative recombination efficiency $\eta$ is defined as the ratio of luminescence photons emitted to the total recombination rate
\begin{equation}
\label{eq:recEff}
    \eta = \frac{k_{\textrm{r}}}{ k_{\textrm{r}}+k_{\textrm{nr}}} = \frac{\tau}{\tau_{\textrm{r}}} = \frac{1}{1+\tau_{\textrm{r}}/\tau_{\textrm{nr}}}.
\end{equation}
Following equation \ref{eq:tauTemp}, for a lattice where thermal quenching is the principal nonradiative transition, the recombination efficiency features a temperature dependence given by
\begin{equation}
    \eta(T) = \frac{1}{1+\tau_0\cdot C\cdot \exp{(-\frac{W}{k_{\textrm{B}}T})}},
\end{equation}
and, therefore, an increase in temperature is expected to result in a lower value for the efficiency \cite[p.~32]{LuminescenceTh2app}. However, as the determination of $k_{\textrm{r}}$ and $k_{\textrm{nr}}$ is not trivial, as every process involved has to be well known, usually the yield $\varepsilon$ is given as a parameter for the recombination efficiency, especially in the case of scintillator physics. The scintillation yield is the average number of emitted photons per unit of absorbed energy ($\textrm{MeV}^{-1}$) and thus the expected number of radiative transitions $\langle N \rangle$ after the absorption of the energy $E$ is given by
\begin{equation}
\label{eq:linScint}
   \langle N \rangle = \varepsilon\cdot E.
\end{equation}
The yield can be more easily determined experimentally than the recombination efficiency, and thus it enjoys common usage. However, the response to different energies is not always proportional and the yield varies between different particles. Both effects are the product of clusters of electronic excitations as a charged particle will produce many electron-hole pairs in a short distance. Luminescence centres can only provide one electron-hole and, if there are more electron-hole pairs in the ionisation volume than luminescent centres, the number of radiative transitions will decrease. Furthermore, the probability of Auger-like processes, where the energy of an electron is transferred to another one, increases with the charge carrier density, thus reducing the luminescence efficiency \cite{Moses_2008}. Henceforth, the higher the stopping power, the lower the yield shown by a sample and, for example, measurements of lithium glass scintillators show that the yield for $\SI{1}{MeV}$ protons, deuterons and alpha particles is lower than the one for $\SI{1}{MeV}$ electrons by a factor $2.1$, $2.8$ and $9.5$, respectively \cite[p.~255]{knoll2010radiation}. This is, nevertheless, highly material dependent.