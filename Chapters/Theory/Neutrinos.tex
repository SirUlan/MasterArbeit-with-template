\chapter{Neutrino astronomy}
This chapter provides a brief introduction to the detection of high-energy neutrinos with the IceCube Observatory. This detector is part of the rapidly-developing neutrino astronomy, therefore the next section will present one of the main motivations for this investigation field. 
\section{The cosmic ray riddle}
At the beginning of the 20th century, Victor Hess discovered that the Earth is constantly being bombarded by charged particles, the \textbf{cosmic rays}. Ever since, the energy and composition of this flux have been measured with precision and today are known in great detail. They are ionised nuclei, being about $\SI{90}{\percent}$ protons, $\SI{9}{\percent}$ $\alpha$-particles and the rest heavier nuclei \cite[p.~6]{Resconi}. Their most outstanding feature is their high-energy, as the spectrum of these particles spans several orders of magnitude and events with energies from $\sim\SI{10}{GeV}$ up to $\mathcal{O}$(EeV) have been observed \cite[p.~12]{Resconi}.
\par Despite the vast studies done with cosmic rays, the origin and production mechanism of these particles remains unclear. In this context, several models that could explain the observed phenomena have been proposed and can be classified into two classes. A \textbf{top-down} scenario assumes that cosmic rays are decay products of heavy remnants of the early universe, such as topological defects or dark matter particles. These models, however, have recently been essentially excluded due to constraints from experimental observations \cite{Semikoz:2007wj}. In contrast, \textbf{bottom-up} theories suggest that low-energy charged particles are gradually accelerated to the observed energies. In this regard, various potential accelerating galactic and extra-galactic objects have been suggested.
\par A promising source in the galactic region are \textbf{SuperNove Remnants} (SNR).  After the detonation of a supernova, a shock wave of the star's material is released at enormous velocities into the interstellar medium. This provides the necessary conditions for the acceleration of cosmic rays. Indeed, based on their energy release, SNRs are thought to be the main galactic source of cosmic rays. On the other side, the charged particles with the highest energies are considered to be accelerated outside the Milky Way. A candidate for the acceleration of this extra-galactic component are \textbf{Active Galactic Nuclei} (AGN). AGNs are compact regions in the centre of a galaxy, source of extremely intense electromagnetic radiation. This radiation is believed to be caused by a super-massive black hole surrounded by a dense accretion disc, from which mass is drawn. The most distinctive feature of AGNs is, however, the two jets of extremely relativistic matter emitted perpendicular to the accretion disk. In these jets, shock waves propagate, which are assumed to efficiently accelerate particles to very high energies \cite{Katz:2011ke}.


\par Nevertheless, a definitive identification of the source of cosmic rays is impeded by the fact that as particles propagate from the acceleration zone, they are deflected by intergalactic magnetic fields. Therefore, the directional information of the particles is lost before they arrive on Earth. To identify the sources, it is necessary to study the neutral component of cosmic rays. As the charged particle is being accelerated, they can interact hadronically with the surrounding matter, producing neutrinos and $\gamma$-rays. These daughter particles feature also very high energies, as their energy spectrum is expected to follow that of the cosmic rays. 
\par In the case of $\gamma$-rays, their use for the search of extra-galactic sources is limited, as their mean free path is restricted by their interaction with the interstellar medium and the cosmic microwave background. For galactic sources, these effects start to be prominent for rays with energies over $\SI{100}{TeV}$ \cite{Franceschini:2017iwq}. Nevertheless, with the increasing technical capabilities, neutrino astronomy has become plausible and an increasingly important tool in the exploration of the source of cosmic radiation. The next section introduces the fundamental properties of this particle and the mechanisms used in the neutrino astronomy for its detection.


\section{Neutrino properties and interactions}
Neutrinos are elementary particles included in the lepton family and thus an integral part of the Standard Model of particle physics. They are quasi-massless\footnote{Although the neutrino mass is not exactly known, they are so light that the gravitational force can be neglected in their interactions.}, do not possess electrical charge and being leptons, they do not undergo strong interactions. Thus, neutrinos can only interact through the weak force. 
\par These particles come in three flavours: electron, muon and tau neutrinos, corresponding to the three charged leptons. A variety of experiments in recent decades have proven that the neutrino lepton flavour is not conserved since the flavour of a neutrino flux can partially change after propagation in vacuum or matter. This process is referred to as neutrino oscillation. Until now, the only consistent explanation for this property is a nonzero neutrino mass, opposed as initially considered in the Standard Model \cite{Olive_2016}. 
\par The neutrino properties make their direct detection impossible and it has to be carried out via charged secondary particles produced through interactions. At high energies (>$\SI{10}{GeV}$), these interactions are dominated by deep inelastic scattering with nucleons. Depending on the exchanged boson, there are two branches in which neutrinos will engage - the neutral current (exchange of a $Z^{0}$) or the charged current (exchange of a $W^{\pm}$):
\begin{equation}
\label{eq:interactions}
    \nu_l+N \xrightarrow{Z^{0}} \nu_l+X,\hspace{1cm} \nu_l+N \xrightarrow{W^{\pm}} l+X,
\end{equation}
where $\nu_l$ represents a neutrino (or an antineutrino) with the lepton flavour $l$, $N$ denotes a nucleon (proton or neutron), $l$ the emitted lepton (or antilepton) and $X$ represents the additional rest products of the interaction, which can be one or more hadronic particles. In equation \ref{eq:interactions} the lepton flavour is strictly conserved, which may seem conflicting with the aforementioned neutrino oscillation. It has to be emphasised that the flavour change is a consequence of the flavour mixture of mass eigenstates after propagation. The violation of the lepton flavour number in an interaction implies CPT symmetry breaking \cite{D_az_2016}, a violation that at this time has not been confirmed by experiments \cite{hep-ph/0611177}.
\par As neutrinos posses a very small cross-section, they rarely interact with matter. Therefore, large interaction volumes are needed for the detection of a sufficient number of these particles. This is especially true for naturally occurring neutrinos, like the neutrinos from astrophysical sources, since their flux is quite low compared to, e.g.\ neutrinos produced in reactors.



\section[Detection of high-energy neutrinos via Cherenkov radiation]{\large{Detection of high-energy neutrinos via Cherenkov radiation}}
\label{sec:cherenkov}
As we already saw, neutrinos cannot be measured directly. The charged secondary particles emitted in interactions such as the ones in the equation \ref{eq:interactions} can, however, be detected by different techniques. The most widespread method involves detection via the Cherenkov effect, which is used, e.g.\ for the detection of high-energy neutrinos in large-volume neutrino telescopes such as ANTARES, BDUNT, NESTOR and IceCube \cite{Ageron_2011, DBUNT, Rapidis_2009, Aartsen_2017}. This effect will be explained in this section. Nevertheless, depending on the energy range and the issue of study, other techniques have been used, such as radiochemical methods(e.g.\ the GALLEX/GNO and SAGE experiments \cite{xing2011neutrinos}), tracking calorimeters (MINOS, NO$\nu$A \cite{Backhouse_2015}) and radio detectors (ANITA, RICE, ARA \cite{ANITA,RICE,Meures_2015}).\hfill
\label{sec:Cherenkov}
\begin{wrapfigure}{o}{0.4\textwidth}
\centering
\begin{flushleft}
\includegraphics[width=0.4\textwidth]{Figures/theory/Cherenkov80.pdf}
\end{flushleft}
\caption{Schematic of Cherenkov radiation, in this case, being emitted by a muon $\mu$.}
\label{fig:CherenkovProcess}
\end{wrapfigure}
\par The Cherenkov effect takes place when a charged particle passes through a dielectric at speeds
 greater than the phase velocity of light in that medium. The electric field of the particle polarises the medium (small displacements undergone by a vast number of electrons) and on the de-excitation of the electrons (return to their normal position), radiation is emitted. If the charged particle is slower than the phase velocity of this radiation, the emission interferes destructively. If this is not the case, i.e.\ it is faster than light in the medium and the wavelets of the track are in phase overlapping constructively on a wavefront, causing the coherent emission of photons \cite{Jelley_1955}. A sketch of this process is shown in figure \ref{fig:CherenkovProcess}. The Cherenkov radiation is released in the form of a cone with an opening angle $\theta$ relative to the particle's trajectory. This angle is defined by the velocity of the wavefronts $v_w = \dfrac{c}{n}$, which depends on the refractive index of the medium $n$, and the velocity of the charged particle $v_p = \beta \cdot c$, where $\beta$ is the ratio of $v_p$ to the speed of light $c$. Following figure \ref{fig:CherenkovProcess} this angle is given by:
\begin{equation}
\cos(\theta) = \dfrac{c\cdot t \cdot n^{-1}}{\beta \cdot c \cdot t} = \dfrac{1}{\beta \cdot n}.
\end{equation}
Employing the condition for the Cherenkov effect $\beta > n^{-1}$, we can directly determine the minimal kinetic energy $E_{\rm{th}}$ of a charged particle for the production of Cherenkov light as follows:
\begin{equation}
    E_{\rm{kin}} = (\gamma-1) m_{0} c^2 =m_{0} c^2 \big(\dfrac{1}{\sqrt{1-\beta^2}}-1\big) > E_{\rm{th}} =m_{0} c^2\big(\dfrac{1}{\sqrt{1-n^{-2}}}-1\big) =  m_{0}c^2\big(\sqrt{\dfrac{n^2}{n^2-1}}-1\big),
\end{equation}
where $m_0$ is the invariant mass of the particle and $\gamma = \dfrac{1}{\sqrt{1-\beta^2}}$ is the Lorentz factor. Hence, the energy threshold only depends on the refractive index of the medium. After inserting the characteristic index for ice $n=1.31$, the needed kinetic energy will be $\sim \SI{55}{\percent}$ of its rest energy. For larger refractive indices, this energy threshold gets lower, as for a typical glass, with $n=1.48$, only $\sim \SI{36}{\percent}$ of the particle's rest energy is necessary. For an electron, this results in $\sim \SI{0.28}{MeV}$ and $\sim \SI{0.18}{MeV}$, respectively.
\begin{wrapfigure}{O}{0.35\textwidth}
\centering
\includegraphics[width=0.35\textwidth]{Figures/theory/neutrinos-sig-1.png}
\rule[2.5mm]{0.33\textwidth}{0.1pt}
\includegraphics[width=0.35\textwidth]{Figures/theory/neutrinos-sig-2.png}
\rule[2.5mm]{0.33\textwidth}{0.1pt}
\includegraphics[width=0.35\textwidth]{Figures/theory/neutrinos-sig-3.png}
\caption{Light deposition signatures of showers (top), track (middle) and double-bang events (bottom). The colour indicates the arrival time of the photos, going from red (early) to blue (later), while the size of the spheres illustrates the amount of detected light. Figures taken from the IceCube MasterClass\\ \url{https://goo.gl/jXorcb}. }
\label{fig:tracks}
\end{wrapfigure}
\par The spectrum of the emission and the number of photons radiated per unit length can be approximated by the Frank-Tamm formula \cite{Frank_1991}
\begin{equation*}
\frac{d^2 N}{dxd\lambda} = \frac{2\pi\alpha}{\lambda^2}\cdot \Big(1-\frac{1}{\beta^2 n(\lambda)^2} \Big),
\end{equation*}
where $N$ is the number of emitted photons and $\alpha\approx\frac{1}{137.04}$ is the fine structure constant. It is noteworthy that the emission spectrum, in the first order, is proportional to $\lambda^{-2}$, although there is a small dependence on the refractive index of the medium $n(\lambda)$. 
By recording the number of emitted photons and its time distribution, different light patterns can be identified, with which it is possible to reconstruct the energy and the direction of the charged particle. 
\par The light signatures produced by leptons after a charged current neutrino interaction are reasonably distinctive depending on their flavour as their lifetime and probability of interaction vary. These signatures start at the interaction vertex, where a hadronic cascade, originating from the debris of the hit nucleon, takes place; this produces a near-spherical mark. Electrons or positrons travel very short distances as electromagnetic cascades are quickly generated as a result of scattering in the medium, bremsstrahlung and pair production. Altogether, their signatures resemble a point source with maximum photon production in the centre and are called \textbf{showers}. In contrast, highly energetic muons can travel very long distances before they decay or are stopped and thus produce a \textbf{track}-like signature. Tau neutrinos will emit a tauon, which, after a short distance, decays via weak interaction into a muon, electron or producing light mesons. These particles will create an electromagnetic or a hadronic cascade, respectively. As spherical light emission is produced by both cascades, the hadronic cascade at the interaction vertex and the one originated from the tau decay, the signature is referred as a \textbf{double-bang}. An example of these three signatures can be seen in figure \ref{fig:tracks}.
\par A shower signature is also created by all neutrino flavours undergoing a neutral current interaction, as it only induces a hadronic cascade. The energy of the neutrino can be well reconstructed from the total light deposition with these kinds of signatures as all the energy is deposited inside of a relatively small volume. However, owing to its almost spherical shape, the direction reconstruction is limited to only the time information. Track signatures, on the other side, allow for very precise directional reconstruction, although to derive the initial energy of the neutrino accurately, the interaction vertex has to be measured within the detector.
\par The IceCube Observatory uses this method for the detection of neutrinos. In the next section, different parts of this detector and its main features are discussed.

