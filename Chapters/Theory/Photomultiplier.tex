
\chapter{Photomultiplier tubes}
As seen in the last sections, the heart of the optical modules is their photomultiplier tubes (PMTs). These devices convert photons into a measurable electrical signal and are extremely sensitive. Their amplification process by emission of secondary electrons enables the detection of single photons. \textcolor{red}{Implications for IceCube}. This section gives a short introduction into these devices. F
\section{Construction and operation}
\begin{wrapfigure}{R}{0.5\textwidth}
\centering
\includegraphics[width=0.5\textwidth]{Figures/theory/test}
\caption{Flower two.}
\label{fig:backgroundComparison}
\end{wrapfigure}
Figure myref shows the essential components of a PMT. (Following the detection and conversion of light), the first stage takes place at the photocathode. This is a thin layer of photosensitive material, which can absorb photons and convert them into electrons (photoelectrons) via the photoelectric effect. These photoelectrons are focused and electrostatically accelerated towards the electron multiplier, which consists of a series of electrodes (dynodes). When an electron strikes a dynode, it liberates secondary electrons, that are then guided towards the next dynode and finally the anode,  which delivers the output signal. For this to happen, high voltage is 
applied to the cathode, dynode and anode by a voltage divider. The intensity of this potential changes stepwise between the different components so that the electrons are always predominantly accelerated towards the next stage.
\section{Important PMT parameters}
\subsection{Dark rate}