
\chapter{Photomultiplier tubes}
\label{ch:PMTs}
As seen in the last sections, the heart of the optical modules are their photomultiplier tubes (PMTs). These devices convert photons into a measurable electrical signal and are extremely sensitive. Their amplification process by the emission of secondary electrons enables the detection of single photons. 
\par This section gives a short introduction to these devices, as they are key in understanding the measurements done in this work.



\section{Construction and operation principle}
\begin{wrapfigure}{O}{0.4\textwidth}
\centering
\includegraphics[width=0.4\textwidth]{Figures/theory/PMT-constr.pdf}
\caption{Schematic of a photomultiplier tube with its main constituents. The red sinusoidal line represents a photon that emits a photoelectron (blue line) at the photocathode. The secondary electrons from the dynodes are represented by yellow lines.}
\label{fig:PMTsketch}
\end{wrapfigure}
Figure \ref{fig:PMTsketch} shows the essential components of a PMT. The first stage of photon detection takes place at the photocathode. This is a thin layer of photosensitive material, which can absorb photons and convert them into electrons (photoelectrons) via the photoelectric effect. These photoelectrons are focused and electrostatically accelerated towards the electron multiplier, which consists of a series of electrodes (dynodes). When an electron strikes a dynode, it liberates secondary electrons, which are then guided towards the next dynode and finally the anode, which delivers the output signal. For this to happen, a high voltage is applied to the cathode, dynodes and anode by a voltage divider. The intensity of this potential changes step-wise between the different components so that the electrons are always predominantly accelerated towards the next stage.
\par The electrons that arrive at the anode can be read out directly as a charge or as a voltage signal. In the first case, the charge of the PMT pulses that arrive within a certain time window is integrated by a current measuring device (e.g. picoammeter) resulting in a current value. This operation mode is called current or analogue mode and is used when the PMT is illuminated with a steady intensity of light, or the time information of individual pulses is not needed. 
\par In pulse or photon-counting mode the output is supplied to a resistor which results in a voltage pulse (e.g. employing an oscilloscope). This way single pulses can be analysed extracting the maximum information from data, which provides an advantage over the current mode in low-light-intensity measurements.
\section{Main PMT parameters}
Several parameters and properties characterise the performance of a PMT. This section provides an overview of the most relevant ones for this thesis. An introduction to this topic can be found in \cite{hamamatsuBook} and \cite{Flyckt}, while PMTs are also covered extensively in the textbook \cite{wright2017photomultiplier}.



\subsection{Quantum efficiency}
\label{QE:Theory}
\begin{wrapfigure}{O}{0.5\textwidth}
\centering
\includegraphics[scale=1.0]{Figures/theory/Hamamatsu-QE.pdf}
\caption{Quantum efficiency of standard bialkali, SBA and UBA with a borosilicate glass entrance window. Data was taken from \cite{suyama_nakamura_2009}.}
\label{fig:QEHamamatsu}
\end{wrapfigure}
The \textbf{quantum efficiency (QE)} is the number of photoelectrons emitted from the photocathode divided by the number of incident photons. This property is wavelength dependent and is mainly determined by the photocathode material and the optical properties of the PMT entrance window. Photocathodes are compound semiconductors of alkali metals that have a low work function. There are several kinds currently employed, although for this work the relevant ones are bialkali compounds (Sb-Rb-Cs, Sb-K-Cs) \cite[30]{hamamatsuBook}. The standard Bialkali exhibits a maximal QE of $\sim \SI{25}{\percent}$ for wavelengths around $\SI{400}{nm}$, but improving its crystallinity the QE can be increased to $\sim \SI{35}{\percent}$ (Super Bialkali, SBA) and over $\SI{40}{\percent}$ (Ultra Bialkali, UBA) \cite{suyama_nakamura_2009}. Typical QE curves for standard bialkali, SBA and UBA photocathodes are shown in figure \ref{fig:QEHamamatsu}.
\par Bilalkali photocathodes also feature high sensitivity down in the ultraviolet (UV) region, but this is limited by the transmission of the window material. Most commonly the windows are made of borosilicate glass, which exhibit a transmission cut-off at around $\SI{300}{nm}$ absorbing photons with wavelengths under this boundary. For an improved QE in the UV range, windows made from e.g. synthetic silica (cut-off at $\sim \SI{160}{nm}$) or UV glass (cut-off at $\sim \SI{185}{nm}$) can be used \cite[36]{hamamatsuBook}.





\subsection{Gain and collection efficiency}
The \textbf{gain} $g$ of the PMT is the mean number of electrons measured at the anode after the emission of one photoelectron from the photocathode, and it is the product of the individual dynode contributions. The latter is described by the secondary emission coefficient $\delta_i$, i.e. the average number of electrons emitted by the dynode $d_i$ for every incoming primary. The emission coefficient is directly proportional to the energy of the primary particle (up to a certain energy threshold), which makes the gain a function of the applied voltage. Nonetheless, there are loss processes, which are owing to the multiplier's geometry and positioning of the dynodes, determining the inter-dynode collection efficiency $c_i$, that is the fraction of electrons that arrive from dynode $d_{i-1}$ to $d_{i}$. For the first dynode $d_{0}$ it is the ratio between released photoelectrons at the photocathode and the ones that reach this dynode. Hence the gain can be expressed with
\begin{equation}
    g = \prod^{N}_{i=1}\delta_i c_i,
\end{equation}
where $N$ is the total number of dynodes in the multiplier system \cite{Flyckt}. Defining the ideal gain as $g_{\rm{ideal}} = \prod^{N}_{i=1}\delta_i$, it is also possible to specify a global \textbf{collection efficiency} for the PMT $CE$, where $CE = \dfrac{g}{g_{\rm{ideal}}}$, which is the fraction between the detected and emitted photoelectrons.
\par A high gain is usually desired, as this achieves a better signal to noise ratio because the photoelectrons can be distinguished better from electrical noise. Besides, assuming that the processes at all stages of the multiplier obey a Poisson distribution, the relative variance of the output charge will be proportional to $\frac{1}{g}$. Therefore a higher gain results in a lower variance of the measured charge. This allows a better reconstruction of the number of photons detected at a given time.









\subsection{Time response}
\begin{wrapfigure}{O}{0pt}
\centering
%\includegraphics[width=0.3\textwidth]{Figures/theory/timerespondPMT-1.png}
\includegraphics[scale=1.0]{Figures/theory/timerespondPMT.pdf}
\caption{Schematic of the photomultiplier time response parameters. }
\label{fig:timeresponse}
\end{wrapfigure}
The output pulse of a PMT can be characterised by manly three parameters. First there is the \textbf{rise time}, which is defined as the time needed for the pulse to rise from $\SI{10}{\percent}$ to $\SI{90}{\percent}$ of the peak height (see figure \ref{fig:timeresponse}). Then there is the \textbf{signal length}. This is usually described by the full width at half maximum (FWHM) of the PMT pulse, which is usually about 2.5 times the rise time \cite[p.~2-10]{Flyckt}. The latter is only true in the case of a Dirac delta input though. Light input pulses always have a certain width, which lengthens the signal.
\par The interval between the arrival of a light pulse onto the photocathode and the detection of its signal at the anode is called the \textbf{transit time}. This time depends on multiple factors and is in the order of several tens of nanoseconds, but it fluctuates from pulse to pulse. First, there is a geometric component for this variations, as the primary path lengths between the photocathode and the first dynode are different depending on the photoelectron's emission location. A second factor stems from the spread of the initial velocities of the photoelectrons, with respect to their initial energy and direction, which causes different transit time even for particles emitted from the same point on the photocathode \cite[pp.~4-11,4-12]{Flyckt}. This jitter of the pulse is called \textbf{transit time spread (TTS)} and is usually defined as the FWHM or the standard deviation of the transit time distribution. The TTS should be kept as low as possible, as it determines the time resolution of the photomultiplier.



\subsection{Dark rates and dark curent}
A PMT always will produce a measurable signal, even in total darkness. This output is called dark rate, in the case of a PMT operated in pulse mode, and dark current for a PMT in analogue mode. Depending on the nature of these dark signals, they can be classified into two categories: random and correlated noise. The origin of this background and their dependence on external factors will be the topic of this section.
\subsubsection{Random background:}
\noindent
    \par $\bullet\,$\textbf{Thermionic emission} is the main source of discrete component of the background and is caused by the spontaneous emission of electrons, which are collected by the multiplier system as a normal pulse. These electrons are released when their thermal energy surpass the work function of the photocathode $W_{\textrm{p-th}}$, which is in principle always possible, as the electrons in the photocathode follow Fermi statistics. Consequently, this effect is strongly temperature dependent, causing more dark signals the higher the temperature. Contrariwise, this effect can be strongly suppressed by cooling down the PMT, so that at negative temperatures its contribution to the total dark rates/current is neglectable. The current density of this process is $\propto T^2 \textrm{exp}(-W_{\textrm{p-th}}\cdot T^{-1})$, where $T$ is the temperature (Richardson's law \cite[p.~3-3]{Flyckt}).
    
    \par$\bullet\,$\textbf{Field emission} is the emission of electrons induced by the electrostatic field between the dynodes due to quantum tunnelling. As this effect does not depend much on temperature, it is also called \textit{cold emission}. It does depend on the applied voltage though, as the current density is $\propto \mathcal{E}^2\textrm{exp}(-W_{\textrm{d-th}}^{3/2}\cdot \mathcal{E}^{-1})$, where $\mathcal{E}$ represents the electric field strength and $W_{\textrm{d-th}}$ the work function of the dynodes (Fowler-Nordheim formula \cite{Fowler_1928}). Consequently, this is one of the principal factors that determine a practical limit to the set gain \cite[p.~3-4]{Flyckt}.
    
    \par $\bullet\,$\textbf{Leakage current} or also called ohmic leakage is as steady charge flow originated from the non-perfect insulating materials used in the tube. The insulation resistance is in the order of $10^{12}\,\Omega$, meaning that for a PMT with a voltage of $\SI{1}{kV}$, the leakage current may reach the nanoampere level (Ohm's law) \cite[70]{hamamatsuBook}.
    
    \par $\bullet\,$Other effects that contribute to the random noise are, for example, radiation from isotopes inside the tube glass envelope, such as $\ch{^{40}K}$, which produce scintillation and Cherenkov radiation. Besides there are external factors, such as cosmic rays and environmental gamma rays may be a background source depending on the experimental setup \cite[70,71]{hamamatsuBook}.
%\end{enumerate}
\subsubsection{Correlated background:}
\noindent
%Secondary electrons can escape the dynode system and produce light also via scintillation in the glass.
\par $\bullet\,$\textbf{Late pulses} occur after several nanoseconds after the initial pulse. These are caused mainly by photoelectrons that back-scatter from the first dynode or any supporting metal structure in its vicinity. This back-scattering process may be elastic, meaning that the signal is first emitted after the photoelectron hits again the dynode, or inelastic. In this case, the photoelectron releases some of its energy as it back-scatters at the dynode, emitting secondaries. If the photoelectron is detected upon the second impact, two pulses are detected from only one initial photoelectron. Another source of late pulses are photons emitted by the last dynodes and anode, which glow under electron bombardment in case of intense signals. In this case, a second pulse is measured after a time equal to the transit time of the PMT  \cite[435-439]{wright2017photomultiplier}\cite{Lubsandorzhiev_2006}.
\par $\bullet\,$\textbf{Afterpulses} are signals measured a few microseconds after the initial pulse. Photoelectrons in their path to the dynode system may ionise residual gases or release particles from the electrodes. This positively charged ion is accelerated toward the photocathode and releases several photoelectrons upon impact with it. The duration of this process depends on the ion mass and the size of the PMT \cite[440-445]{wright2017photomultiplier}.
\par $\bullet\,$\textbf{Pre-pulses} arise from electrons released by photons incident on the first dynode after crossing the photocathode without being absorbed. As one dynode multiplication stage is missing, the generated pulses are, depending on the set gain, around a factor 10 smaller than a typical single-photoelectron (SPE) pulse and arrive early by an interval equal to the time normally needed by electrons to hit the first dynode (a few nanoseconds)\cite[438]{wright2017photomultiplier}. As the photon producing the pre-pulse does not get absorbed at the photocathode, this phenomenon should not be seen strictly as correlated background (unless the PMT gets hit by several photons at the same time), but rather as a feature measurable in investigations of pulse time distributions, as the one in section \ref{sec:TTS}.
%\end{enumerate}










\section{Calibration and the photoelectron spectrum}
\begin{wrapfigure}{O}{0.4\textwidth}
\centering
\includegraphics[scale=1.]{Figures/theory/ChargeToSpectrum_labeled.pdf}
\caption{Qualitative sketch of the charge spectrum acquisition procedure. Based on \cite[50]{LEW}.}
\label{fig:chargeSpecTheory}
\end{wrapfigure}
Since there is an intrinsic spread between parameters among different PMTs, it is important to calibrate the used tube, i.e. to know the charge distribution generated by a certain number of photoelectrons. This can be achieved by saving the charge of PMT signals generated by a pulsed light source. This principle is illustrated in figure \ref{fig:chargeSpecTheory} for a PMT operated in pulse mode. The PMT signal is externally triggered by the reference signal of the light source and its charge integrated with a constant integration window. The result is a value proportional to the charge of the signal\footnote{If the measurement is done in pulse mode, the integration value has units in Weber and has to be divided by the input resistance of the oscilloscope in order to get the charge of the signal.}. As the gain is not constant, the distribution of the charge of one photoelectron will have a certain variance (blue line in figure \ref{fig:chargeSpecTheory}). If the intensity of the light source is low enough, sometimes no PMT signal will be measured after the trigger, and therefore the baseline will be integrated. The distribution of this charge is called the \textbf{pedestal} (red curve in figure \ref{fig:chargeSpecTheory}). 
\par The calibration is then done by deconvoluting the obtained charge spectrum using a realistic PMT response function. In this work, the method developed by Bellamy et al. \cite{bellamy} is used. Here the response function of the PMT $S_{\textrm{real}}(q)$ is described as convolution between an ideal PMT charge distribution  $S_{\textrm{ideal}}(q)$ and a background function $B(q)$:
\begin{equation}
\label{eq:Sreal}
S_{\textrm{real}}(q) = \int  S_{\textrm{ideal}}(\overline{q})\cdot B(q-\overline{q})\, \textrm{d}\overline{q}.
\end{equation}
\par The ideal response of the PMT $S_{\textrm{ideal}}$ is itself a convolution between the distribution of the number of photoelectrons emitted after one pulse of the light source and a function that describes the response of the dynode system. The number of photoelectrons $n$ released after each light pulse is represented by a Poisson distribution $P(n,\mu)$, with the average $\mu$, which depends on the intensity of the light source and the detection efficiency of the PMT. The response of the multiplication system for $n$ photoelectrons can be approximated by a Gaussian distribution $G_n(q)$ with mean $n Q_1$ and sigma $\sqrt{n}\sigma_1$, where $Q_1$ is the average charge at the PMT output when one electron is collected, and $\sigma_1$ is the corresponding standard deviation of this charge. Thus
\begin{equation} 
\label{eq:ideal}
\begin{split}
   S_{\textrm{ideal}} (q) & =  P(n,\mu) \otimes G_n(q)\\
   & = \frac{\mu^n \textrm{e}^{-\mu}}{n!} \otimes \frac{1}{\sqrt{2\pi n} \sigma_1}\exp(-\frac{(q-nQ_1)^2}{2n\sigma^2_1}) \\
   & = \sum^\infty_{n=0}\frac{\mu^n \textrm{e}^{-\mu}}{n!}\frac{1}{\sqrt{2\pi n} \sigma_1}\exp(-\frac{(q-nQ_1)^2}{2n\sigma^2_1}).
\end{split}
\end{equation}

It is to notice that this function approaches a delta peak for $n \rightarrow 0$ (non-existent pedestal), as no noise is included. The latter is described by the background response function $B(q)$. On the one hand, it represents low charge processes (such as the leakage current) that produce the pedestal, with a Gaussian of standard deviation $\sigma_0$. On the other hand, it also describes the additional charge measured in the signal stemming from discrete processes (e.g. thermionic and field emission) as an exponential decay with charge constant $ \alpha ^ {-1}$:
\begin{equation}
\label{eq:background}
    B(q) = \frac{(1-\omega)}{\sigma_0 \sqrt{2\pi}}\exp(\frac{-q^2}{2\sigma^2_0}) + \omega \cdot \Theta(q) \cdot \alpha \cdot \exp(-\alpha q),
\end{equation}
where $\omega$ is the probability for measuring discrete processes in the signal and $\Theta(q)$ is the step function. 
\par Solving equation \ref{eq:Sreal} results in a rather complex expression, which is difficult to be treated as a fitting function. For cases where the discrete background intensity is low, the background of the nonpedestal part of the spectrum can be treated as an effective additional charge $Q_{\textrm{sh}} = \omega \alpha^{-1}$, which means the background response function for $n \geq 1$ is reduced to
\begin{equation*}
B(q) \lvert_{n \geq 1} = \frac{1}{\sigma_0 \sqrt{2\pi}}\exp(\frac{-(q-Q_0-Q_{\textrm{sh}})^2}{2\sigma^2_0}),
\end{equation*}
where $Q_0$ es the mean charge of the pedestal. In this case equation \ref{eq:Sreal} can be approximated to
\begin{equation}
\label{th:fitPMTfunction}
    S_{\textrm{real}} \approx B(q-Q_0)\cdot \exp(-\mu) +  S_{\textrm{ideal}} (q-Q_0-Q_{\textrm{sh}})\lvert_{n \geq 1},
\end{equation}
which is equation \ref{eq:background} modulated by the probability of measuring no photoelectrons $P(0,\mu) = \exp(-\mu) $ describing the pedestal, while the nonpedestal part is represented by the ideal PMT response from equation \ref{eq:ideal} shifted by the pedestal position $Q_0$ and the mean charge contribution from discrete background processes $Q_{\textrm{sh}}$.
\par Once the calibration is done, the gain $g$ of the PMT can be derived via
\begin{equation}
\label{th:gaincalculation}
    g = \frac{Q_1-Q_0}{e},
\end{equation}
where $e$ is the elementary charge.