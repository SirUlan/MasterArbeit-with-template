\section{The IceCube Neutrino Observatory}
\begin{wrapfigure}{O}{0.6\textwidth}
    \centering
    \includegraphics[width = 0.55\textwidth]{Figures/theory/edited-1.png}
    \caption{Diagram of the IceCube Observatory at the South Pole. Figure courtesy of the IceCube Collaboration.}
    \label{fig:IceCubeDetector}
\end{wrapfigure}
IceCube is the first kilometre-scale neutrino detector and is located at the South Pole, starting at $1450$ metres below the surface, with a total of $5160$ optical sensors for detecting the Cherenkov light produced by particle interactions \cite{Aartsen_2017}. A sketch of the Observatory is shown in figure \ref{fig:IceCubeDetector}. The primary array consists of 78 strings (cables) with a length of $\SI{1}{km}$ distributed over a square kilometre, each holding $60$ digital optical modules (DOMs) in $\SI{17}{m}$ vertical intervals. The horizontal inter-string distance is $\SI{125}{m}$, which allows the investigation of neutrinos in an energy range between $\SI{100}{GeV}$ and $\mathcal{O}(\textrm{PeV})$. The DOMs incorporate a 10-inch-diameter photomultiplier tube (PMT) facing down, inside a glass pressure vessel, with circuit boards that allow near-autonomous operation. PMTs are extremely sensitive devices that are capable of measuring single photons. These are described later in this thesis in chapter \ref{ch:PMTs}.
\par Eight additional strings separated, on average, by about $\SI{72}{m}$ are located at the centre of the detector. The modules are deployed at depths ranging from $\SI{2100}{m}$ to $\SI{2450}{m}$, with an inter-sensor spacing of only $\SI{7}{m}$. This more densely instrumented sub-detector is called \textit{DeepCore} and is optimised to detect neutrinos of energies in the order of tens of $\textrm{GeV}$. 
\par On the surface is located \textit{IceTop}, which consists of water tanks near the top of each string, with two DOMs each, forming a square kilometre air shower array. This can be used as a veto against atmospheric neutrinos and as a detector for cosmic rays in the $\SI{300}{TeV}$ to $\SI{1}{EeV}$ region. All the cables from the detector are routed to the IceCube Laboratory, which is the operation building of the detector located at the surface in the centre of the array. 
\par The IceCube's commissioning was completed 2011 and since then it has already discovered a flux of high-energy neutrinos of cosmic origin \cite{Kopper2013}. An overview of the recent findings of IceCube with regard to astrophysical neutrinos and cosmic rays can be found in \cite{Aartsen_2017_Results}. Nevertheless, the modest number of events limits the efficiency of this observatory. Therefore, a substantial expansion of the detector is sought, \textit{IceCube-Gen2}, aiming at an instrumentation of up to $\SI{10}{km^3}$ of glacial ice with novel technology \cite{1412.5106}.

\subsection{IceCube-Gen2}
\begin{wrapfigure}{O}{0.51\textwidth}
    \centering
    \vspace{20pt}
    \includegraphics[width = 0.51\textwidth]{Figures/theory/gen2.png}
    \caption{Benchmark detector string layout with a string-to-string distance of about $\sim \SI{240}{m}$. The new $120$ strings surround the more densely instrumented IceCube detector. Other geometries and sets of spacing are under consideration. Figure taken from \cite{1412.5106}.}
    \vspace{5pt}
    \label{fig:IceCubeGen2Detector}
\end{wrapfigure}

The benchmark detector layout for IceCube-Gen2 considers the extension of $120$ new strings, aiming for the detection of high-energy neutrinos, sometimes called \textit{HEA} (High-Energy Array)\cite{1412.5106}.
\par Following the calibration and the building of IceCube, the optical properties of the glacial ice are now known in detail over a range of great depth. The absorption length of the ice for Cherenkov light is larger than initially assumed and exceeds $\SI{100}{m}$-$\SI{200}{m}$, depending on the depth. This enables the instrumentation of considerably larger volumes with lower string densities than observed at IceCube. In this context, spacings between $\SI{240}{m}$ and $\SI{300}{m}$ are being considered. An example string layout is illustrated in figure \ref{fig:IceCubeGen2Detector}. The larger spacing results in a higher energy threshold above $\sim \SI{50}{TeV}$, but it does not represent a loss of astrophysical neutrino signal. Owing to the size of the detector, the HEA should be capable of measuring neutrinos up to the $\mathcal{O}(\textrm{EeV})$ level.

\par Besides the HEA, an extension of DeepCore is being proposed: \textit{PINGU}, a dense array for the detection of low-energy neutrinos, which will target precision measurements of the atmospheric oscillation parameters and the neutrino mass hierarchy \cite{1412.5106}. An expansion of the IceTop array is also considered in the scope of IceCube-Gen2.

\par The detector sensitivities for neutrino point sources scale approximately with the square-root of the increase in its cross-sectional area, but linearly with the angular resolution. The latter can be further improved with different approaches, e.g.\ advanced reconstruction methods and a more detailed model of the ice properties. In this respect, new optical sensors are being proposed, which will increase not only the angular resolution but also the overall efficiency.

\par Two new concepts for optical sensors are being studied. On the one hand, the employment of wavelength-shifting and light-guiding techniques for an increased sensitivity to UV photons and hence the Cherenkov spectrum is represented by the WOM project (Wavelength-shifting Optical Module) \cite{WOM}. On the other hand, the segmentation of the active area of the modules using several PMTs is being investigated. There are two approaches for this: the DEgg, which consist of two 8'' PMTs facing downwards and upwards \cite{DEGG} and the mDOM (multi-PMT Digital Optical Module), which features multiple small photomultiplier tubes \cite{mDOM}. A more detailed introduction to the latter is presented in the next section.

\subsection{The multi-PMT digital optical module}
\begin{wrapfigure}{r}{0pt}
    \centering
    \vspace{-40pt}
    \includegraphics[scale=1.0]{Figures/theory/mDOM_exploded_view-1.pdf}
    \caption{Explosion view of the mDOM with its main components. Courtesy of the IceCube Collaboration.}
    \label{fig:mDOM}
     \vspace{20pt}
    \includegraphics[scale=1.0]{Figures/theory/DOM.pdf}
    \caption{Rendering of the current IceCube DOM. Courtesy of the IceCube Collaboration.}
    \label{fig:DOM}
    \vspace{-20pt}
\end{wrapfigure}
The mDOM consists of $24$ 3-inch-diameter PMTs facing multiple directions, as illustrated in figure \ref{fig:mDOM}. In contrast, the current IceCube DOM is shown in figure \ref{fig:DOM}. The mDOM PMTs are mounted surrounded by a reflector, which increases the sensitive area of the module \cite[p.~160]{LEW}, on a 3D-printed holding structure. An active base is attached to the end of the PMT and provides the latter with high-voltage and reads out its output. The signals are digitised by the mainboard, located in the equatorial plane at the centre of the module. All these components are enclosed by a glass pressure vessel that protects the module against rough external conditions. Between the vessel and the internal components of the module, there is a layer of optical gel, which acts as an optical coupler, thereby preventing light reflections due to the different refractive indices of glass and air.
\par The segmentation of the sensitive area results in a range of advantages in comparison to the conventional single-PMT DOM, such as:\\
\noindent
\hspace*{0.3cm} $\bullet$ a larger effective area (hypothetical geometrical area of the module assuming $\SI{100}{\percent}$ detection efficiency), since $24$ 3-inch PMTs provide a larger total photocathode surface than one 10-inch PMT. Moreover, the mDOM has a near homogeneous $4\pi$ angular acceptance in contrast with the DOM, where the sensitivity for the downward light is almost zero.\\
\hspace*{0.3cm} $\bullet$ intrinsic directional sensitivity, as the PMT orientation entails information about the direction of the detected light. In the case of the DOM, this is very limited due to the large field of view of its PMT.\\
\hspace*{0.3cm} $\bullet$ better photon-counting, as the photons are detected among the different PMTs, their numbers and arriving times can be reconstructed more easily, unlike a multi-photon waveform from a single photomultiplier. Consequently, multiple PMTs also feature a better performance with regard to saturation.\\
\hspace*{0.3cm} $\bullet$ the possibility for the development of reconstruction algorithms based on local coincidences, i.e.\ correlated signals in several PMTs from a single module. This enables novel methods for background suppression, PMT self-calibration \cite[p.~215]{LEW} and a higher efficiency and better reconstruction of low-energetic signals, such as supernova neutrinos, as shown in recent studies \cite{cris}.\\

\begin{figure}[t]
\centering
\begin{minipage}{.5\textwidth}
  \centering
  \includegraphics[scale = 1.0]{Figures/theory/glass-comparison.pdf}
\end{minipage}%
\begin{minipage}{.5\textwidth}
  \centering
  \includegraphics[scale = 1.0]{Figures/theory/gel-comparison.pdf}
\end{minipage}  
\caption{\textbf{Left:} Wavelength dependence of the transmittance of the glass brand Vitrovex and Benthos in dashed lines. The unmodified and the transmitted Cherenkov spectrum is shown with solid lines. \textbf{Right:} Wavelength dependence of the transmittance of light of the gel brand Wacker, Chiba and QSI are shown in dashed lines. The unmodified and the transmitted Cherenkov spectrum are shown with solid lines. The grey region represents the range, where there is no light transmission due to the absorption of the glass vessel.}
\label{fig:TransmissionComparison}
\end{figure}

\par Aside from the technological improvements being made with the active constituents of the module, the optical properties of the passive components can also be enhanced. From this perspective, alternative brands for the vessel glass and the optical gel are being considered. 
\par The main function of the pressure vessel is to endure pressures up to $\SI{700}{bar}$. Owing to the temperature gradient of the glacier, the water freezes from the top downwards, resulting in a high-pressure excess of up to $\SI{690}{bar}$ \cite{Aartsen_2017}. Consequently, the vessel has to be quite thick (over $\SI{1}{cm}$ at the current DOM), which makes the transparency of the material very important as a portion of the light will be absorbed in it. The current operating neutrino telescopes use vessels made of borosilicate glass. The transmission spectrum of this material exhibits a wavelength cutoff at around $\SI{300}{nm}$, depending on the glass thickness and the manufacturer. In this respect, the Vitrovex brand, the standard glass used by Nautilus GmbH., presents an improved transmission in the UV region compared to the one currently used in the DOMs called Benthos, as illustrated in figure \ref{fig:TransmissionComparison}. For the mDOM's vessel, which has a wall-thickness of $\SI{13}{mm}$, this difference results in $\sim \SI{13}{\percent}$ more Cherenkov photons, in the range of $\SI{300}{nm}$ to $\SI{700}{nm}$, being detected. 
\par The silicon-based optical gels tend to feature better transparency in the UV range, reducing their influence on the overall performance of the module compared to the glass. Nevertheless, brands offering higher transmittance to the gel currently used in IceCube (QGel 900 from QSI) are being considered for the mDOM, such as SilGel 612 from Wacker and the optical gel produced in Chiba, Japan, for the DEgg module. Their transmission at $\SI{5}{mm}$\footnote{The thickness of the gel-layer in front of the PMT varies between $\SI{2}{mm}$ and $\SI{25}{mm}$ depending on the location.} thickness is shown in figure \ref{fig:TransmissionComparison}. As the transmission cutoff of the glass is located at larger wavelengths than the one of the gels, the difference regarding the number of transmitted photons in the interval between $\SI{300}{nm}$ and $\SI{700}{nm}$ is of only a few percentage points.
\par However, while aiming to choose the best material, there are other factors to be taken into account. Different brands exhibit various levels of trace radiation, which is an important aspect of the measured background, as decay products can produce Cherenkov light as well as scintillation photons. This thesis focuses on this topic and investigates the influence of the radioactive isotopes presented in the material on the overall noise of the mDOM and DOM, since a detailed understanding of background production mechanisms is crucial for the signal processing and reconstruction. In this context, both glass brands, Benthos and Vitrovex, and the QSI QGel 900 and Wacker SilGel 612 gel are investigated\footnote{Hereinafter in this work, the optical gels QGel 900 and SilGel 612 will be referred to by the name of their manufacturers, QSI and Wacker, respectively.}. The optical gel used in Chiba was unfortunately not available for measurements. 