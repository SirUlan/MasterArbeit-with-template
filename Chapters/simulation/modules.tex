\chapter[Simulation of dark rates in optical modules]{\LARGE{Simulation of dark rates in optical modules}}
\label{ch:moduleSimulation}


In the previous chapters, the amount of isotopes inside the glass and gel samples was determined and the light produced by the radioactive decays was characterised. With this information, the background produced by the vessel glass and the gel inside the optical modules can be simulated. In this context, the following sections will describe the expected noise separating it between correlated and uncorrelated background, thus presenting its time distribution. Furthermore, as measured in chapter \ref{ch:scintParameters} the scintillation properties are highly temperature dependent and the expected background will also be introduced as a function of the temperature.
\par The time distribution of the noise is usually presented in a log$_{10}$($\Delta t$) diagram. As this kind of plots may not be a common way of displaying data, section \ref{sec:intro2log10} gives a small introduction on the interpretation of correlated and uncorrelated noise in the log$_{10}$($\Delta t$) representation.
\par Section \ref{sec:simDOM} and \ref{sec:simmDOM} present some simulation studies for the DOM and mDOM, respectively. Although the motivation of this work is to estimate the background from scintillation for the mDOM, it is also important to study the case of the current IceCube DOM, since for the latter there is experimental data available. Hereby it is possible to estimate, how precise the prediction for the mDOM is.


\section{Introduction to log$_{10}$($\Delta t$) diagramms}
\label{sec:intro2log10}
\begin{figure}[H]
  \centering
  \begin{minipage}[b]{0.49\textwidth}
    \includegraphics[scale=1.0]{Figures/simulation/linearX.pdf}
  \end{minipage}
  \hfill
  \begin{minipage}[b]{0.49\textwidth}
    \includegraphics[scale=1.0]{Figures/simulation/logX.pdf}
  \end{minipage}
  \caption{The probability density function of the waiting time $t$ for a background event stemming from random noise of rate $\SI{50}{s^{-1}}$ and $\SI{500}{s^{-1}}$. \textbf{Left:} with linear and \textbf{right:} with a logarithmic representation of the time.}
  \label{fig:linearAndLog}
\end{figure}
Random noise of average rate $\mu$ (hits per second) is a Poissonian process since the probability for an event is independent of the past. Let $D$ be the waiting time for the detection of a hit. The probability that $D$ is larger than a given time $t$ is
\begin{equation}
\label{eq:probab}
    P(D>t) = \frac{(\mu \cdot t)^0 e^{(-\mu\cdot t)}}{0!} = \exp{(-\mu\cdot t)}.
\end{equation}
Hence, the cumulative density function (CDF) $F(t)$ is given by 
\begin{equation}
F(t) = P(D\leq t) = 1 - \exp{(-\mu\cdot t)} 
\end{equation}
and its  probability density function (PDF) $f(t)$ by the derivative of the CDF 
\begin{equation}
f(t) = \frac{d}{dt} F(t) = \mu\cdot\exp{(-\mu\cdot t)} .
\end{equation}
Therefore, the time difference between subsequent hits from random background is expected to produce an exponential decay with a slope equal to the average rate of the noise (assuming a logarithmic scale in the ordinate). As an example, the $\Delta t$ distribution for random background of rate $\SI{50}{s^{-1}}$ and $\SI{500}{s^{-1}}$ is shown in the left part of figure \ref{fig:linearAndLog}. However, this is only valid for a linear application of the abscissa. If the time is represented by its logarithm (applying the substitution $t=10^x$), the CDF of the distribution in \ref{eq:probab} is equal to  $F_l(x) = 1 - \exp{(-\mu\cdot 10^x)} $ and thus the PDF is given by
\begin{equation}
\label{eq:PDFlog}
    f_l(x) = \frac{d}{dx} F_l(x) = \mu\cdot 10^x \cdot \log{(10)} \cdot \exp{(-\mu\cdot 10^x)}.
\end{equation}
\par For comparison, the random background of $\SI{50}{s^{-1}}$ and $\SI{500}{s^{-1}}$ is given in this representation in the right part of figure \ref{fig:linearAndLog}. Although the PDF for the logarithm of the time is more complex, it has some advantages over the linear representation, e.g.\ the maximum of equation \ref{eq:PDFlog} lies at $x = -\log_{10}{(\mu)}$, and thus the rate can be more easily visualised than in the linear case. Moreover, the benefits of a log$_{10}$($\Delta t$) representation manifest itself in background that combines both, correlated and uncorrelated signals, like the case of a PMT or the optical modules, as these processes can be better differentiated.
\begin{figure}[H]
  \centering
  \begin{minipage}[b]{0.49\textwidth}
    \includegraphics[scale=1.0]{Figures/simulation/HVDT-Benthos-linear.pdf}
  \end{minipage}
  \hfill
  \begin{minipage}[b]{0.49\textwidth}
    \includegraphics[scale=1.0]{Figures/simulation/HVDT-Benthos-log.pdf}
  \end{minipage}
  \caption{Measured time between subsequent signals $\Delta t$ of a PMT inside a Benthos half vessel at $\SI{-50}{\celsius}$. Also shown with a yellow line is the fit of the Poisson expectation for the uncorrelated part of the background. \textbf{Left:} histogram of the $\Delta t$ distribution. \textbf{Right:} histogram of the logarithm of the $\Delta t$ data. Here, also the simulated distribution stemming from scintillation is shown with a green line.}
  \label{fig:HVtime}
\end{figure}

\par In order to give an example of correlated and uncorrelated background, the time of subsequent hits from a PMT inside the Benthos half pressure vessel was measured. The signal of the PMT is split with a T-adapter and sent to two different oscilloscopes\footnote{A PicoScope 6404C and a Lecroy Waverunner 640ZI}. One of them saved the time difference of pulses inside waveforms of $\SI{100}{\mu s}$ and the second oscilloscope the $\Delta t$ between triggers, which has a lower boundary as the rearming of the trigger takes around $\sim \SI{1}{\mu s}$. Combining the data from both devices into a histogram yields the results shown in figure \ref{fig:HVtime}(left). Since there is correlated noise originating from the PMT (afterpulsing) and the vessel (coincident hits from scintillation of radioactive decays), the $\Delta t$ distribution deviates from the Poissonian one showing a larger number of counts for very short time differences. Taking the logarithm of the data and graphing it as a histogram yields the results shown in the right part of figure \ref{fig:HVtime}. In this case, the correlated background can be better distinguished from the Poissonian expectation and the process causing the different structures can be deduced. In section \ref{ch:lifetime}, the most probable time delay for late afterpulsing of the PMT was found to be at around $\SI{3}{\mu s}$, while for the early afterpulsing at $<\SI{50}{ns}$ (see figure \ref{fig:TimeDistributionCorr}). These times correspond to the peaks seen at  $\sim \SI{-5.5}{log_{10}(s)}$ and $\sim \SI{-7.6}{log_{10}(s)}$ in figure \ref{fig:HVtime} (right).
\begin{wrapfigure}{o}{0pt}
\centering
\includegraphics[scale=1.01]{Figures/simulation/HVDT-Benthos-20.pdf}
\caption{Measured time between subsequent signals of a PMT inside a Benthos half vessel at $\SI{20}{\celsius}$.}
\label{fig:Benthos20Deg}
\end{wrapfigure}
\par The expected $\Delta t$ distribution from luminescence for this setup was simulated with Geant4\footnote{The next section \ref{sec:simDOM} explains how these distributions are simulated.} using the scintillation parameters measured in chapter \ref{ch:scintParameters} and the mass-specific activity of the isotopes inside the half vessel measured in chapter \ref{ch:GammaSpectroscopy}. The result is also presented in figure \ref{fig:HVtime}. Here the amplitude of the simulated histogram was modified in order to fit the measured curve. It is noticeable that luminescence accounts for most of the correlated background under $\SI{-5}{log_{10}(s)}$, but that it also contributes to the uncorrelated peak. This part stems from the time differences between photons from different decays, which is random and determined by the activities of the isotopes. The sum of several Poissonian processes with a rate $\lambda_i$ results in a single Poisson distribution of rate $\sum_i\lambda_i$, and thus only one Peak from the uncorrelated background is measured. Furthermore, the simulated distribution does not account for the correlated noise measured at longer time differences at around $\SI{-4}{log_{10}(s)}$.
\par For comparison figure \ref{fig:Benthos20Deg} shows the same measurement done at $\SI{20}{\celsius}$. Here, the effects of the luminescence background are almost unnoticeable, as the yield decreases with higher temperatures (see section \ref{sec:YieldwSource}), leaving only the dark rate from the PMT. As the thermionic noise from the PMT is much higher at this temperature, the Poissonian peak shifts towards shorter times. However, it is to notice that the correlated background measured between $\sim \SI{-4.7}{log_{10}(s)}$ and $\sim \SI{-3}{log_{10}(s)}$ in figure \ref{fig:HVtime}, which cannot be explained by luminescence parametrised with the result from chapter \ref{ch:scintParameters}, also does not appear in the measurement. This suggests that this background must be also caused by scintillation, which would mean that there are one or more extra time constant in the order of $\textrm{ms}$. The exponential decays from lifetimes in this order of magnitude would be seen as a constant and could not have been extracted from the waveforms of $\SI{100}{\mu s}$ used in the measurements of section \ref{ch:lifetime}.
\par Altogether, log$_{10}$($\Delta t$) diagrams are a helpful resource for analysing correlated and uncorrelated noise. In this context, in the next two sections the contribution of the scintillation light to the overall background of the modules will be simulated and discussed in more detail.











\section{Current IceCube DOM}
\label{sec:simDOM}






\begin{figure}[H]
  \centering
  \includegraphics[scale=1.]{Figures/simulation/Stanisha-Simu.pdf}
  \caption{Simulated time between subsequent hits in log$_{10}$($\Delta t$) for a DOM in ice at $\SI{-35}{\celsius}$. Both measured sets of isotope activities BAS-1 (table \ref{table:actCanada}) and BAS-2 (table \ref{tab:actMue}) were used. The error bars only consider the statistical uncertainty and are smaller than width of the line.}
  \label{fig:DOM-SIMU}
\end{figure}
The time distribution of the background and its total rate depends on the amount of radioactivity in the glass. There were two sets of isotope activities measured for the Benthos glass in chapter \ref{ch:GammaSpectroscopy}, which will produce different results. For the sake of simplicity, the set of activities for the Benthos samples summarised in table \ref{table:actCanada} and \ref{tab:actMue} will be referenced as ``BAS-1'' and ``BAS-2'' respectively.

\par In order to simulate the time distributions of the background caused by radioactive decays in the DOM surrounded by ice, the decay of the natural series and the \ch{^{40}K} are simulated $100000$ times. Here, the scintillation parameters at $\SI{-35}{\celsius}$ were used and only luminescence of the glass was taken into consideration. The effects of gel luminescence are studied later in section \ref{sec:gelscint}. Since only the scintillation yield for $\alpha$-particles was determined, it is assumed that electrons exhibit a yield $9.5$ times larger, following the results for another kind of glass \cite[p.~255]{knoll2010radiation}. An output file is created for each isotope, where one line of the file contains the hit time of the detected photons from a single decay. If in an event no photon was detected, the line is left empty. Thus, each file contains $100000$ lines. Hereby, the time distribution of the noise can be constructed by mixing the decay results with a Python code. The number of decays of a specific isotope during one second is sampled randomly from a Poisson distribution with a mean equal to the isotope activity (assuming a pressure vessel of $\SI{9.07}{kg}$\footnote{Teledyne Benthos, 2013. Deep Sea Glass Spheres. Available online at: \url{http://www.teledynemarine.com/Lists/Downloads/Flotation_Spheres_Data_Sheet_2013_lo.pdf} (Last accessed 18 December 2017).}). These decays are ordered randomly in time and for each, one line from the isotope's file is sampled, containing the hit times of the detected photons. This is done consecutively $600$ times with all isotopes, resulting in a time array corresponding to $\SI{10}{minutes}$ of background. The resulting $\Delta t$ distribution is presented in figure \ref{fig:DOM-SIMU} for BAS-1 and BAS-2.


\begin{wrapfigure}{o}{0pt}
\centering
\includegraphics[scale=1.0]{Figures/simulation/cerScintDT.pdf}
\caption{Time between subsequent hits considering only Cherenkov photons (blue line), only scintillation (yellow line) and both (green line). The statistical uncertainty is not depicted, as it is smaller than the line width.}
\label{fig:cerscintTDF}
\end{wrapfigure}
\par Since in BAS-2 the activity of the $\ch{^{238}U}$ chain is $\sim \SI{42}{\percent}$ larger, the expected background for this set is higher and the maximum of the Poissonian peak lies also at shorter time intervals.
Figure \ref{fig:cerscintTDF} shows the fastest hits in the interval $-10$ to $\SI{-8}{log_{10}(s)}$, which correspond to time differences between $\SI{0.1}{ns}$ and $\SI{10}{ns}$. Here are depicted the results of the simulation of BAS-1 separating the Cherenkov and scintillation photons. It can be seen that the principal source of hits in this region originate from Cherenkov light and that the contribution of the luminescence noise decreases exponentially with shorter times. However, it has to be noticed that in the simulation either TTS nor the PMT pulse length was taken into account, and therefore most of this time differences cannot be measured. The pulse length (FWHM) of the PMT used inside DOMs (Hamamatsu R7081-02) is about $\SI{7.5}{ns}$ \cite{Abbasi:2010vc}, and hence $\Delta t$ shorter than $\SI{-8}{log_{10}(s)}$ will be strongly suppressed, as is the case in the measurements shown in figures \ref{fig:HVtime} and \ref{fig:Benthos20Deg}.
\begin{figure}[b!]
  \centering
  \begin{minipage}[b]{0.49\textwidth}
    \includegraphics[scale=1.0]{Figures/simulation/differentIsotopeActivities.pdf}
  \end{minipage}
  \hfill
  \begin{minipage}[b]{0.49\textwidth}
      \includegraphics[scale=1.0]{Figures/simulation/differentYieldsDT.pdf}
  \end{minipage}
  \caption{Relative change of the background time distribution by varying the isotope activity in the glass (\textbf{left}) and the scintillation yield (\textbf{right}). The statistical uncertainty is not depicted, as it is smaller than the line width.}
  \label{fig:yieldAdependency}
\end{figure}
\par In order to compare the shapes of the time distribution curve from both isotope activity sets, these were normalised in the left side of figure \ref{fig:yieldAdependency}. Here are also shown the results for the case of a vessel with $\SI{50}{\percent}$ and $\SI{150}{\percent}$ of the isotope activities from BAS-1. The peak from luminescence does not change with a variation of the amount of radioactivity and only the position of the Poissonian part moves towards higher or lower rates. This is to be expected, as the scintillation peak is produced by photons from the same decay and therefore its shape is independent of the radioactive decay rate. On the right side of figure \ref{fig:yieldAdependency} is shown the normalised distribution for three different scintillation yield values. The results shown in figure \ref{fig:cerscintTDF} are for the scintillation parameters at $\SI{-35}{\celsius}$, viz.\ a yield of $\SI{34}{MeV^{-1}}$. Comparing the results for this yield value with a lower of $\SI{20}{MeV^{-1}}$ and a larger one of $\SI{50}{MeV^{-1}}$ it can be noticed that the relative height of the luminescence peak in regards to the Poissonian part changes. With an increase of the yield, the number of hits from a single decay is larger, while only the first and last detected photon from the decays contribute to the uncorrelated part, reducing its relative contribution to the overall distribution. A similar reasoning can be applied for the region of the distribution stemming from the Cherenkov radiation, since this is independent of the yield. Although the rate of Cherenkov photons detected is the same, the Cherenkov part of the distribution has a larger relative contribution to the distribution if the yield is smaller. Furthermore, it is noteworthy that the shape of the scintillation part also changes with the yield, having fewer counts at shorter $\Delta t$. This may be a similar effect as the one encounter in the measurements of the lifetime (see section \ref{ch:lifetime}). The first photon detected from a decay does not contribute to the correlated part of the distribution, but to the Poissonian peak. Therefore the number of detected $\Delta t$ corresponding to the shortest lifetime is lower than the emitted one. This effect is further enhanced when the number of photons per decay decreases.
\begin{figure}[h!]
  \centering
\includegraphics[width = 0.87\textwidth]{Figures/simulation/Stanisha.png}
  \caption{Time distribution of the DOM background in IceCube. In order to make this histogram, the HitSpool and FRT data of all modules were combined. Figure taken from \cite{Stanisha}.}
  \label{fig:Stanisha}
\end{figure}
\par In IceCube the time distribution of the background can be studied separately for each DOM with the HitSpool data\footnote{HitSpooling is a standard DAQ feature that buffers the raw data stream of the modules around supernova candidate triggers. Further information can be found in \cite{Heereman}.}. This data is however limited to the dead time of the DOM mainboard, with a minimum time between triggers of $\SI{2.45}{\mu s}$ depending on the readout sequence \cite{Heereman}. This precludes the study of most of the correlated noise. Nevertheless, a glimpse of this region was provided in \cite{Stanisha}, where the HitSpool distribution was combined with FRT\footnote{At Fixed Rate Trigger (FRT) events, $\SI{10}{ms}$ of raw data is saved every $\SI{30}{s}$ for every DOM.} data. This data set does not have enough statistics in order to construct a time distribution of the background for each DOM and thus the information of every module was combined in a single histogram. The results obtained with this approach are shown in figure \ref{fig:Stanisha}. Here roughly the same features can be found as in the measurement shown in \ref{fig:HVtime}, including an afterpulsing peak at around $\sim \SI{-5.1}{log_{10}(s)}$ (as the R7081-02 PMT is larger than the Hamamatsu R12199-02, it takes longer for the ions to reach the photocathode).
\par It is noteworthy that the Gaussian fitted to the correlated noise in figure \ref{fig:Stanisha} is very similar to the results shown in figure \ref{fig:DOM-SIMU}. Excluding the Cherenkov contribution at very short time differences, the simulated distribution exhibits a maximum at around $\SI{-6}{log_{10}(s)}$ and decreasing quite symmetrically until about $\SI{-8}{log_{10}(s)}$ and  $\SI{-4}{log_{10}(s)}$. Nevertheless, the distribution in figure \ref{fig:Stanisha} also features a long timescale correlated peak centred at $\SI{-4}{log_{10}(s)}$, which is not explained by the scintillation parameters measured in this thesis. This was seen also in the results of the measurement done with the Benthos half vessel in last section (see figure \ref{fig:HVtime}) and probably stems from long-lived luminescence transitions.
\par With the Geant4 simulation, it is also possible to calculate the temperature-dependence of the background rate cause by radioactive decays in the pressure vessel. First, the decay of the three decay chains and $\ch{^{40}K}$ is simulated $2\times 10^6$ times each, once for every scintillation parameters at $T_i = [-15,-25,-35,-45,-50]\,^\circ\rm{C}$. As the results of these calculations are to be compared with the HitSpool data, in the simulation is set a dead time of $\SI{2.45}{\mu s}$ between hits. Also, the PMT waveform and QE are considered, as explained in chapter \ref{ch:g4}. The output of the simulation provides the average number of photons detected $H(T_i,I)$ for the temperature $T$ and isotope $I$. Thus, the rate $R$ at $T_i$ is calculated with
\begin{equation}
    R(T_i) = \sum_I H(T_i,I)\times A_I \times m, 
\end{equation}
where $A_I$ is the mass-specific activity of the isotope $I$ and $m=\SI{9.07}{kg}$ is the mass of the pressure vessel. As the scintillation yield $y(T)$ was determined for all temperatures between $\SI{-50}{\celsius}$ and $\SI{-15}{\celsius}$ in $\SI{1}{\celsius}$ steps, the rate can be calculated at these same temperatures by linear interpolation. Therefore, the rate $R(T)$ at the temperature $T$, when $T_i>T>T_{i-1}$, is given by
\begin{equation}
    R(T) = R(T_{i-1})+ (y(T)-y(T_{i-1}))\times \frac{R(T_i)-R(T_{i-1})}{y(T_i)-y(T_{i-1})}.
\end{equation}
This rate can be furthermore separated between correlated and uncorrelated noise by constructing the $\Delta t$ distribution for every $T_i$ and fitting the Poissonian peak with equation \ref{eq:PDFlog}. The percentage of uncorrelated noise can be then calculated by integrating this fit and dividing it by the total number of counts of the distribution. The results of this approach are shown in figure \ref{fig:ratesDOM}. Here is also depicted the rate from the HitSpool data, where each point represents the average rate of 12 DOM layers from 78 strings \cite{Aartsen_2017}. 
\begin{figure}[t!]
  \centering
  \begin{minipage}[b]{0.49\textwidth}
    \includegraphics[scale=1.0]{Figures/simulation/RatesDOM-Corr.pdf}
  \end{minipage}
  \hfill
  \begin{minipage}[b]{0.49\textwidth}
      \includegraphics[scale=1.0]{Figures/simulation/RatesDOM-UCorr.pdf}
  \end{minipage}
  \caption{Temperature-dependence of the background rate caused by radioactive decays inside the glass of the DOM, simulated using the two sets of isotope activities BAS-1 (blue) and BAS-2 (yellow). The large uncertainties are caused by the systematical error of the yield and isotope activities. For comparison is also depicted the DOM background rate calculated with the HitSpool data (taken from \cite{Aartsen_2017}). The latter also includes the dark rate of the PMT. \textbf{Left:} rate of the correlated and \textbf{right:} of the uncorrelated noise. }
  \label{fig:ratesDOM}
\end{figure}
 \par The simulated temperature dependence shown by the correlated part of the background (left side of figure \ref{fig:ratesDOM}) is in good agreement with the experimental data. This behaviour is a direct result of the dependency shown by the scintillation yield of the glass. However, the absolute values of the rate do not coincide with each other, as the set of isotope activities BAS-2 result in a higher, while the one of BAS-1 in a lower rate compared to the HitSpool data. Since in this thesis the amount of radioactivity of only two Benthos samples was measured, it is not known how much the activities vary between the pressure vessel of different DOMs. As the experimental data is the average of several modules, it could be asserted that the mean isotope activity lies between the sets measured in this work, assuming that every scintillation parameter is correct. Anyway, the correlated rate of the HitSpool data also includes the afterpulsing of the PMT, thus the rate caused by radioactive decays is a bit lower than the one shown in figure \ref{fig:ratesDOM}.
\par The analysis of the uncorrelated part, shown in the right side of figure \ref{fig:ratesDOM} is more complex, as most of the dark rate of the PMT contributes to the Poissonian noise. The influence of the PMT can be clearly seen, as the rate starts to increase with higher temperatures at around $\SI{-20}{\celsius}$ due to the thermionic emission. However, the PMT dark rate is not known with great precision, as the measurements done with bare PMTs can not completely reproduce the optical coupling inside the DOM. In \cite{Abbasi:2010vc}, the PMT dark rate was determined to be close to $\SI{300}{s^{-1}}$ in the $\SI{-40}{\celsius}$ to $\SI{-20}{\celsius}$ range. In this measurement, an artificial dead time of $\SI{6}{\mu s}$ was added, which suppressed around the half of all afterpulses. If the PMT dark rate in the DOMs is of the same order, then the uncorrelated part of the HitSpool data would be almost exclusively caused by the PMT. This would mean that there is a big overestimation done in the simulations of this thesis. In order to maintain the values of the correlated rate of the background caused by radioactive decays and reduce its uncorrelated contribution, the yield of the glass would have to be larger (see right side of figure \ref{fig:yieldAdependency}), and thus also the isotope activity inside the glass would have to be lower. This makes evident the requirement for more statistics for both, the scintillation parameters and the amount of radioactivity expected to be found in the glass, if the causes of the different features of the background are to be determined with greater precision.





\section{mDOM}
\label{sec:simmDOM}



\begin{figure}[h!]
  \centering
  \begin{minipage}[b]{0.49\textwidth}
    \includegraphics[scale=1.0]{Figures/simulation/CPS-mdom.pdf}
  \end{minipage}
  \hfill
  \begin{minipage}[b]{0.49\textwidth}
      \includegraphics[scale=1.0]{Figures/simulation/CPS-mdom-averagePMT.pdf}
  \end{minipage}
  \caption{Simulated time between subsequent hits in log$_{10}$($\Delta t$) for an mDOM in ice at $\SI{-35}{\celsius}$. \textbf{Left:} the distribution combining all hits of the module, \textbf{right:} the average distribution only considering the hits at single PMTs. Two sets of isotope activities labelled as VAS-1 and VAS-2 were used (see text). The statistical uncertainty is not depicted, as it is smaller than the line width.}
  \label{fig:mDOMdt}
\end{figure}

In chapter \ref{ch:GammaSpectroscopy} the amount of radioactivity in four Vitrovex samples was determined. Three of them exhibited similar activities (see table \ref{table:actCanada} and \ref{tab:actMue}), as they belong to the same production batch. In this section, the average of these three results will be used and referred as ``VAS-1''. The results for the Vitrovex half vessel (see table \ref{tab:actMue}) as ``VAS-2''. With this information and the scintillation parameters of the Vitrovex glass measured in chapter \ref{ch:scintParameters}, the $\Delta t$ distribution can be simulated for the mDOM as done in the last section for the DOM. In this case, an mDOM surrounded by ice at $\SI{-35}{\celsius}$ is simulated only taking into account glass scintillation for a pressure vessel of $\SI{13}{kg}$. Considering that the mDOM features 24 PMTs, the background can be given considering the whole module or only single PMTs. Figure \ref{fig:mDOMdt} shows the noise time distribution for both cases. Since in the case of the whole module the uncorrelated noise detected at all PMTs is summed to a single rate, the Poissonian peak exhibit a maximum at a shorter time than in the curve for a single PMT. The two isotope activity sets VAS-1 and VAS-2 yield almost the same results, with the most noticeable deviation at the very short time differences. This is due to the big contrast of $\ch{^{40}K}$ activity with VAS-1 featuring $\SI{61\pm0.9}{\frac{Bq}{kg}}$ and  $\SI{0\pm 1}{\frac{Bq}{kg}}$ for VAS-2. The $\SI{89}{\percent}$ of $\ch{^{40}K}$ decays emit an electron with mean energy $\SI{560}{keV}$ \cite{NuclideChart} and is therefore one of the main sources of Cherenkov light in the glass. However, as aforementioned, this difference would not manifest itself in measurements, as the pulse length (FWHM) of the PMTs is $\sim \SI{5}{ns}$, merging almost all Cherenkov photons in one pulse. 

\par A novel feature offered by the mDOM design is the possibility of using coincidences between different PMT in the data analysis. In this regard, it will be important to set trigger conditions (either online in the DAQ or in the reconstruction algorithms) that suppress most of the coincidences caused by the background. With this motivation, the time distribution for background coincidences between PMTs will be briefly investigated next. 

\begin{figure}[t!]
  \centering
  \begin{minipage}[b]{0.49\textwidth}
    \includegraphics[scale=1.0]{Figures/simulation/dT-mDOM-PMTS.pdf}
  \end{minipage}
  \hfill
  \begin{minipage}[b]{0.49\textwidth}
    \includegraphics[scale=1.0]{Figures/simulation/dT-Theta.pdf}
  \end{minipage}
  \centering
  \includegraphics[scale=1.0]{Figures/simulation/dT-Phi.pdf}
  \caption{Time between subsequent hits considering different PMT pairs. \textbf{Top left:} comparison of coincidence rate with the same PMT (blue) and different PMT (yellow).  \textbf{Top right:} coincidence rate separating the PMT regarding their $\vartheta$ angle difference, \textbf{bottom} regarding their $\varphi$ angle difference.}
  \label{fig:mDOM-diffPMTs}
\end{figure}


\begin{wrapfigure}{O}{0pt}
\centering
\includegraphics[scale=1.0]{Figures/simulation/mDOMaxis.pdf}
\caption{Frame of reference used in this section.}
\label{fig:frameofereference}
\vspace{-15pt}
\end{wrapfigure}
\par Since the output of the simulation entails the PMT number for every hit, the coincidences between PMTs can be analysed from the same data. For this, the data set from the simulation of VAS-1 at $\SI{-35}{\celsius}$ presented in figure \ref{fig:mDOMdt} is used. First, the most basic investigation is to separate the $\Delta t$ distribution from subsequent hits detected in the same PMT from those that were measured at different PMTs. The results are shown on the top left side of figure \ref{fig:mDOM-diffPMTs}. Here, the shortest $\Delta t$ are measured at the same PMTs, while most of the uncorrelated noise is measured at different sensors. This is to be expected since the radioactive decays are localised in a single point of the glass volume. Therefore, the detection of the scintillation and Cherenkov photons will occur most probably at the nearest PMT from the decay. On the other side, the Poissonian peak is caused by the $\Delta t$ from the last and first detected photon from different radioactive decays. As these decays take place at different locations, the uncorrelated part is mostly measured by different sensors. Without considering the effects of the PMT pulse length and TTS, it is expected a rate of $\SI{224.4\pm0.6}{s^{-1}}$, $\SI{67.9\pm0.4}{s^{-1}}$, $\SI{13.2\pm0.2}{s^{-1}}$ and $\SI{2.96\pm0.08}{s^{-1}}$ of noise at different PMTs occurring in a time interval shorter than $\SI{1}{\mu s}$, $\SI{100}{ns}$, $\SI{10}{ns}$ and $\SI{1}{ns}$, respectively. The uncertainty of these values only considers the statistical error from the simulations.


\par The coincidence distributions can be further broken down by taking into account the position of the PMTs. Figure \ref{fig:frameofereference} displays the frame of reference used, where the origin is the centre of the mDOM. The location of each PMT corresponds to a $(\vartheta,\varphi)$-pair determined by the line that pass through the centre of the PMT's photocathode and the origin. The top right side of figure \ref{fig:mDOM-diffPMTs} presents the noise time distribution regarding the angle difference $\Delta \vartheta$ between the subsequently hit PMTs. Since in the mDOM design there are four rings of sensors symmetrically positioned along the $z$-axis, there are five different possible $\Delta \vartheta$ values. The curve in the case of $\Delta \vartheta = 0^\circ$ exhibits the highest counts for correlated background, as this includes the $\Delta t$ for hits at the same PMT. As expected, the probability for coincidences from the scintillation noise decreases the further away the PMT pairs are, while the intensity of the uncorrelated peak does not depend on the PMT location, except for the case $\Delta \vartheta = 114^\circ$. Only PMTs on the rings at the extremes of the mDOM contribute to this angle difference. Since these rings have only four PMTs each, while the rings from the mid section have eight, the rate of uncorrelated coincidences is smaller. The same analysis can be done separating the $\Delta t$ distributions regarding the $\Delta \varphi$ distance between subsequently hit PMTs. The results are depicted in the lower part of figure \ref{fig:mDOM-diffPMTs}. Here, similar conclusions can be drawn as in the case of the $\Delta \vartheta$ separation, exhibiting a higher rate of coincidences in the correlated part of the distribution the smaller the angle difference $\Delta \varphi$. The only exception is the distribution for$\Delta \varphi = 135^\circ$, which features the lowest rate. This can be explained applying the same reasoning as before, since only the $8$ PMTs from the outer rings contribute to this distribution.
\par These results indicate that it is possible, if is necessary for a trigger algorithm, to reduce the background coincidence rate between PMTs by considering the PMT position. For example, if only are taken into account coincidences between PMTs separated by at least one ring ($\Delta \vartheta = 114^\circ$ and $\Delta \vartheta = 75^\circ$), the expected rate is reduced to $\SI{42.4\pm0.3}{s^{-1}}$, $\SI{12.7\pm0.2}{s^{-1}}$, $\SI{2.45\pm0.07}{s^{-1}}$ and $\SI{0.54\pm0.03}{s^{-1}}$of coincidences occurring in a time interval shorter than $\SI{1}{\mu s}$, $\SI{100}{ns}$, $\SI{10}{ns}$ and $\SI{1}{ns}$, respectively.



\subsubsection*{Background rate as a function of the temperature}
\par The temperature-dependence of the background rate caused by radioactive decays is simulated the same way as in section \ref{sec:simDOM}. The results for both isotope activity sets VAS-1 and VAS-2 are depicted in figure \ref{fig:mDOM-rates}. Like in the case of the DOM, the correlated noise features almost the same temperature-dependence of the scintillation yield of the glass - the division between the average number of hits $H(T_i,I)$ and the simulated yield is similar for all temperatures. In the case of the Benthos glass, this is to be expected, since the lifetime did not change much with temperature (see section \ref{ch:lifetime}). However, the time constant measured with the Vitrovex sample did exhibit an increase with lower temperatures. This suggests that the lifetime, at least in the $\mu s$ level, does not influence a lot the expected noise rate. The average number of hits per decay (in the case of the natural chains this means the decay of the whole isotope chain) divided by the yield is summarised in table \ref{tab:averageHit} for the three natural chains and $\ch{^{40}K}$. The most important relative contribution is from $\ch{^{238}U}$ series, which is not surprising, considering that this chain exhibits three isotopes more than the $\ch{^{235}U}$ and $\ch{^{232}Th}$ chains (see table \ref{tab:allisotopes}). 

\begin{figure}[b!]
  \centering
  \begin{minipage}[b]{0.49\textwidth}
    \includegraphics[scale=1.0]{Figures/simulation/RatesmDOM-VAS1-Simu.pdf}
  \end{minipage}
  \hfill
  \begin{minipage}[b]{0.49\textwidth}
    \includegraphics[scale=1.0]{Figures/simulation/RatesmDOM-VAS2-Simu.pdf}
  \end{minipage}
  \caption{Simulated background rate per PMT from radioactive decays inside the mDOM vessel as a function of the temperature, without (blue) and with a dead time of $\SI{2.45}{\mu s}$ (yellow). The large uncertainties are caused by the systematical error of the yield and isotope activities.}
  \label{fig:mDOM-rates}
\end{figure}

\begin{table}[b!]
\centering
\caption{Average number of detected photons per decay in the vessel glass (in the case of the natural series, after the decay of a whole chain) divided by the simulated yield for $\ch{^{40}K}$ and the natural decay series.}
\label{tab:averageHit}
\begin{tabular}{@{}ll@{}}
\toprule
                      & \begin{tabular}[c]{@{}l@{}}Hits per decay \\ per yield $\times10^{-2}$\end{tabular} \\ \midrule
$\ch{^{40}K}$         & $0.4089\pm0.0004$                                                                   \\
$\ch{^{238}U}$-chain  & $5.296\pm0.002$                                                                     \\
$\ch{^{235}U}$-chain  & $4.543\pm0.001$                                                                     \\
$\ch{^{232}Th}$-chain & $4.264\pm0.002$                                                                     \\ \bottomrule
\end{tabular}
\end{table}

\par The rate per PMT was calculated considering no dead time and with $\SI{2.45}{\mu s}$ dead time between hits, in order to be able to compare the results with the DOM. First is to be noted that with both isotope activity sets almost the same results are obtained. At $\SI{-35}{\celsius}$ and considering no dead time, the expected rate is $\SI{428\pm13}{s^{-1}}$ and $\SI{430\pm14}{s^{-1}}$ for VAS-1 and VAS-2, respectively. A $\SI{13.6\pm0.1}{\percent}$ of this rate is uncorrelated. Applying a dead time of $\SI{2.45}{\mu s}$, the rate is reduced to $\SI{230\pm5}{s^{-1}}$ (VAS-1) and $\SI{240\pm6}{s^{-1}}$ (VAS-2), whereas the uncorrelated rate remains almost unchanged (its relative contribution is increased to $\SI{23.0\pm0.1}{\percent}$ of the total). Considering that log$_{10}$($\SI{2.45}{\mu s}$)= $-5.61\,$log$_{10}$(s), the maximum of the Poissonian peak in figure \ref{fig:mDOMdt} lies far from the dead time cutoff and thus is not greatly affected by it.
\begin{figure}[t!]
  \centering

    \includegraphics[scale=1.0]{Figures/simulation/electronfactor.pdf}

  \caption{Relative difference of the background rate for different electron yield factors in respect to the scintillation yield of $\alpha$-particles. The rate calculations done in this section assumed a factor of $9.5$.}
  \label{fig:electronYieldFactor}
\end{figure}


\par All the calculations so far have assumed that the scintillation yield for electrons is $9.5$ times larger than the one for $\alpha$-particles, which is based on measurements done in other glass samples \cite[p.~255]{knoll2010radiation}. This is however only an assumption and the real factor will most probably be different. In order to estimate how much the rate calculated in this section would change with this parameter, the simulations were done again varying the electron yield factor from $0$ to $20$. The results are shown in figure \ref{fig:electronYieldFactor}. Since VAS-1 exhibits a much larger amount of $\ch{^{40}K}$ the background is more strongly dependent on the electron yield factor than in the case of VAS-2, which features only the beta decay from the natural chains. If the scintillation yield would be the same as for $\alpha$-particles, the rate presented in figure \ref{fig:mDOM-rates} would be $\SI{52.61 \pm 0.04}{\percent}$ and  $\SI{28.81\pm0.07}{\percent}$ lower for a pressure vessel with the amount of radioactivity VAS-1 and VAS-2 respectively. This means, without a set dead time, at $\SI{-35}{\celsius}$ each PMT would be expected to measure a rate of $\SI{203\pm6}{s^{-1}}$ (VAS-1) and $\SI{306\pm10}{s^{-1}}$ from light produced by radioactive decays. Conversely, if the electron yield factor would be the double ($19$), the expected rate per PMT would rise by  $\SI{53.81\pm 0.15}{\percent}$ (VAS-1) and  $\SI{30.35 \pm 0.13}{\percent}$ (VAS-2), which results in $\SI{658\pm20}{s^{-1}}$ and $\SI{560\pm18}{s^{-1}}$ respectively. Unfortunately, the $\beta$-sources available for this work were not suitable for a yield measurement. However, as long as the scintillation yield for electrons is not measured with a Vitrovex sample, it is difficult to make a more precise evaluation of the expected rate for the mDOM.  


% Please add the following required packages to your document preamble:
% \usepackage{booktabs}
%\begin{table}[h!]
%\centering
%\caption{My caption}
%\label{my-label}
%\begin{tabular}{@{}lll@{}}
%\toprule
%                      & \begin{tabular}[c]{@{}l@{}}Hits per decay \\ per yield $\times10^{-2}$\end{tabular} & \begin{tabular}[c]{@{}l@{}}Hits per decay \\ @ $\SI{-35}{\celsius}$\end{tabular} \\ \midrule
%$\ch{^{40}K}$         & $0.4089\pm0.0004$                                                                   & $0.234\pm0.006$                                                                  \\
%$\ch{^{238}U}$-chain  & $5.296\pm0.002$                                                                     & $3.03\pm0.08$                                                                    \\
%$\ch{^{235}U}$-chain  & $4.543\pm0.001$                                                                     & $2.60\pm0.07$                                                                    \\
%$\ch{^{232}Th}$-chain & $4.264\pm0.002$                                                                     & $2.44\pm0.07$                                                                    \\ \bottomrule
%\end{tabular}
%\end{table}



\section{Influence of gel scintillation}
\label{sec:gelscint}

\begin{figure}[h]
  \centering
  \begin{minipage}[b]{0.49\textwidth}
    \includegraphics[scale=1]{Figures/simulation/GEL-DOM.pdf}
  \end{minipage}
  \hfill
  \begin{minipage}[b]{0.49\textwidth}
    \includegraphics[scale=1]{Figures/simulation/GEL-mDOM.pdf}
  \end{minipage}
  \caption{Background rate stemming from scintillation of the optical gel expected to be measured in the DOM (\textbf{left}) and mDOM (\textbf{right}). The large uncertainties are caused by the systematical error of the yield and isotope activities.}
  \label{fig:GEL-influence}
\end{figure}

So far the simulations have not considered the scintillation of the optical gel. In chapter \ref{ch:scintParameters} some luminescence could be measured from both brands, QSI and Wacker. Nevertheless, the gel samples studied with gamma spectroscopy featured no measurable radioactivity. This means that this material will only emit photons when particles from decays in the pressure vessel pass through the gel.

 In order to find out if the luminescence of the gel should be a concern, radioactive decays in the glass of the module were simulated like done in section \ref{sec:simDOM} and \ref{sec:simmDOM}, but this time the gel was defined as the scintillating material instead of the glass. The Cherenkov effect was turned off in these simulations, in order to ensure that only the rate of the gel luminescence is being calculated and no dead time between hits was included.  Furthermore, the parameters of the QSI gel brand were used for both modules, since the scintillation yield of the Wacker sample could not be measured. The results are shown in figure \ref{fig:GEL-influence}. 

\par It is noteworthy that the scintillation background expected to be caused by the gel is neglectable compared to the one caused by the pressure vessel. For the DOM this has as a maximum of $\sim \SI{8}{s^{-1}}$ with the activity set BAS-2 and for the mDOM less than $\SI{1.1}{s^{-1}}$ per PMT is expected to be measured, considering the set VAS-2. In chapter \ref{ch:scintParameters} it was estimated that the Wacker gel emits photons mostly in the UV-region under $\SI{300}{nm}$ and therefore the rate presented in figure \ref{fig:GEL-influence} is an overestimation for the mDOM if the Wacker gel were to be used. Nevertheless, the crystallisation of this gel observed during the investigations of this work, excludes it from being considered for the mDOM design.