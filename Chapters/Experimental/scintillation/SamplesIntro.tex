\chapter{Measurement of scintillation parameters}
\label{ch:scintParameters}
In order to simulate the response of the optical modules to the luminescence background inside the vessel glass, it is important to rightfully parameterise0 this process. The theory behind the scintillation was already introduced in chapter \ref{ch:scintillation}. Empirically, the most important parameters are the energy distribution of the emission (scintillation spectrum), its lifetime, which determines the time distribution and the expected amount of emitted photons (scintillation yield). These three are also the default parameters needed for fully describing a scintillating material inside a Geant4 simulation. This chapter presents the method and results of the measurement of these parameters, starting with the scintillation spectrum (section \ref{sec:spectrum}), continuing with the lifetime (section \ref{ch:lifetime}) and at last the yield (section \ref{sec:yield}).
\par For practical reasons, the experiments done in the scope of this chapter used small glass samples, instead of a whole pressure vessel. The specimens of Vitrovex glass were provided in different thicknesses ranging from $\sim \SI{2}{mm}$ to $\sim \SI{10}{mm}$ by the manufacturer, while in the case of the Benthos glass a fragment of the pressure vessel of an IceCube DOM was cut into smaller square-shaped samples. Figure \ref{fig:SamplePicture} shows a picture of most of these glass specimens. In the case of the gel, the samples can be made by mixing two liquid components, which vulcanises at room temperature to a soft gel after a few hours of curing time. Hence, the shape and thickness of the gel specimens can be constructed as needed for each setup.

\begin{figure}[H]
  \centering
  \begin{minipage}[b]{0.49\textwidth}
    \includegraphics[width = 1.0\textwidth, draft = false]{Figures/other/vitrovex.png}
  \end{minipage}
  \hfill
  \begin{minipage}[b]{0.49\textwidth}
    \includegraphics[width = 0.98\textwidth, draft = false]{Figures/other/benthos.png}
  \end{minipage}
  \caption{\textbf{Left:} Some of the Vitrovex samples provided by the manufacturer, which have different thicknesses. These are on a square grid of side length $\SI{1}{cm}$. \textbf{Right:} Fragment of the IceCubes DOM vessel (Benthos glass) with the smaller samples cut from it. The side length of the grid is $\SI{1}{cm}$.  }
  \label{fig:SamplePicture}
\end{figure}