\chapter{Characterization of the PMT}

In order to rightfully simulate and calculate the scintillation properties, it is necessary to fully know the response of the detectors. In this chapter the quantum efficiency, gain and TTS of the PMT used in the experiments of this work are presented. 


\begin{figure}[H]
    \centering
    \includegraphics[width=0.8\textwidth]{Figures/PMTChar/Gain_setup08.pdf}
    \caption{Experimental setup for the determination of gain and TTS of a PMT. The dark box is placed inside a climate chamber for the temperature regulation (not shown in the figure).}
    \label{fig:gainsetup}
\end{figure}

The setup for the gain and TTS measurement can be seen in figure \ref{fig:gainsetup}. The photomultiplier is positioned in a light-tight box and illuminated with a pulsed LED\footnote{Marca}, which is triggered by a pulse generator\footnote{marca}. The latter also triggers the oscilloscope\footnote{marca}, so that the measured signal is produced by a photon from the LED and reducing the probability for measuring a dark rate event. The light of the diode is fed into the dark box by means of an optical fibre, which is terminated by a diffuser, achieving a homogeneous illumination onto the photocathode. The dark box is placed in the climate chamber for temperature regulation. As there a temperature gradient between the inside of the chamber and inside the dark box, a temperature and humidity sensor is attached at the base of the PMT, which can be controlled remotely. The PMT voltage trough out this measurements is $\SI{996\pm1}{V}$, which is the maximum for the first coarse of the high voltage supplier used. This way the same voltage can be set for all measurements .
\par The signal of the PMT gets amplified \footnote{Marca}, digitised by the oscilloscope and processed online in a computer with a Python code, saving the charge, amplitude and time of the main pulse. With this data, the gain, TTS and the amplitude to charge calibration can be done. These results are shown in the next sections.
\section{Gain}
\begin{figure}[H]
    \centering
    \includegraphics[scale=1.0]{Figures/PMTChar/SPE-amp-085.pdf}
    \caption{Charge histogram for the measurement at a temperature of $\SI{20}{\celsius}$. The data is fitted with the function REF. The resulting gaussians represent the deconvoluted phe$.$ distributions, while the pedestal has an exponential that describe noise and background processes.}
    \label{fig:SPEFIT20degree}
\end{figure}

\begin{wrapfigure}{O}{0pt}
\centering
\includegraphics[scale=1.0]{Figures/PMTChar/Gain-05.pdf}
\caption{Gain of PMT with amplifier at different temperatures. Uncertainties from the fit are smaller than the data point markers. }
\label{fig:gainVStemp}
\end{wrapfigure}
For the determination of the gain, the measured charge is arranged into a histogram and then this is fitted with the distribution from REF!. One of the fits is shown in figure \ref{fig:SPEFIT20degree}. From this, the mean charge of one photoelectron pulse $Q_1$ is obtained and the gain can be calculated with equation REF!. Figure \ref{fig:gainVStemp} shows the obtained gain values at different temperatures. In concordance with other measurements with this PMT model \cite{LEW}, the gain grows with lower temperatures, reaching at $\SI{-50}{^\circ C}$ an increase of $\sim\SI{22}{\percent}$ compared to the measurement at $\SI{-20}{^\circ C}$ . The reason for this behaviour is not known to date, although a rise on the secondary emission with lower temperatures has been observed with different electrode materials \cite{Singh_1987}, which may imply an increase of the dynode's work function with the temperature. 
\par The increase of the mean charge of one-phe. pulses with decreasing temperatures has as a consequence, that the amount of undetected one-photoelectrons due to a constant amplitude trigger (as in the case of the PMT rate measurements) will also be temperature dependent. This effect is explained and quantified in the next section.

\subsection{Photoelectron loss due to trigger}
In order to separate the PMT signals from the noise it is necessary to apply a trigger level. This means a voltage amplitude threshold has to be set, if the PMT signal is being read out in pulse mode. Although the charge and amplitude of the pulses increase linearly with the number of input photons \footnote{blah}, there is no direct translation from a pulse amplitude to its charge. This can be seen in the upper plot of figure \ref{fig:triggerAmptoPE-Method}, where the charge distribution with different voltage thresholds is depicted. With lower amplitude thresholds, only pulses with higher phe. are measured, but there is no clean cutoff. Consequently, it is necesary to set a condition that defines the charge threshold. Following the convention, this is set as the charge at which $\SI{50}{\percent}$ of the detections is lost.
\par As already explained in the beginning of this chapter, in the measurement the voltage and charge of the pulses are saved. Therefore, the corresponding charge trigger can be calculated by dividing the charge distribution of pulses with voltages lower than a certain threshold $h$ with the full charge distribution $H$ and finding the point at which the quotient is $f = 0.5$. As the division is done bin-wise, $f$ can only take discrete numbers, and hence the $0.5$ point is found by fitting the interval between $0.2$ and $0.8$ with a linear function. The charge histograms are made with the same data set, therefore there is always only one point that meets this condition. The lower plot in figure \ref{fig:triggerAmptoPE-Method} shows this procedure.

\newpage
\begin{figure}[H]
\centering
\begin{minipage}{.45\textwidth}
  \centering
  \includegraphics[scale = 1.0]{Figures/PMTChar/ampToCharge-08.pdf}
  \label{fig:test1}
\end{minipage}%
\begin{minipage}{.5\textwidth}
  \centering
  \includegraphics[scale = 1.0]{Figures/PMTChar/calibrationVsTemp.pdf}
  \label{fig:test2}
\end{minipage}
\caption{Charge trigger equivalent in phe$.$ against the set amplitude trigger in mV. The curve presents a kink at around $\sim \SI{1.2}{phe}.$, going from an affine function to a linear one.}
\end{figure}

\begin{figure}[H]
\centering
\begin{minipage}{.45\textwidth}
  \centering
  \includegraphics[scale = 1.0]{Figures/PMTChar/triggerCharge.pdf}
  \label{fig:test1}
\end{minipage}%
\begin{minipage}{.5\textwidth}
  \centering
  \includegraphics[scale = 1.0]{Figures/PMTChar/TriggerSearch.pdf}
  \label{fig:test2}
\end{minipage}
\caption{Charge trigger equivalent in phe$.$ against the set amplitude trigger in mV. The curve presents a kink at around $\sim \SI{1.2}{phe}.$, going from an affine function to a linear one.}
\end{figure}



\begin{figure}[]
    \centering
    \includegraphics[scale=1.0]{Figures/PMTChar/onePheDistributions.pdf}
    \caption{Charge trigger equivalent in phe$.$ against the set amplitude trigger in mV. The curve presents a kink at around $\sim \SI{1.2}{phe}.$, going from an affine function to a linear one.}
    \label{fig:backgroundgamma}
\end{figure}

Loss : $\sum_{i=0}^{3}a_i x^i$ $x_i= \SI{17.57\pm0.06}{\percent}  \SI{0.126\pm0.004}{\celsius^{-1}\percent}  \SI{5.8\pm1.1}{10^{-4} \celsius^{-2}\percent}$ \\
phe : $\sum_{i=0}^{3}a_i x^i$ $x_i= \SI{0.639\pm0.001}{phe.}  \SI{1.76\pm0.006}{10^{-3}\celsius^{-1}phe.}  \SI{3.3\pm1.7}{10^{-6} \celsius^{-2}phe.}$ 
 



\begin{figure}[H]
\centering
\begin{minipage}{.5\textwidth}
  \centering
  \includegraphics[scale = 1.0]{Figures/PMTChar/TriggerPEVSTemp05.pdf}
  
  \label{fig:test1}
\end{minipage}%
\begin{minipage}{.5\textwidth}
  \centering
  \includegraphics[scale = 1.0]{Figures/PMTChar/lossVSTemp05.pdf}
  \label{fig:test2}
\end{minipage}
\caption{The increase of the gain with lower temperatures effectively reduces the charge trigger level and therefore the amount of undetected photons. Left: Trigger level in phe$.$ against temperature in $\rm{^\circ C}$ for a constant amplitude trigger at $\SI{-18}{mV}$. Right: Percentage of lost one photoelectrons due to a constant trigger of $\SI{-18}{mV}$. at different temperatures}
\end{figure}

\section{Transit time spread}

With the raw data of the measurement described at the beginning of the chapter it is also possible to calculate the transit time spread. The time at which each PMT pulse surpassed the $\SI{-15}{mV}$ ($\sim 0.5\,\rm{phe.}$) voltage level was saved (leading edge time). The resulting time distribution features a characteristic main peak. The TTS is defined as the FWHM\footnote{Sometimes the TTS is defined as the standard deviation $\sigma$ of the distribution. In this work always the FWHM will be asumed, unless otherwise noted.} of this peak. Hence, one can fit a Gaussian function and the FWHM ist calculated with
\begin{equation}
FWHM = 2\cdot\sqrt{2\cdot\rm{ln}2}\cdot\sigma_{\rm{fit}},
\end{equation}
where $\sigma_{\rm{fit}}$ is the standard deviation of the fitted Gaussian. This, however, is a convolution of the time responses of all components of the measurement, which makes the measured distribution wider. The TTS can be calculated by 
\begin{equation}
TTS_{\rm{PMT}} = 2\cdot\sqrt{2\cdot\rm{ln}2}\cdot\sigma_{\rm{PMT}} = 2\cdot\sqrt{2\cdot\rm{ln}2}\cdot \sqrt{\sigma_{\rm{fit}}-\sigma_{\rm{LED}}-\sigma_{\rm{jit}}-\sigma_{\rm{sam}}},
\end{equation}
where $\sigma_{\rm{LED}} = \SI{300\pm30}{ps}$ \footnote{The length of the light pulse depends on the set intensity of the LED. These values are provided by the company. The uncertainty of $\sigma_{\rm{LED}}$ covers most of the given value range, excluding the ones at very high intensities.} comes from the non-zero duration of the light source, $\sigma_{\rm{jit}} = \SI{30}{ps}$ the jitter of the external trigger, $\sigma_{\rm{sam}} = \SI{230}{ps}$ caused by the sampling period of the oscilloscope $T = \SI{800}{ps}$\footnote{Assuming a flat distribution of length $T$, the standard deviation is calculated by $\sigma_{\rm{sam}} = \frac{T}{\sqrt{12}} = \SI{230}{ps}$.}. This calculation was done with the data set of all measured temperatures giving the results shown in figure \label{fig:TTStempdep}. The TTS does not seem to be affected by temperature, as seen in other measurements done with this PMT model \cite{LEW}. For the next calculations and simulations in this thesis the average value $\SI{3.018 \pm 0.017}{ns}$ will be used. , 
\begin{figure}[H]
\centering
\begin{minipage}{.5\textwidth}
  \centering
  \includegraphics[scale = 1.0]{Figures/PMTChar/TTSexample05.pdf}
\end{minipage}%
\begin{minipage}{.5\textwidth}
  \centering
  \includegraphics[scale = 1.0]{Figures/PMTChar/TTS-05.pdf}
\end{minipage}  
\caption{\textbf{Left:} Leading edge time distribution of the measurement at $\SI{20}{\celsius}$. In addition to the transit time spectrum, other effects like prepulsing and delayed pulses can be found (see section ref!). \textbf{Right:} Calculated TTS of the PMT used at different temperatures. The mean and its statistical and systematical error are marked with a line.}
\label{fig:TTStempdep}
\end{figure}




\section{Quantum efficiency}
\label{sec:QE}
\begin{figure}[H]
    \centering
    \includegraphics[scale=1.0]{Figures/PMTChar/QE-measurement.pdf}
    \caption{Schematic drawing of the quantum efficiency setup. The photodiode is screwed to a 3D-motor, which can move the PHD into or out of the monochromatic light ray. The photocurrent of the PMT and PHD is measured by the picoammeter, which sends the data to the computer outside the dark box.}
    \label{fig:QE-measurement}
\end{figure}

This section shows the wavelength dependence of the quantum efficiency of the used PMT. The experimental setup used for this measurements can be found in figure \ref{fig:QE-measurement}. The light of a Xenon-lamp is fed into a remote controllable monochromator, with which a wavelength section can be selected. The resolution of the monochromator in this setup is $\SI{1.2\pm0.6}{nm}$ (the slit aperture is $\SI{0.15}{mm}$, see section \ref{sec:ITSFUCKINGRAW}). This monochromatic light is then guided through a pinhole into a dark box, where the PMT, a photodiode (PHD) and the picoammeter are located. Additionally, the PHD is screwed to a 3D-motor, which can move the diode into and out of the monochromatic light ray. The diode is calibrated (its QE is known) and is used for the acquisition of the reference photocurrent. The light ray diverges, so that it illuminates most of the photocathode of the PMT, which is placed at a distance of $\sim\SI{1}{m}$ from the pinhole, but also is completely measured by the photodiode (which has a smaller sensitive area than the PMT) once it is located right in front of the pinhole. 
\par In order to circumvent the effects of the collection efficiency, the PMT is connected to a base, which applies the high voltage directly between the photocathode and the first dynode, while shorting out the multiplier system of the PMT. This way, photoelectrons can hit anywhere at the inner structure contributing to the output, although this reduces enormously the amplitude of the signal, as there is almost no electron multiplication. The applied high voltage during the measurement was $\SI{200}{V}$, similar to the typical potential between 
the cathode and first dynode in normal operation. Due to the almost nonexistent gain $\sim 1$ from the PMT and PHD, the read-out is performed by the picoammeter, which is connected to a computer. 




\begin{figure}[H]
  \centering
  \begin{minipage}[b]{0.49\textwidth}
    \includegraphics[width=\textwidth]{Figures/PMTChar/CurrentPMTQE05}
  \end{minipage}
  \hfill
  \begin{minipage}[b]{0.49\textwidth}
    \includegraphics[width=\textwidth]{Figures/PMTChar/QEPHD05}
  \end{minipage}
  \caption{\textbf{Left:} Current measured with the photodiode and the photomultiplier against the selected wavelength of the Xenon lamp spectrum. The statistical errors are smaller than the line width. \textbf{Right:} Quantum efficiency of the photodiode.}
\end{figure}


\begin{figure}[H]
    \centering
    \includegraphics[scale=1.0]{Figures/PMTChar/QE08}
    \caption{Measured quantum efficiency in dependence of the wavelength.}
    \label{fig:my_label}
\end{figure}
