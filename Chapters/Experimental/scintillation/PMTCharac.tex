\chapter{Characterisation of PMTs}
To accurately simulate and calculate the scintillation properties, it is necessary to fully know the response of the detectors. This chapter presents the quantum efficiency, gain and TTS of the PMT used in this work. 
\section{Gain and TTS}

\subsection{Setup}
\begin{SCfigure}
    \centering
    \includegraphics[scale=1.0]{Figures/PMTChar/Gain_setup08.pdf}
    \caption{Experimental setup for the determination of gain and TTS of a PMT. The dark box is placed inside a climate chamber for temperature regulation (not shown in the figure).}
    \label{fig:gainsetup}
\end{SCfigure}

The setup for the gain and TTS measurement is illustrated in figure \ref{fig:gainsetup}. The photomultiplier is positioned in a light-tight box and illuminated with a pulsed LED\footnote{PicoQuant PDL 800-B with LED PLS-8-2-719.}, which is triggered by a pulse generator\footnote{RIGOL DG1032Z}. The latter also triggers the oscilloscope\footnote{PicoScope 6404C}, allowing for effective background suppression. The light of the diode is fed into the dark box by means of an optical fibre terminated by a diffuser, achieving a homogeneous illumination of the photocathode. The dark box is placed in a climate chamber for temperature regulation. As there is a temperature gradient between the chamber and the inside of the dark box, a remotely controlled temperature and humidity sensor is attached to the base of the PMT. In order to set the same PMT voltage througout all the measurements, the maximum for the first coarse of the high voltage supplier is used, which is $\SI{996\pm1}{V}$,
\par The signal of the PMT is amplified\footnote{Fast amplifier NE 468}, digitised by the oscilloscope and processed online in a computer with a Python code, saving the charge, amplitude and leading edge time of the main pulse. This was done for $3\cdot10^5$ pulses every $\SI{5}{\celsius}$ from $\SI{20}{\celsius}$ to $\SI{-50}{\celsius}$. With this data, the gain and TTS were determined in dependence of the temperature. The results are shown in the following sections.
\subsection{Gain}
\begin{figure}[H]
    \centering
    \includegraphics[scale=1.0]{Figures/PMTChar/SPE-amp-085.pdf}
    \caption{Charge histogram for the measurement at a temperature of $\SI{20}{\celsius}$. The data is fitted with the function introduced in eq. \ref{th:fitPMTfunction}. The resulting Gaussians represent the deconvoluted phe$.$ distributions, while the pedestal has an exponential that represents noise and background processes.}
    \label{fig:SPEFIT20degree}
\end{figure}

\begin{wrapfigure}{O}{0pt}
\centering
\includegraphics[scale=1.0]{Figures/PMTChar/Gain-05.pdf}
\caption{Gain of PMT with an amplifier at different temperatures. Uncertainties from the fit are smaller than the data point markers.}
\label{fig:gainVStemp}
\end{wrapfigure}
For the determination of the gain, the measured charge is arranged into a histogram, which is then fitted with the distribution from equation \ref{th:fitPMTfunction}. However, the discrete background distribution $Q_{\textrm{sh}}$ is set to zero, as it breaks the linearity between charge and the number of photoelectrons in the intervall $0$ - $\SI{1}{phe.}$. This does not change the results by much, as a regression with the full distribution \ref{th:fitPMTfunction} also yields $Q_{\textrm{sh}}\approx 0$. An example fit is shown in figure \ref{fig:SPEFIT20degree}. From this result, the mean charge of one photoelectron pulse $Q_1$ is obtained and the gain can be calculated by the equation \ref{th:gaincalculation}. Figure \ref{fig:gainVStemp} shows the obtained gain values at different temperatures. 
\par In concordance with other measurements with this PMT model \cite[103]{LEW}, the gain grows with lower temperatures, reaching at $\SI{-50}{^\circ C}$ an increase of $\sim\SI{22}{\percent}$ compared with respect to the measurement at $\SI{20}{^\circ C}$. The reason for this behaviour is not known to date, although a rise on the secondary emission with lower temperatures has been observed with different electrode materials \cite{Singh_1987}, which may imply an increase of the dynode's work function with the temperature. 
\par The increase of the mean charge of one-phe pulses with decreasing temperatures has as a consequence that the number of undetected SPE events due to a constant amplitude trigger (as in the case of the PMT rate measurements) will also be temperature dependent. This effect is explained and quantified in the next section.

\subsubsection*{Photoelectron loss due to trigger}
\label{sec:PHEloss}

\begin{figure}[H]
\centering
\begin{minipage}{.45\textwidth}
  \centering
  \includegraphics[scale = 1.0]{Figures/PMTChar/triggerCharge.pdf}
\end{minipage}%
\begin{minipage}{.5\textwidth}
  \centering
  \includegraphics[scale = 1.0]{Figures/PMTChar/TriggerSearch.pdf}
\end{minipage}
\caption{\textbf{Left:} Effects of applying a voltage amplitude trigger to the measured pulses on the charge histogram. The charge is limited to higher phe numbers, but it does not exhibit a steep cut-off. \textbf{Right:} Example of the procedure for determining the threshold in terms of phe. The charge histogram of triggered pulses is divided by the full charge distribution bin-wise. The charge at which the quotient is $0.5$ is defined as the charge threshold equivalent to a voltage trigger level. Data from the measurement done at $\SI{-50}{\celsius}$.}
\label{fig:colorfullChargedistributions}
\end{figure}

In order to separate the PMT signals from the noise, it is necessary to apply a trigger level. This means a voltage amplitude threshold has to be set if the PMT signal is being read out in pulse mode. Although both, the charge and amplitude of the pulses, increase linearly with the number of input photons, there is no direct translation from a pulse amplitude to its charge. This can be seen in the left plot of figure \ref{fig:colorfullChargedistributions}, where the charge distribution with different voltage thresholds is depicted. With lower amplitude thresholds, only pulses with larger charge are measured, but there is no steep cutoff. Consequently, it is necessary to set a condition that defines the charge threshold. Following the convention, this is the charge at which $\SI{50}{\percent}$ of the photoelectrons is lost\footnote{Oleg Kalekin, personal communication, 18.08.2017}.
\par As already explained, in the measurement the voltage and charge of the pulses are saved simultaneously. Therefore, from the same data set, the charge distribution of pulses that surpasses a specific voltage threshold can be made. The corresponding charge trigger to this amplitude threshold can be calculated by dividing the charge distribution with trigger applied $h$ by the full charge distribution $H$ and finding the point at which the quotient is $f = 0.5$. As the division is done bin-wise, $f$ can only take discrete numbers. Therefore, the $0.5$ point is found by fitting the interval between $0.2$ and $0.8$ with a linear function. The plot on the right of figure \ref{fig:colorfullChargedistributions} shows this procedure.
\par By calculating the trigger equivalent in phe for voltage thresholds between $\SI{-80}{mV}$ and $\SI{-2.4}{mV}$ the line illustrated in figure \ref{fig:ampToPhe} is obtained. At high voltage threshold values (near zero), most of the data correspond to noise. In the charge distribution, these correspond to events in the pedestal, which have a charge close to near zero. Nevertheless, these pulses can have relatively large amplitudes, due to their fast fluctuation around the baseline. This produces a sharp line at the beginning of the curve of the left plot in figure \ref{fig:ampToPhe}. After leaving the zone of the pedestal, the line shows an expected affine behaviour, although this then turns into a linear one at around $\SI{1.2}{phe}.$. The cause of this change is at the time unclear. A possibility could be a long LED pulse duration, which would reduce the maximum voltage of the PMT pulse by increasing its length, but not the total charge. Another possibility is a restricting slew rate effect of the electronics, which also could lengthen the PMT signal\footnote{Oleg Kalekin, personal communication, 14.09.2017}. The values in the range of interest for this work $<\SI{1}{phe}.$ should not be affected by these effects.
\begin{figure}[H]
\centering
\begin{minipage}{.45\textwidth}
  \centering
  \includegraphics[scale = 1.0]{Figures/PMTChar/ampToCharge-08.pdf}
\end{minipage}%
\begin{minipage}{.5\textwidth}
  \centering
  \includegraphics[scale = 1.0]{Figures/PMTChar/calibrationVsTemp.pdf}
\end{minipage}
\caption{\textbf{Left:} Charge trigger equivalent in phe$.$ against the set amplitude trigger in mV. The curve features a kink at around $\sim \SI{1.2}{phe}.$, going from an affine function to a linear one. \textbf{Right:} Charge trigger equivalent in phe$.$ against the set amplitude trigger in mV at different temperatures. The increase in gain with lower temperatures effectively reduces the trigger level.}
\label{fig:ampToPhe}
\end{figure}
\par The right plot of figure \ref{fig:ampToPhe} shows the charge threshold at different temperatures. As a consequence of the increase in gain with lower temperatures, the charge equivalent to a constant voltage trigger sinks. The rate measurements presented in the next chapters will use a voltage threshold of $\SI{-18}{mV}$ (see section REF). The evaluation of the relation at this amplitude in figure \ref{fig:ampToPhe} yields the trigger values shown in figure \ref{fig:LossandPE}. 
\par With the deconvoluted charge distribution it is also possible to calculate the fraction of undetected photoelectrons due to the set trigger. For this, the one photoelectron Gaussian distribution is multiplied by the already calculated quotient $f$ between the triggered charge distribution and the full charge histogram. The area of this distribution $A_{\rm{triggered}}$ will be therefore smaller than the one of the original single photoelectron Gaussian $A_{\rm{full}}$, and the percentage of undetected photoelectrons can be calculated by
\begin{equation}
  \dfrac{A_{\rm{full}}-A_{\rm{triggered}}}{A_{\rm{full}}}\cdot \SI{100}{\percent}.
\end{equation}
\begin{figure}[H]
\centering
\begin{minipage}{.5\textwidth}
  \centering
  \includegraphics[scale = 1.0]{Figures/PMTChar/TriggerPEVSTemp05.pdf}
\end{minipage}%
\begin{minipage}{.5\textwidth}
  \centering
  \includegraphics[scale = 1.0]{Figures/PMTChar/lossVSTemp05.pdf}
\end{minipage}
\caption{The increase of the gain with lower temperatures effectively reduces the charge trigger level and therefore the number of undetected photons. \textbf{Left:} Trigger level in phe$.$ against temperature in $\rm{^\circ C}$ for a constant amplitude trigger at $\SI{-18}{mV}$. \textbf{Right:} Percentage of undetected one photoelectrons due to a constant trigger of $\SI{-18}{mV}$ at different temperatures.}
\label{fig:LossandPE}
\end{figure}
\begin{wrapfigure}{O}{0pt}
\centering
\vspace{5pt}
    \includegraphics[scale=1.0]{Figures/PMTChar/onePheDistributions.pdf}
    \vspace{5pt}
    \caption{Deconvoluted single photoelectron distribution with and without voltage trigger applied at $\SI{20}{\celsius}$ and $\SI{-50}{\celsius}$. With the integration of these Gaussian functions, it is possible to calculate the percentage of undetected photoelectrons due to the set trigger level.}
    \label{fig:OnePheDistributions}
\end{wrapfigure}
The SPE charge distributions at $\SI{20}{\celsius}$ and $\SI{-50}{\celsius}$ for a trigger of $\SI{-18}{mV}$ can be seen in figure \ref{fig:OnePheDistributions}. As expected, the measurement at $\SI{-50}{\celsius}$ loses fewer photoelectrons than the one at $\SI{20}{\celsius}$ owing to the gain difference. The percentage of one photoelectron loss can be seen in the left picture of figure \ref{fig:LossandPE}. 



\par The trigger charge equivalent and photoelectron loss results were fitted with a quadratic function $f(T) = \sum^{2}_{i=0} a_i T^{i}$  for its use in calculations of the next chapters. The fit coefficients of the charge trigger level in dependence of the temperature are $a^{t}_1 = \SI{0.639\pm0.001}{phe}$, $a^{t}_2 = \SI{1.76\pm0.006}{10^{-3}\celsius^{-1}phe}$ and  $ a^{t}_3 = \SI{3.3\pm1.7}{10^{-6} \celsius^{-2}phe}.$, which yield relative errors below $\SI{0.5}{\percent}$ relative to the data points of the measurement. In the case of the photoelectron loss percentage the fit coefficients are $a^{pl}_1 = \SI{17.57\pm0.06}{\percent}$, $a^{pl}_2 =\SI{0.126\pm0.004}{\celsius^{-1}\percent}$ and $a^{pl}_3 =\SI{5.8\pm1.1}{10^{-4} \celsius^{-2}\percent}$. Here the fit residuals are lower than $\SI{1.7}{\percent}$.





\subsection{Transit time spread}
\label{sec:TTS}
\begin{figure}[H]
\centering
\begin{minipage}{.5\textwidth}
  \centering
  \includegraphics[scale = 1.0]{Figures/PMTChar/TTSexample05.pdf}
\end{minipage}%
\begin{minipage}{.5\textwidth}
  \centering
  \includegraphics[scale = 1.0]{Figures/PMTChar/TTS-05.pdf}
\end{minipage}  
\caption{\textbf{Left:} Leading edge time distribution of the measurement at $\SI{20}{\celsius}$. In addition to the transit time spectrum, other effects like prepulsing and delayed pulses can be found (see section ref!). \textbf{Right:} Calculated TTS of the PMT used at different temperatures. The mean and its statistical (steming from the fit) and systematical error (including the uncertaintie from $\sigma_{\rm{LED}}$) are marked with a line.}
\label{fig:TTStempdep}
\end{figure}
With the raw data of the measurement described at the beginning of the chapter it is also possible to calculate the transit time spread. The time at which each PMT pulse surpassed the $\SI{-15}{mV}$ ($\sim 0.5\,\rm{phe.}$) voltage level was saved (leading edge time). The resulting time distribution features a characteristic main peak, as seen in the left figure \ref{fig:TTStempdep}. The TTS is defined as the FWHM \footnote{Sometimes the TTS is defined as the standard deviation $\sigma$ of the distribution. In this work always the FWHM will be assumed unless otherwise noted.}
of this peak. Hence, one can fit a Gaussian function and the FWHM is calculated with $FWHM = 2\cdot\sqrt{2\cdot\rm{ln}2}\cdot\sigma_{\rm{fit}}$, where $\sigma_{\rm{fit}}$ is the standard deviation of the fitted Gaussian. This, however, is a convolution of the time responses of all components of the measurement, which makes the measured distribution wider. The TTS can be calculated by subtracting these contributions: 
\begin{equation}
TTS_{\rm{PMT}} = 2\cdot\sqrt{2\cdot\rm{ln}2}\cdot\sigma_{\rm{PMT}} = 2\cdot\sqrt{2\cdot\rm{ln}2}\cdot \sqrt{\sigma_{\rm{fit}}-\sigma_{\rm{LED}}-\sigma_{\rm{jit}}-\sigma_{\rm{sam}}},
\end{equation}
where $\sigma_{\rm{LED}} = \SI{300\pm30}{ps}$\footnote{The length of the light pulse depends on the set intensity of the LED. These values are provided by the company. The uncertainty of $\sigma_{\rm{LED}}$ covers most of the given value range, excluding the ones at very high intensities.} comes from the non-zero duration of the light source, $\sigma_{\rm{jit}} = \SI{30}{ps}$ the jitter of the external trigger, $\sigma_{\rm{sam}} = \SI{230}{ps}$ caused by the sampling period of the oscilloscope $T = \SI{800}{ps}$\footnote{Assuming a flat distribution of length $T$, the standard deviation is calculated by $\sigma_{\rm{sam}} = \frac{T}{\sqrt{12}} = \SI{230}{ps}$.}. This was measured for a temperature range between $\SI{20}{\celsius}$ and $\SI{-50}{\celsius}$ giving the results shown in figure \ref{fig:TTStempdep}. The TTS does not seem to be affected by temperature, in concordance with other measurements done with this PMT model \cite[103]{LEW}. In the subsequent calculations and simulations in this thesis the average value $\SI{3.018 \pm 0.017}{ns}$ will be used. 











\section{Quantum efficiency}
\label{sec:QE}
\begin{figure}[H]
    \centering
    \includegraphics[scale=1.0]{Figures/PMTChar/QE-measurement.pdf}
    \caption{Schematic drawing of the quantum efficiency setup. The photodiode is attached to a 3D-motor allowing to move the PHD into or out of the monochromatic light ray. The photocurrents of the PMT and PHD are measured by the picoammeter, which sends the data to the computer outside the dark box.}
    \label{fig:QE-measurement}
\end{figure}
\noindent
This section examines the wavelength dependence of the quantum efficiency (see \cite{QE:Theory}) of the PMT. The experimental setup used for this measurements is illustrated in figure \ref{fig:QE-measurement}. The light of a Xenon-lamp\footnote{LSDH102 lamp housing with a $\SI{75}{W}$ LSB511 Xe bulb by LOT QuantumDesign.} is fed into a remotely controllable monochromator\footnote{MSH-300 with grating MSG-T-1200-300 by LOT QuantumDesign.}, which selects a wavelength section of the whole spectrum. The resolution of the monochromator in this setup is $\SI{1.2\pm0.6}{nm}$ (the slit aperture is $\SI{0.15}{mm}$, see section \ref{sec:ITSFUCKINGRAW}). This monochromatic light is then guided through a pinhole into a dark box, where the PMT, a photodiode (PHD) and the picoammeter (a high precision amperemeter) are located. Additionally, the PHD is attached to a 3D-motor, which can move the diode into and out of the monochromatic light ray. The diode is calibrated (its QE is known) and is used for the acquisition of the reference photocurrent. The light beam diverges, illuminating most of the photocathode of the PMT, which is placed at a distance of $\sim\SI{1}{m}$ from the pinhole, but is also completely measured by the photodiode (which has a smaller sensitive area than the PMT) once it is located right in front of the pinhole. 
\par In order to eliminate the effects of the collection efficiency, the PMT is connected to a base, which applies only the high voltage between the photocathode and the first dynode, while shorting out the multiplier system of the PMT. This way, photoelectrons can hit anywhere at the inner structure contributing to the output, although this reduces enormously the amplitude of the signal, as there is almost no electron multiplication. The applied high voltage during the measurement was $\SI{200}{V}$, similar to the typical potential between the cathode and first dynode in normal operation. Due to the almost nonexistent gain $\sim 1$ from the PMT and PHD, the read-out is performed by the picoammeter, which is connected to a computer.
\begin{figure}[t]
  \centering
  \begin{minipage}[b]{0.49\textwidth}
    \includegraphics[scale=1.0]{Figures/PMTChar/CurrentPMTQE05}
  \end{minipage}
  \hfill
  \begin{minipage}[b]{0.49\textwidth}
    \includegraphics[scale=1.0]{Figures/PMTChar/QEPHD05}
  \end{minipage}
  \caption{\textbf{Left:} Current measured with the photodiode and the photomultiplier against the selected wavelength of the Xenon lamp spectrum. The statistical errors are smaller than the line width. \textbf{Right:} Quantum efficiency of the photodiode.}
  \label{fig:QECurrent}
\end{figure}

\par In this measurement the average of 20 current measurements at each wavelength between $250$ and $\SI{750}{nm}$ were taken in $\SI{5}{nm}$ steps, once with the photodiode and once with the PMT. Also, before and
\begin{wrapfigure}{O}{0pt}
\centering
\includegraphics[scale=1.0]{Figures/PMTChar/QE08}
\caption{Measured quantum efficiency in dependence of the wavelength. }
\label{fig:QEresultsNormalPMT}
\vspace{-10pt}
\end{wrapfigure}
after the measurement the dark current of the devices is measured. For this the wavelength $\SI{1200}{nm}$ is selected, as the detectors are not sensitive in this range, and, in the case of the PMT, the photodiode is placed in front of the pinhole. For the dark current of the diode, the PHD is moved out of the light beam, in order to avoid incident light.
\par The measured photocurrents are shown in figure \ref{fig:QECurrent}. As there is practically no amplification, the currents of the two devices can be directly compared and the quantum efficiency of the PMT $QE_{\rm{PMT}}(\lambda)$ is derived from the known efficiency of the photodiode $QE_{\rm{diode}}(\lambda)$ (see figure \ref{fig:QECurrent}) by
\begin{equation}
    QE_{\rm{PMT}}(\lambda)=\dfrac{I_{\rm{PMT}}(\lambda)-DC_{\rm{PMT}}}{I_{\rm{diode}}(\lambda)-DC_{\rm{diode}}}\cdot QE_{\rm{diode}}(\lambda), 
\end{equation}
where  $QE(\lambda)$ is the quantum efficiency for the wavelength $\lambda$, $I(\lambda)$ the measured current at $\lambda$ and $DC$ the average dark current. 
\par Figure \ref{fig:QEresultsNormalPMT} shows the calculated QE of the PMT. It is to notice, that the uncertainty of the measurement increases in the UV-region. This is a measurement artifact due to the low intensity of the measured photocurrent, which drops to levels similar to the dark current. Also, the QE rises in the interval $250$-$\SI{265}{nm}$, most probably attributable to effects from scattered light inside the monochromator. A small fraction of the output from the monochromator is scattered light from every wavelength of the light source. The spectrum of the Xenon lamp in the UV region is very dim and hence the intensity of the contamination can be similar or larger than the one from the selected wavelength, overestimating the QE of the device being measured. Therefore the data points between $250$-$\SI{280}{nm}$ are excluded from every further calculation. In figure \ref{fig:QEresultsNormalPMT} the average QE of X PMTs and its standard deviation is depicted as a reference. The measured PMT exhibits thus an expected response. (!!!!!!!?)

\section{Average SPE pulse}

\begin{wrapfigure}{o}{0pt}
\centering
\includegraphics[scale=1.0]{Figures/PMTChar/SPE-amplifier.pdf}
\caption{Average SPE from the PMT used in the measurements.}
\label{fig:SPEamp}
\vspace{-10pt}
\end{wrapfigure}
As introduced in section \ref{sec:photonhandlingG4}, the Geant4 simulation used in this thesis can form waveforms given a curve that represents a single photoelectron pulse. Therefore the average SPE from the PMT in this setup is needed. For this, the PMT dark rate pulses are measured at room temperature, considering only the signals with a charge in the interval $Q_1\pm \sigma_1$, where $Q_1$ is the mean charge of the SPE and $\sigma_1$ its standard deviation obtained from the calibration. In order to match the positioning of the pulses, each one is shifted so that its minimum lies in $t=0$. The result of averaging $\sim 3700$ of such pulses is illustrated in figure \ref{fig:SPEamp}. Noteworthy are the reflections, which are probably produce by the coupling between the cables and the amplifier, as this SPE can be compared with the pulse shown in figure \ref{fig:waveforms}, which was made from this same setup and PMT, but without the amplifier.
\par As the gain change with temperature, also the SPE minima will vary. This is however not considered in the simulation and only the SPE at room temperature is used. Nevertheless, this should be still a good approximation for the purposes of this thesis, as the waveforms are simulated just for reasons of time resolution. 

\section{Summary}