\section{Scintillation Yield}
\label{sec:yield}
In the last two sections, the scintillation spectrum and lifetime of the investigated materials were determined. This section presents the calculation of the last parameter, the scintillation yield. The approach described in section \ref{sec:YieldwSource}  was exciting the samples with an alpha source and measuring the PMT rate produced from the scintillation. The yield can be then determined if the geometry of the setup and the $\alpha$-emission is well known. For this purpose, the activity of the source was measured with an alpha spectrometer in section \ref{sec:alphaactivity}. 
Key for the experiments of this section is to have a reliable method for measuring the PMT rate, as many factors of the setup may influence the measured value. The next part is dedicated to this topic.


\subsection{Measuring PMT rates}
In order to measure the rates of the PMT an oscilloscope \textit{PicoScope 6404C} is used. This can be operated in \textit{rapid block mode}, where an adjustable number of waveforms (or captions) is measured and then sent to the computer. The number of sampling points in each caption can be set to one so that the dead time between two waveforms is only determined by the duration of the trigger rearming (a few $\rm{\mu s}$ according to Picoscope REF!). This whole process is controlled by a Python code, which also calculates the time $t_\textrm{N}$ needed for the oscilloscope for measuring $N$ captures. The requested number of waveforms is automatically set by the code, as having a constant number of captions could result in very few data points, if the PMT rate is low, or too many when it is high. This would be especially problematic for measurements in the climate chamber, as the temperature can change relatively quickly. Hence, in order to have a steady measurement rate, the number of captions is set so that $t_\textrm{N}$ lies between $2$ to $\SI{3}{s}$.
\par It stands to reason that the rates are given with $N\cdot t^{-1}_{\textrm{N}}$, however $t_\textrm{N}$ has to be corrected, as it is determined by the python code and not the oscilloscope, meaning that in this time the processing and data transfer time from the oscilloscope to the PC alters the real PMT rate. This correction is calculated in the next subsection. Nevertheless, still after this, the calculated rate is constrained by other parameters and deviates from the real number of detections of the PMT. One obvious variable is the undetected photoelectrons due to the trigger level, as already explained in section \ref{sec:PHEloss}. Another parameter is the dead time between captures, which will severely reduce the measured rate, as it is in the same order of magnitude as the decay time constant of the scintillation. 


\subsubsection*{Rates correction}
\label{sec:ratecorrection}

For estimating the error of the rate measurement described above and its correction, a pulse generator\footnote{RIGOL model DG1032Z} was connected to the oscilloscope producing rectangular pulses with $\SI{16}{ns}$ width and random frequencies between $0$ and $\SI{10}{kHz}$. The accuracy of this generator is high, with an uncertainty of $\SI{\pm 1}{ppm}$ of the set frequency value \cite{RIGOL}. These pulses were measured with the same code used for the PMT rates measurement, meaning that the number of captions is set in such a way, that the measurement time per data point is $\SI{2}{s}$ to $\SI{3}{s}$. Figure \ref{fig:correctionDarkrates} shows the results of this measurement. It can be seen, that as expected the measured rates are smaller than the set frequency, especially at higher rates. 
\begin{wrapfigure}{l}{0pt}
    \centering
    \includegraphics[scale=1.0]{Figures/yield/correctedDarkRates085.pdf}
    \caption{Measured rate by the oscilloscope for different input frequencies generated by the pulse generator. The data is fitted with a fifth-degree polynomial, yielding relative residuals of less than $\SI{0.3}{\percent}$.}
    \label{fig:correctionDarkrates}
    \vspace{-50pt}
\end{wrapfigure}
This can be partially explained by an increase in processing time of the oscilloscope with the number of captions or longer delay for the data delivering to the computer. 
\par The measurement shows an almost linear relationship, nevertheless it was fitted with polynomials of different numbers of degrees. The best result was achieved with a fifth-degree polynomial 
\begin{equation}
    f(x) = \sum^{5}_{i=0}a_i \cdot x^{i},
\end{equation}
with the coefficients $a_i$
\begin{footnotesize}
\begin{align*}
&a_0 = \SI{-0.08\pm0.5}{s^{-1}}, \hspace*{10pt} & a_1 &= 1.0026\pm0.0010, \\
&a_2=\SI{7.3\pm0.7}{\cdot 10^{-6}\,s}, \hspace*{10pt} & a_3 &= \SI{-0.8\pm1.7}{\cdot 10^{-10}\,s^{2}},\\
&a_4 = \SI{-1.8\pm2.1}{\cdot 10^{-14}\,s^{3}}, \hspace*{10pt} & a_5 &= \SI{-8.1\pm8.8}{\cdot 10^{-19}\,s^{4}}.
\end{align*}
\end{footnotesize}
\noindent
Correcting the rates with this function produces a maximal relative error of about $\SI{0.3}{\%}$, as seen at the residuals of the fit in figure \ref{fig:correctionDarkrates}.



%\begin{wraptable}{O}{5.cm}
%\centering
%\caption{Coefficients of the quintic fit function.}
%\label{tab:quinticCoeff}
%\begin{tabular}{@{}ll@{}}
%\toprule
%$a_0$ & $\SI{-0.08\pm0.5}{s^{-1}}$               \\
%$a_1$ & $1.0026\pm0.0010$                        \\
%$a_2$ & $\SI{7.3\pm0.7}{\cdot 10^{-6}\,s}$       \\
%$a_3$ & $\SI{-0.8\pm1.7}{\cdot 10^{-10}\,s^{2}}$ \\
%$a_4$ & $\SI{-1.8\pm2.1}{\cdot 10^{-14}\,s^{3}}$ \\
%$a_5$ & $\SI{-8.1\pm8.8}{\cdot 10^{-19}\,s^{4}}$ \\ \bottomrule
%\end{tabular}
%\end{wraptable} 



\subsubsection*{Dead time between captures}
\label{sec:deathTime}
To determine the dead time between captures a sine wave with frequency $f_{in}$ is generated with the pulse generator and sampled in rapid block mode, as in the configuration of the rate measurement, but without a set trigger. Each capture takes one value of the sine amplitude, which gets digitised at 8 bit by the oscilloscope. The amplitude of the signal is set in such a way, that the whole range of the oscilloscope is used, for maximising the achieved resolution.
\par As the oscilloscope is measuring without a trigger, the signal is being sampled at the maximal sampling rate, where the sampling period is determined by the dead time between the captures. Thus, by fitting the data with a sine function, the number of captures $N$ needed for the measurement of one period of the sine wave $P_{\textrm{in}}$ can be calculated, and thus also the sampling period
\begin{equation*}
    P_{\textrm{s}} = \frac{P_{\textrm{in}}}{N} = \frac{f_{\textrm{fit}}}{f_{\textrm{in}}},
\end{equation*}
where the $f_{\textrm{fit}}$ is the obtained frequency from the fit. This is done with a different number of captures, in order to investigate any dependency between the measurement configuration and the dead time. These results can be seen at the left side of figure \ref{fig:deathTimeResults}, showing no apparent dependency between the mean dead times measured with various number of captures, with deviations of less than $\SI{0.05}{\percent}$. The mean sampling period is $\SI{1063.701\pm0.023}{ns}$.
\begin{figure}[H]
\centering
\begin{minipage}{.5\textwidth}
  \centering
  \includegraphics[scale = 1.0]{Figures/yield/dTvCaptures05.pdf}
\end{minipage}%
\begin{minipage}{.5\textwidth}
  \centering
  \includegraphics[scale = 1.0]{Figures/yield/distribution.pdf}
\end{minipage}
  \caption{\textbf{Left:} Comparison between the mean dead time retrieved with different number of captures. \textbf{Right:} The distribution of dead time between captures. It can be approximated as a Gaussian function.}
  \label{fig:deathTimeResults}
\end{figure}
\par Although the mean dead time can be retrieved with a very high accuaracy, this also will vary between captures.  As the fit curve always marks the position of the mean, the standard deviation of the sampling period can be estimated by calculating the distance between the data points to the fit curve along the x-axis. This yields the distribution shown in the right part of figure \ref{fig:deathTimeResults}, which can be fitted with a Gaussian with the standard deviation of $\sigma_{\textrm{1}}=\SI{82.68\pm0.28}{ns}$. However, this distribution get widened by the y-resolution steming from the digitasition of the oscilloscope. As commented, the data is ordered in 8 bit, therefore the amplitude can get assigned integer values between $-128$ to $128$. This means, there is an uncertainty of $0.5$ for every y-value, which also affects the x-residuals. Therefore the real variance of the dead time will be $\sigma_{\textrm{R}}^2 = \sigma_{\textrm{1}}^2 - \sigma_{\textrm{y-res.}}^2$, where $\sigma_{\textrm{y-res.}}$ is the x-residual widening stemming from the $y$ uncertainty. The latter is estimated by artificially doubelling the uncertainty to the $y$-values, increasing them randomly from a flat distribution of length $1$. The resulting x-residual distribution has a standard deviation of $\sigma_{\textrm{2}}= \sqrt{\sigma_{\textrm{R}}^2 + 2\cdot\sigma_{\textrm{y-res.}}^2 } = \SI{83.75\pm0.28}{ns}$ and therefore the standard deviation of the death time is  $\sigma_{\textrm{R}} =\sqrt{2\cdot\sigma_{\textrm{1}}^2 -\sigma_{\textrm{2}}^2 } = \SI{81.6\pm0.6}{ns}$.


\subsection{Yield determination by external excitation}
\label{sec:YieldwSource}
In this section, the yield is calculated exciting the samples with the alpha source. The experimental setup used for this is the same as the one from the lifetime measurement shown in figure \ref{fig:LifetimeMeasurement}. The activity and an estimation of the isotope distribution on the source surface are calculated in the next subsection, as these parameters are needed for the yield estimation. Besides, the air scintillation yield is calculated, since there is a small gap between the samples and the source, which contaminates the results.
\subsubsection*{Activity of the alpha source}
\label{sec:alphaactivity}
\begin{figure}[H]
\centering
\begin{minipage}{.5\textwidth}
  \centering
  \includegraphics[scale = 1.0]{Figures/yield/RadiusEstimation-Source.pdf}
\end{minipage}%
\begin{minipage}{.5\textwidth}
  \centering
  \includegraphics[scale = 1.0]{Figures/yield/Average_activity.pdf}
\end{minipage}
\caption{\textbf{Left:} Comparison between simulation (lines) and the measurement results (points) of the relative efficiency variation at different source-detector distances for an assumed circular emission of radii $R_e \in [0,15]\,\rm{mm}$. The values of the measurement lie consistently between the results for $R_e = \SI{9}{mm}$ and $R_e = \SI{10}{mm}$. The uncertainties are smaller than the point markers. The simulation values are only valid at the distances where the measurements were done and are represented as lines for reasons of clarity. \textbf{Right:} Activity resulting from the measurement done at different distances. The mean and standard deviation (STD) of this points are also given with lines. }
\label{fig:alphasourceRadiiandActivity}
\end{figure}
For the determination of the absolute activity of the source, an alpha spectrometer is used. The principle of an alpha spectrometer is very similar to the gamma spectroscopy explained in section \ref{sec:gammaspec}. In this case, the source is placed right in front of a semiconductor detector inside a vacuum chamber (pressure $<10^{-7}\,\rm{Bar}$), to avoid energy loss with air molecules. The signal of the semiconductor is shaped and amplified by a preamplifier and an amplifier, and then the signal's amplitude is digitised by a multichannel analyzer. This way, an energy spectrum of the emission is obtained. In the scope of this work, only the activity of the source is needed, thus only the integration of the spectrum is necessary (total number of counts). The activity of the source $A$ can be calculated with 
\begin{equation}
\label{eq:SourceActivity}
    A = \frac{1}{4\pi}\Omega \cdot \frac{N^{\textrm{m}}_D}{t} = \varepsilon \cdot \frac{N^{\textrm{m}}_D}{t}, 
\end{equation}
where $\Omega$ is the solid angle covered by the detector, $\varepsilon$ the absolute detection efficiency, $N^{\textrm{m}}_D$ the number of detections and $t$ the duration of the measurement. The solid angle is only defined by a point source emission, otherwise $\varepsilon$ should be used.
\par The calculation of the absolute detection efficiency is not trivial, as the distribution of the $\ch{^{241}Am}$ isotopes on the surface of the source is not known. To estimate this distribution, one can measure the relative change of $\varepsilon$ when the source is placed at different distances to the detector, and compare this to simulation results of a certain distribution. In the setup, $\SI{2\pm0.2}{mm}$ is the nearest the source and the semiconductor can be placed. To increase the distance $d$, up to three holders each with width $\SI{11.0\pm0.1}{mm}$ are used. 
\par The source emission is first assumed to be circular with radius $R_e$ and isotope. A radius of $R_e = \SI{0}{mm}$ would correspond to the case of a point source in the middle of the source. For this, a Geant4 code is used, where the setup geometry is reproduced and the fraction of detected particles $N^{\textrm{s}}_D$ from a total of $10^6$ is obtained. This is simulated for emissions with radii between $\SI{0}{mm}$ and $\SI{15}{mm}$ in $\SI{1}{mm}$ steps and source-to-detector distance as in the experimental setup. To compare the simulation results with each other and the measurement, the number of detections at each distance $N^{\textrm{s}}_D$ is divided by the mean $\overline{N^{\textrm{s}}_D}$ between the four distances. For the measurement results, the counts per second ($N^{\textrm{m}}_D \cdot t^{-1}$) is used instead of the total counts, as the duration of the measurements varies. The results are presented in the left plot of figure \ref{fig:alphasourceRadiiandActivity}. It can be seen, that the measurement results lie between the values of the simulation with radii $\SI{9}{mm}$ and $\SI{10}{mm}$ at the four distances. Interpolating the $N^{\textrm{s}}_D / \overline{N^{\textrm{s}}_D}$ values with the measurement points, one obtains an effective emission radius of $R_{\textrm{eff}}=\SI{9.62\pm0.11}{mm}$.  Although most probably the density and shape of the isotope distribution on the source's surface are very complex, these results connote that it can be effectively described as a circular distribution of constant density.
\par Simulating again with the calculated emission radius, the absolute detection efficiency $\varepsilon = N^{\textrm{s}}_D \cdot 10^{-6}$ is obtained for each distance and the activity is calculated with equation \ref{eq:SourceActivity}. The results can be seen at the right of figure \ref{fig:alphasourceRadiiandActivity}. The mean activity between the four distances is $\SI{2834\pm4}{Bq}$. The uncertainty of this value is only the statistical part originated from the error of the integration of the measured spectra and the efficiency calculated from the simulation ($\sqrt{N_D}$ in both cases). For determining the uncertainty stemming from $\Delta d$ and $\Delta R_{\textrm{eff}}$, the efficiency is simulated again but with distances $d \pm \Delta d$ and radii $R_{\textrm{eff}} \pm  \Delta R_{\textrm{eff}}$. The average activity between the four distances is calculated with these efficiencies, and the maximal deviation arising from $\Delta R_{\textrm{eff}}$ and $\Delta d$ are $\SI{16}{Bq}$ and $\SI{79}{Bq}$ respectively. Thus the absolute activity of the source is $A_s = \SI{2834\pm81}{Bq}$.


\subsubsection*{Air scintillation yield}
\label{sec:Airscint}
\begin{wrapfigure}{o}{0.5\textwidth}
\centering
\includegraphics[scale=1.0]{Figures/yield/AirRate.pdf}
\caption{Measured rate from the scintillation of the air. The raw data is binned in $\SI{1}{\celsius}$ steps for the sake of clarity. Also illustrated is the corrected rate after subtraction of the PMT dark rate. The second $x$-axis shows the percentage of SPE loss due to the set threshold of $\SI{-18}{mV}$ and the trigger equivalent in phe (see section \ref{sec:PHEloss}).}
\label{fig:airScintRate}
\vspace{-30pt}
\end{wrapfigure}
In order to measure the scintillation yield of the air, the source is placed $\SI{5.90\pm0.05}{cm}$ in front of the PMT, without any sample between them. At this distance, alphas cannot reach the PMT and are completely absorbed in the air. The climate chamber is first cooled to $\SI{-50}{\celsius}$, then switched off and from then on the PMT rates saved as the chamber slowly warms up to room temperature. The results are shown in figure \ref{fig:airScintRate}. As the rate is measured every $2$ to $\SI{3}{s}$, there is a lot of data points for each temperature. For better handling and clarity this data is binned in $\SI{1}{\celsius}$ steps.
\par The measured rate contains also the intrinsic background from the PMT and has to be corrected. Hence, the dark rate of the PMT is measured (only the PMT is inside the dark box) three times, obtaining the results shown in the left side of figure \ref{fig:PMTrates}. For the sake of clarity, only the binned data is illustrated. The three measurement exhibit similar results in the temperature range from $\SI{-50}{\celsius}$ to $\sim \SI{-10}{\celsius}$, but then the increase from the thermionic effect start at different temperatures and its slope also differs. The reason for this behaviour is not clear at this time.For the further calculations, the average rate in the temperature interval from $\SI{-50}{\celsius}$ to $\SI{-15}{\celsius}$ in $\SI{1}{\celsius}$ steps will be used. This is illustrated in the right plot of figure \ref{fig:PMTrates}. 
\begin{figure}[t]
  \centering
  \begin{minipage}[b]{0.49\textwidth}
    \includegraphics[scale=1.0]{Figures/PMTChar/PMTrates.pdf}
  \end{minipage}
  \hfill
  \begin{minipage}[b]{0.49\textwidth}
    \includegraphics[scale=1.0]{Figures/PMTChar/PMTratesZoom.pdf}
  \end{minipage}
  \caption{Three measurements of the PMT dark rate in dependency of the temperature. The second $x$-axis shows the percentage of SPE loss due to the set threshold of $\SI{-18}{mV}$ and the trigger equivalent in phe (see section \ref{sec:PHEloss}). \textbf{Left:} For the whole temperature range of the measurement, \textbf{right:} only in the region of interest $\SI{-50}{\celsius}$ to $\SI{-15}{\celsius}$ with the mean rate between the three runs. }
  \label{fig:PMTrates}
\end{figure}

\begin{wrapfigure}{O}{0.5\textwidth}
\centering
\includegraphics[scale=1.0]{Figures/yield/gammaIncrease.pdf}
\caption{PMT rate increase from the mean dark rate due to gamma-rays emitted from the radioactive source.}
\label{fig:gammaInfluence}
\end{wrapfigure}


\par Furthermore it is important to determine the influence of the gamma rays emitted by the $\ch{^{241}Am}$-source on the measured rate, as these photons could release photoelectrons at the photocathode or dynodes. For this, the radioactive source was placed inside a black plastic cup in front of the PMT, in order to shield the PMT from the air luminescence. The rate increase compared to the dark rate is illustrated in figure \ref{fig:gammaInfluence}. There is no apparent temperature dependency, which is expected because of the constant decay rate of the source. The average rate increase is of $\SI{4.2\pm0.2}{s^{-1}}$. However, this is just slightly larger than the variation between the measurements of the PMT dark rate (see figure \ref{fig:PMTrates} right) and thus, this increase could be mostly caused from the deviation of the PMT dark rate. For the further measurements with the source, the rate will be corrected by $\SI{4\pm2}{s^{-1}}$. Nevertheless, the influence of the gammas is almost neglectable, as the rate from the samples and air scintillation is in the order of $\mathcal{O}(10^2)$-$\mathcal{O}(10^3)$.

\par The corrected rate of the air scintillation is illustrated in figure \ref{fig:airScintRate}. With this, it is possible to calculate the yield using the Geant4 simulation. Here only the PMT and the source are simulated with the same distance between each other as in the measurement. The air scintillation spectrum used is the one measured in \cite{AIRFLY} shown in figure \ref{fig:airscintillation}. The photon detection is done as described in chapter \ref{ch:g4}, using the dead time between captures calculated previously. In one simulation event, an alpha is emitted from the source with an isotropic direction. The scintillation photons are separated into two categories depending on their amplitude, if two or more photons were detected within a time that cannot be resolved by the PMT, it counts as a ``definitive'' hit $H_D$. If this is not the case, e.g. only one photon is detected, there is a possibility that it does not goes through the threshold of the PMT and is saved as a ``possible'' hit $H_P$. After $2000000$ events, the mean $\overline{H_P}(y)$ and $\overline{H_D}(y)$ per decay for the air scintillation yield $y$  is obtained. Simulating with several $y$, it is possible to calculate the expected rate $R(y, T)$ in dependence on the yield $y$ and temperature $T$:
\begin{equation}
\label{eq:airExpRate}
    R(y, T) = \Big( \overline{H_D}(y)+\overline{H_P}(y)\cdot \big(1-P(T)\big) \Big) \cdot A_s,
\end{equation}

\begin{wrapfigure}{I}{0.5\textwidth}
\centering
\includegraphics[scale=1.0]{Figures/yield/airLFcomparison.pdf}
\caption{Calculated air scintillation yield for alpha particles with two different lifetimes $\tau_{\textrm{air}}=\SI{0.5}{ns}$ and $\tau_{\textrm{air}}=\SI{2}{ns}$. Error bars include only the statistical uncertainty (see text).}
\label{fig:twoAiryield}
\end{wrapfigure}
where $A_s = \SI{2834\pm81}{Bq}$ is the activity of the source, determined in section \ref{sec:alphaactivity}, and $P(T)$ the probability of a SPE for not being detected due to the trigger level, calculated in section \ref{sec:PHEloss}. The air scintillation yield is then calculated by interpolation the simulation data with the measured rate $R_m$. This is done with two different scintillation lifetimes. According to REF the decay time constant of the air scintillation at atmospheric pressure is $\SI{0.5}{ns}$, however, this value can vary between $\SI{1.9}{ns}$ and $\SI{0.4}{ns}$, depending on the air composition and pressure. These are values well under the time resolution of the PMT, and it should make just a little difference, as most of the time all photons are detected as a single pulse. Nevertheless, the simulation was done with $\tau_{\textrm{air}}=\SI{0.5}{ns}$ and $\tau_{\textrm{air}}=\SI{2}{ns}$, the latter as a conservative value for the estimation of the uncertainty. The results are shown in figure \ref{fig:twoAiryield}. As expected, the case with $\tau_{\textrm{air}}=\SI{2}{ns}$ results in a slightly larger yield ($\sim\SI{0.8}{\percent}$) than the case with $\tau_{\textrm{air}}=\SI{0.5}{ns}$, as more hits are SPE and therefore categorised into the $\overline{H_P}(y)$ part instead of $\overline{H_D}(y)$. The difference between these two distributions is used as the error stemming from the lifetime. The final result for the yield and its uncertainty is presented in the right part of figure \ref{fig:airYield}. Most of the uncertainty is the systematical error, which includes the uncertainty from the source activity $\Delta A_s$, the error of the distance between the PMT and source and the uncertainty from the lifetime just calculated. In the statistical error, the uncertainty from $\Delta P(T)$, $\Delta R_m$ and the simulation parameters $\Delta \overline{H_{D,P}}$ are considered.
\begin{figure}[H]
  \centering
  \begin{minipage}[b]{0.49\textwidth}
    \includegraphics[scale=1.0]{Figures/yield/airFinal.pdf}
  \end{minipage}
  \hfill
  \begin{minipage}[b]{0.49\textwidth}
    \includegraphics[scale=1.0]{Figures/yield/airFinalHumidity.pdf}
  \end{minipage}
  \caption{Calculated yield illustrated in dependency of the temperature (left) and of the relative humidity (right). The blue region represents the total uncertainty (systematical and statistical error, see text).}
  \label{fig:airYield}
\end{figure}
\par It is noteworthy, that the air scintillation yield seems to increase with the temperature between $\SI{-50}{\celsius}$ and $\SI{-30}{\celsius}$ and then remains fairly constant as the temperature increases. The average yield in the temperature range from $\SI{-30}{\celsius}$ to $\SI{-15}{\celsius}$ is $\SI{18.7\pm1.2}{MeV^{-1}}$. This agrees with the results from other studies that were done at room temperature. Sand et al  reported a yield for alpha particles of $\SI{19\pm3}{MeV^{-1}}$ in 2014 \cite{Sand_2014} and Thompson et al. $\SI{18.9\pm2.5}{MeV^{-1}}$ in the year 2016 \cite{Thompson_2016}. Quantitative studies on air scintillation yield with alpha particles are rather rare, and no values in dependency of the temperature could be found in the literature. It is known, however, that there is a dependency on the air composition, and therefore also on the number of water molecules \cite{Sand_2014}. Hence, the decrease at lower temperatures seen in figure \ref{fig:airYield} is most probably an effect of the change of the humidity with the temperature rather than from the temperature itself. On the right side of figure \ref{fig:airYield} the same data points are illustrated but in dependency of the measured relative humidity from the sensor on the base of the PMT.  Here, the results show a fairly linear behaviour, which supports the premise. Nevertheless, for the sake of simplicity at further calculations, the air yield in dependency of the temperature will be taken, as the relative humidity does not change much between different measurements. Furthermore, the uncertainty of the yield is larger than the differences seen between different temperatures or humidities and therefore any shift between measurements is considered in the error.





\subsubsection*{Measurement with the samples}

\begin{figure}[t]
  \centering
  \begin{minipage}[b]{0.49\textwidth}
    \includegraphics[scale=1.0]{Figures/yield/RatesVitrovex.pdf}
    \includegraphics[scale=1.0]{Figures/yield/RatesQSI.pdf}
  \end{minipage}
  \hfill
  \begin{minipage}[b]{0.49\textwidth}
    \includegraphics[scale=1.0]{Figures/yield/RatesBenthos.pdf}
    \includegraphics[scale=1.0]{Figures/yield/RatesWacker.pdf}
  \end{minipage}
  \caption{Measured rate from the scintillation of the for samples. The raw data is binned in $\SI{1}{\celsius}$ steps. Also illustrated is the corrected rate after subtraction of the PMT dark rate. For sake of clarity in this case the the percentage of SPE loss and the trigger equivalent in phe were not depicted in a second $x$-axis (cf. figure \ref{fig:PMTrates} and \ref{fig:airScintRate}).}
  \label{fig:RatesSamples}
\end{figure}

\begin{wrapfigure}{I}{0.5\textwidth}
\centering
\includegraphics[scale=1.0]{Figures/yield/AirRatesSetups.pdf}
\caption{Simulated rate caused by air scintillation in the experimental setup for the different samples.}
\label{fig:airInSetup}
\end{wrapfigure}

The rate measured with the samples being excited with the $\alpha$-source is illustrated in figure \ref{fig:RatesSamples}. All specimens exhibit a linear increase of the rates with lower temperatures, excepting the Wacker gel, which features a shift at $\sim \SI{-35}{\celsius}$. According to the manufacturer, this gel crystalises between $\SI{-45}{\celsius}$ and $\SI{-50}{\celsius}$ into a white harder state. Therefore this shift is caused probably either due to a higher optical absorption or larger scattering probability inside the sample. As the rate measurement is done while the chamber is warming up, the deviation happening $\SI{10}{\celsius}$ away from expected may indicate a hysteresis effect for the decrystallisation process. Nevertheless, it is noteworthy that the rate decreases linearly with the temperature before and after the deviation, which may imply that the emission of the sample exhibits a similar behaviour as the one of the other materials.
\par For each setup, the influence of air scintillation was simulated, in order to correct the measurement. This will vary, depending on the geometry of the setup (gap of air between the sample and radioactive source, distance to PMT, etc.) and the transmission of the specimen being examined. The simulated geometry of the glass and gel samples are shown in figure \ref{fig:G4samples}. While the glass samples where directly positioned in front of the PMT with the holding clamp, the gel samples where cured inside the holding cylinder of the source. In order to protect the source against the gel, a ring with width $\SI{0.6\pm0.1}{mm}$ was positioned in-between. The results of the air rate simulation are shown in figure \ref{fig:airInSetup}. As the gap between the gel specimens and the source is small, a fairly low rate is measured from the air luminescence. In the case of the glass samples, this gap was wider ($\SI{1\pm0.1}{mm}$), which results in a higher rate. The setup with the largest contamination is the one of the Vitrovex specimen, as this is the biggest sample (side length of $\sim \SI{3}{cm}$ compared to $\sim \SI{1}{cm}$ of the Benthos specimens), the PMT covers a larger solid angle.

\begin{figure}[H]
  \centering
  \begin{minipage}[b]{0.49\textwidth}
    \includegraphics[scale=1.0]{Figures/yield/G4-Vitrovex.pdf}
  \end{minipage}
  \hfill
  \begin{minipage}[b]{0.49\textwidth}
    \includegraphics[scale=1.0]{Figures/yield/G4-gel.pdf}
  \end{minipage}

  \caption{Simulated geometry of the experimental setup shown in figure \ref{fig:LifetimeMeasurement}. \textbf{Left:} for the glass samples; in the picture, it has the dimensions of the Vitrovex specimen. \textbf{Right:} for the gel samples. }
  \label{fig:G4samples}
\end{figure}

\par The calculation of the yield is quite similar as it was done with the air luminescence (section \ref{sec:Airscint}). Here, however, one also has to take into account the variation of the lifetime with the temperature. Thus, the simulation is done with the five lifetimes values measured in section \ref{ch:lifetime}, obtaining an expected number of detections per decay that definitely surpassed the trigger level $\overline{H_D}(y)$, and the ones that have to be corrected for the detection loss $\overline{H_P}(y)$. As these lifetimes correspond to the temperatures $T_n=[-50,-45,-35,-25,-15]^{\circ}C$, the simulations results can be ordered to a more general $\overline{H_D}(y,T_n)$ and $\overline{H_P}(y,T_n)$. With this it is possible to make an interpolation $\ell$ for every temperature between $T_i$ and $T_{i+1}$, following
\begin{equation}
\ell_{D,P}(y,T) = \overline{H_{D, P}}(y, T_i) + \big(T-T_i\big)\cdot \frac{\overline{H_{D, P}}(y, T_{i+1})-\overline{H_{D, P}}(y, T_i)}{T_{i+1}-T_i}.
\end{equation}
Hence, the expected rate measured at the temperature $T$ with an emission of yield $y$ is calculated as in equation \ref{eq:airExpRate}:
\begin{equation}
     R(y, T) = \Big( \ell_D(y,T)+\ell_P(y)\cdot \big(1-P(T)\big) \Big) \cdot A_s,
\end{equation}
where $A_s = \SI{2834\pm81}{Bq}$ is the activity of the source, and $P(T)$ the probability of a SPE for not being detected due to the trigger level. Comparing this expected rate with the measured one, it is possible to calculate the emission yield of the samples. These results are presented in figure \ref{fig:YieldSamples}. 


The yield for the Wacker gel was not simulated since there is no information about the spectrum, which is needed for the simulation. As mentioned in section \ref{sec:correctionOfTHeSpectra}, this material probably emits mostly in the UV region, where the PMT is not sensitive. Hence, the measured rate corresponds probably to a small fraction of the whole spectrum.
\begin{figure}[t]
  \centering
  \begin{minipage}[b]{0.49\textwidth}
    \includegraphics[scale=1.0]{Figures/yield/VitrovexYield.pdf}
  \end{minipage}
  \hfill
  \begin{minipage}[b]{0.49\textwidth}
    \includegraphics[scale=1.0]{Figures/yield/Benthosyield.pdf}
  \end{minipage}
  \centering
   \includegraphics[scale=1.0]{Figures/yield/QSIYield.pdf}
  \caption{Calculated yield for alpha particles in photons emitted per absorbed MeV for both glass samples and the QSI gel in dependence of the temperature. }
  \label{fig:YieldSamples}
\end{figure}
\begin{wrapfigure}{i}{0.35\textwidth}
  \centering
    \includegraphics[width=0.45\textwidth, angle = 90]{Figures/yield/radiationdamage.png}
  \caption{Darkening of the transmission of the sample at the location where the source was placed indicating radiation damage. The contrast of the image was strongly increased for better discernability. Grid of side length $\SI{1}{cm}$.}
  \label{fig:radiationDamage}
  \vspace{-10pt}
\end{wrapfigure}
\noindent 

\par In the case of the other three samples, between $\SI{-15}{\celsius}$ and $\SI{-50}{\celsius}$ the yield increases $\sim \SI{57}{\percent}$ (from $\SI{42.1\pm1.2}{MeV^{-1}}$ to $\SI{65.9\pm2.2}{MeV^{-1}}$) for the Vitrovex, $\sim \SI{48}{\percent}$ (from $\SI{26.7\pm0.8}{MeV^{-1}}$ to $\SI{39.4\pm1.3}{MeV^{-1}}$) for the Benthos and $\sim \SI{60}{\percent}$ (from $\SI{36.6\pm1.4}{MeV^{-1}}$ to $\SI{58.6\pm2.5}{MeV^{-1}}$) for the QSI specimen. This was expected for the Vitrovex and QSI samples, as this was also the behaviour shown by their average lifetimes (see figure \ref{fig:meanAverage}). In the model considered in section \ref{sec:QYLif}, the recombination efficiency is directly proportional to the lifetime, as $\eta = \tau \cdot \tau_{\textrm{r}}^{-1}$, where $\tau_{\textrm{r}}$ is the time constant of the radiative transition, which is normally treated not to be temperature dependent. However, for the Benthos glass the average lifetime remained fairly constant at different temperatures, which contradicts the measurement of its yield in figure \ref{fig:YieldSamples}. This may imply an error during the measurement of the time constant. With only five measurement points for the lifetime temperature dependence it is difficult to spot any incongruity in the data. Nevertheless, the points between $\SI{-35}{\celsius}$ and $\SI{-15}{\celsius}$ show a decreasing behaviour, which could suggest an error in the measurement of the values at $\SI{-50}{\celsius}$ and $\SI{-45}{\celsius}$. Another possibility could be a more complex underlying process, which is not considered in the model. 
\par Furthermore, the calculated yield contains some errors, as the scintillation spectrum was only measured at room temperature and this changes with the temperature since the transitions occur at different energy levels due to the influence of phonons (see section \ref{sec:abandem}). It is difficult to estimate this deviation, as also the absorption of the material will change. Nevertheless, these results can be taken as an effective yield, that is valid for PMTs with similar QE, and thus this should not be a big problem for the studies on dark rates, as the same PMT model was used for this measurements as the one planned for the mDOM.
\par A noteworthy effect observed after the measurements were disk-shaped darkening spots on the glass samples (see figure \ref{fig:radiationDamage}), which could not be cleaned with solvents like water, ethanol, isopropanol or acetone. It is a known fact that radiation damage can cause a reduction of the optical transmission of inorganic semiconductors, although most literature refers to accelerator physics, where the energy of the incident particles is orders of magnitude higher \cite{Lecoq_2016}. The rate measurement was done right after the one of the lifetime, and thus the samples were continuously irradiated with the $\alpha$-source at least four days, and even longer as some measurement had to be repeated. Thus it is not known the time span needed for this radiation damage to take effect. Moreover, it is difficult to determine if and how much this effect may have influenced the yield measurement, as although the transmission change is not very dramatic (the contrast of figure \ref{fig:radiationDamage} was highly enhanced), the darkening implies a change in the electronic states configuration in the region where the absorption and emission occurs \cite{Lecoq_2016}.



