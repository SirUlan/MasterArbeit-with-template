\section{Possible measurement improvements}
\label{sec:SetupImprovements}

 In the scope of this chapter, the main parameters of the scintillation of glass and gel samples were measured for their application in the background simulation of the optical modules. 
 \par Starting with the \textbf{scintillation spectrum}, it was possible to measure the wavelength of the luminescence from the exposure of the samples to two different radioactive sources, an $\ch{^{241}Am}$- $\alpha$- and a stronger $\ch{^{90}Sr}$- $\beta$-source. This gave some information about the nature of this process, as the measured spectra lie near the band gap energy of the material, meaning that the radiative transitions are produced most probably by defects near the conduction and valence band and excluding the possibility of a unique de-excitation centre from the contamination of rare-earth elements as initially thought. In the case of the Wacker gel sample, no spectrum was measured with the $\alpha$-source, since, following the same reasoning, the bandgap energy of this material lies in the UV-region, where the used PMT is not sensible. A signal was measured with the $\beta$-source, although this was shown by comparing it with simulations to be most probably only Cherenkov radiation.

 \par Yet, there is room for improvement of the experimental approach, as different factors affected the results. On the one hand, the $\alpha$-source excited the air in its surrounding, contaminating the results with the discrete emission of molecular nitrogen. This radioactive source also yields a low-emission intensity, and therefore, a rather low signal to noise ratio. On the other hand, the $\beta$-source provided better results regarding noise, but it also generated Cherenkov photons. Additionally, as the emission was near the transmission cutoff of the sample, information was lost and the real spectrum could not be measured.  

\begin{wrapfigure}{O}{0.45\textwidth}
\centering
\includegraphics[scale=1.0]{Figures/spectrum/spectrumMeasurementImproved.pdf}
\caption{Experimental approach for an improvement of the results. The sample is positioned at an angle towards the optical system, such that only photons that travelled a short path through the sample are measured.}
\label{fig:spectrumImproved}
\end{wrapfigure}

\par A better approach would be to take advantage of the fact that scintillation light is emitted in all directions equally. Hence, one could measure the light that travelled the shortest path through the sample\footnote{Only a few $\rm{\mu m}$ for excitation with alpha particles.}, by irradiating the sample with $\alpha$-particles with a particular angle towards the entrance of the optical system (see figure \ref{fig:spectrumImproved}). This requires, however, two conditions: on the one hand, the measurement has to be done in a low-pressure environment, so that the $\alpha$-particles reach the sample and to reduce the air scintillation. Also, the $\alpha$-source should have an activity in the order of hundreds of $\rm{GBq}$ for a better signal to noise ratio. This activity may need to be even higher, as probably the light must be coupled into an optical fibre if the vacuum chamber is not big enough for the monochromator, which would severely reduce the detection efficiency. This setup, however, should deliver results with no external contamination and almost no information loss due to absorption inside the glass. Also, a PMT sensible in the region between $200$-$\SI{300}{nm}$ should be preferred, as the transmission cut-off of the PMT's window is the next limiting factor after the absorption of the samples.
\par Furthermore, this measurement should be done at different temperatures, as the scintillation spectrum in most semiconductors changes with it. 
\par The measurement of the \textbf{lifetime} resulted in a long time distribution in the order of $\mu$s for the emission of the glass samples and a much faster one in the order of $\textrm{ns}$ in the case of the gel specimens. For both gel and the Vitrovex sample, the average lifetime decreased with higher temperatures, which was not the case for the specimen of the Benthos glass. This decrease was expected for materials where thermal quenching is an important nonradiative process since it is more probable to release the energy through phononic vibrational states, if these are in higher (thermal) energy levels. However, the measurement was constrained by two factors. On the one side, the number of data points measured in the temperature range was limited by the long duration of the measurement to only five temperatures between $\SI{-50}{\celsius}$ and $\SI{-15}{\celsius}$. In the context of the background simulation of the optical modules, this should not represent any major problem, as the data can be interpolated. Nevertheless, more measurements at different temperatures should be done, if interpretations of the underlying luminescence processes are to be made, especially with the glass samples, since these did show a more erratic behaviour with the temperature. On the other side, the lifetimes had to be corrected, as the time point of the decay was not known. This could be avoided by utilising a more complex setup. For example, the radioactive source could be surrounded by scintillators optically coupled with optical detectors, which can measure the gamma-rays emitted at the decays, e.g. the $\SI{59.5}{keV}$ photon emitted with a $\SI{35.9}{\percent}$ probability from the decay of $\ch{^{241}Am}$ \cite{NuclideChart}. This would then trigger the measurement of the waveforms of the PMT measuring the luminescence of the samples, instead of using the first detected photon as the trigger, as it was done in this chapter. However, this would radically increase the duration of the measurement, which would restrain even more the statistics.
\par The determination of the \textbf{scintillation yield} is in hindsight of the other two parameters far simpler, as only the increase of the PMT rate is measured. All samples exhibited a rise of the emission with lower temperatures, except for the Wacker gel, as the crystallisation of the material changed the optical properties of the sample. In the case of the Benthos glass, this emission raise contradicts the results of the lifetime, which however could be explained by the low number of temperatures measured.
\par Moreover, for the calculation of the yield both, the spectra and lifetime of the samples are needed. Hence, any constraints and uncertainties applied to these parameters will also affect the yield results. In this regard, the main improvement for the determination of the yield would be more precise measurements of the lifetime and scintillation spectra, although better results could also be obtained by measuring the rates in a low-pressure environment in order to prevent air scintillation.  Furthermore, in this chapter only the yield for $\alpha$-particles was investigated. As introduced in chapter \ref{ch:scintillation} the intensity of the emission changes with different kind of charged particles, viz.\ the yield for electrons is normally larger than for heavier particles like protons or $\alpha$-particles. The $\beta$-sources that were available for this work were either too strong or encapsulated in transparent plastic, which will probably scintillate by itself. Thus, these sources have to be characterised in more detail if the yield for electrons is to be determined. In addition, the effects of the radiation damage observed on the glass samples regarding the calculated yield should be investigated. A possibility for this would be to measure the rate of scintillation light for a long period with a constant temperature, in order to observe any decrease of the emission due to the darkening of the material caused by the radiation damage.
\par As a final remark, it is important to observe that there is a lack of statistic as only one sample of every brand was measured. Since the best assumption is that the scintillation is produced by lattice defects and/or impurity atoms, the luminescence properties will probably vary between different samples, especially between specimens from different production batches. Hence, more samples should be studied for their scintillation properties, in order to determine an expected deviation of the luminescence parameters for the optical modules.