\section{Summary and outlook}

\begin{wrapfigure}{O}{0pt}
\centering
\includegraphics[scale=1.0]{Figures/spectrum/spectrumMeasurementImproved.pdf}
\caption{Experimental approach for an improvement of the results. The sample is positioned at an angle towards the optical system, such that only photons that travelled a short path through the sample are measured.}
\label{fig:spectrumImproved}
\end{wrapfigure}

In the scope of this chapter, we saw that it is possible to measure the wavelength of scintillation light coming from the samples. This gave us some information about the nature of this process, as the measured spectra lie near the bandgap energy of the material, excluding the possibility of a unique de-excitation centre. 
\par Yet, there is room for improvement of the experimental approach, as different factors affected the results. On one side, the $\alpha$-source excited the air in its surrounding, contaminating the results with the discrete emission of molecular nitrogen. This radioactive source also yields a low emission intensity, and therefore a rather weak signal to noise ratio. The $\beta$-source on the other side, provided better results regarding noise, but it also generated Cherenkov photons. Additionally, as the emission was near the transmission cutoff of the sample, information was lost and the real spectrum could not be measured.
\par A better approach would be to take advantage of the fact that scintillation light is emitted in all directions equally. Hence, one could measure the light that travelled the shortest path through the sample\footnote{Only a few $\rm{\mu m}$ for excitation with alpha particles.}, by irradiating the sample with $\alpha$-particles with a particular angle towards the entrance of the optical system (see figure \ref{fig:spectrumImproved}). This requires, however, two conditions: on the one hand, the measurement has to be done in a low-pressure environment, so that the $\alpha$-particles reach the sample and to reduce the air scintillation. Also, the $\alpha$-source should have an activity in the order of hundreds of $\rm{GBq}$ for a better signal to noise ratio. This activity may need to be even higher, as probably the light must be coupled into an optical fibre if the vacuum chamber is not big enough for the monochromator, which would severally reduce the detection efficiency. This setup, however, should deliver results with no external contamination and almost no information loss due to absorption inside the glass. Also, a PMT sensible in the region between $200$-$\SI{300}{nm}$ should be prefered, as the transmission cut-off of the PMT's window is the next limiting factor after the absorption of the samples.
\par Furthermore, this measurement should be done at different temperatures, as the scintillation spectrum in most semiconductors changes with it. 