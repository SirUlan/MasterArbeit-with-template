\chapter{Scintillation spectrum}
 The three variables for the parametrization of the scintillation were introduced at section REF!. In the scope of dark rates, only the amount of emitted photons and its time distribution are needed. As the measurements are done with the same PMT type that will be used in the mDOM, one could waive determining the emission spectrum and work with a relative yield. However, as it was shown in section REF!, the scintillation yield and lifetime are interdependent parameters. Thus, for a correct description of the time distribution (lifetime) the absolute yield is needed, making the determination of the emission spectrum mandatory.
 \par This section provides a description of the measurement of the scintillation spectrum for the different materials. 
First, the acquisition of the spectra is presented and explained. These results are afterwards corrected, compensating various influencing factors of the experimental method, as the wavelength dependence of the QE of the PMT and the transmittance of the samples. 


\section{Raw scintillation spectra}
\begin{wrapfigure}{O}{0.4\textwidth}
\centering
\includegraphics[width=0.4\textwidth]{Figures/spectrum/monochromator_45}
\caption{default}
\label{fig:expSpec}
\end{wrapfigure}
The measurements of the scintillation spectra were conducted with the setup seen in figure \ref{fig:expSpec}. A radioactive source excites the sample to scintillation. The emitted light goes then through the entrance slit of a remote controllable monochromator\footnote{LOT Quantum Design model MSH 300 with a grating model MSG-T-1200-250.}, which breaks the scintillation down to a narrow wavelength segment. Right in front of the exit slit of the monochromator sits a small PMT, which measures the output light in current mode. The PMT is a Hamamatsu R7600U-200, which has a flat photocathode that can be placed in such a way that almost all the light emitted by the monochromator is measured. Its photocathode material is ultra bialkali, which exhibits a much higher QE in comparison to the normal bialkali of the Hamamatsu R12199. The PMT signal is measured with a picoammeter\footnote{Keithley model 487.}.
\par For this measurement, it is indispensable that the samples emit with vast intensity, since only a small fraction of the emitted light is measured at a time. This makes necessary the use of a strong radioactive source. The first measurements were done with an $\ch{^{241}Am}$ alpha source, which had an approximate activity of $\sim \SI{370}{kBq}$. The average energy of the emitted alphas is $\SI{5.479}{MeV}$, producing an energy deposition in the material of $\sim \SI{1}{TeV\cdot s^{-1}}$. Nevertheless, the number of detected photons is low, yielding a rather poor signal to noise ratio. The intensity of the output light is highly dependent on the entrance and exit slit aperture. If the aperture of one of the slits is reduced by the half, the intensity is also halved. Additionally, both slits should have the same aperture, meaning that by reducing the width to the half, the intensity is fourfold smaller. Their aperture also determine the resolution of the monochromator. Figure REF shows this dependency, where the wider the slits are, the broader is the output light. Consequently, one has to make compromise between the signal to noise ratio and the resoulution of the measured spectrum. In this case it was at an aperture of $\SI{2.5}{mm}$, where the resolution of the monochromator is $\SI{7.3 \pm 0.7}{nm}$.
\begin{wrapfigure}{O}{0.5\textwidth}
\centering
\includegraphics[width=0.5\textwidth]{Figures/spectrum/resolution05}
\caption{Resolution of the monochromator output in dependence on the slits apertures. This was calculated measuring the FWHM of emission lines of a Hg lamp.}
\label{fig:expSpec}
\end{wrapfigure}
\par The intensity of the PMT current is measured in $\SI{2}{nm}$ for wavelengths between $250$ and $\SI{650}{nm}$. One data point delivered by the picoammeter is the integration of the PMT current for a period of $\SI{16}{ms}$. For one measurement cycle, the average of 100 of this data points is made for each wavelength. Depending on the sample, at least 15 cycles are done and then averaged, until the random noise is flattened. The resulting emission spectra of the samples are depicted in figure \ref{fig:am241Vitrovex}.








































\newpage

 
\begin{wrapfigure}{L}{0.5\textwidth}
\centering
\includegraphics[width=0.5\textwidth]{Figures/spectrum/Sr90beta049}
\caption{PMT output current as a function of the selected wavelength, once for the Sr-90 source alone and once where a Vitrovex sample was excited.}
\label{fig:Sr90Vitrovex}
\end{wrapfigure}
Because of the constraints that are presented by the measurement with the Am-241 source, another measurement with a Sr-90 of $\sim \SI{0.3}{GBq}$ was done. Sr-90 is a beta source with a mean energy of $\SI{195.8}{keV}$ and decays into Y-90. The latter also undergoes beta decay with a mean energy of $\SI{933.7}{keV}$ REF! . As the half-life of the yttrium isotope ($\SI{64}{h}$ REF!) is much lower than the one of its mother nucleus ($\SI{28.8}{y}$ REF!), they are in secular equilibrium. With both isotopes, the yielded energy deposition in the sample lies around $\sim \SI{200}{TeV\cdot s^{-1}}$. A beta source has the advantage, that it has a range in the air between $1$ to $\SI{8}{m}$ REF! depending on its energy. This reduces the probability of detecting light coming from air luminescence. Thus, this also means that the sample has to be thicker, for the energy to be totally absorbed in the material. The thicker the sample is, the more photons will be absorbed in the material, losing sensitivity of the spectrum, especially in the UV region (see section REF!). \par
The results obtained with the strontium source exciting a Vitrovex sample are shown in figure \ref{fig:Sr90Vitrovex}. Here, the measured spectrum has the maximum at the same location as in the last case with Am-241, but its tail reaches the $\SI{600}{nm}$. The approached signal to noise ratio is much better, and long measurements are not necessary. In this case, the source also scintillated, but its intensity is much lower than the sample's scintillation and it does not seem to be coming from air luminescence.
\begin{figure}[H]
    \centering
    \includegraphics[width=1.\textwidth]{Figures/spectrum/Pictures.pdf}
    \caption{Long exposure ($\SI{15}{min}$) photos from the glass and gel samples scintillating (left, center), as well from the source itself (right).}
    \label{fig:pcitures}
\end{figure}

\begin{wrapfigure}{R}{0.55\textwidth}
\centering
\includegraphics[width=0.55\textwidth]{Figures/spectrum/background-wrap055}
\caption{Flower two.}
\label{fig:backgroundComparison}
\end{wrapfigure}
\par To find the origin of the background light from the source, long exposure pictures were taken with a DSLR camera from each glass and gel sample scintillating, as well from the source itself \footnote{Nikon D5500 with AF-S 18-55 VR lense}. The photos are depicted in figure \ref{fig:pcitures}. Apart from the scintillation itself, small yellow spots can be seen. Comparing the photo from the source (see figrue \ref{fig:pcitures} top right) with the long exposure picture (see figrue \ref{fig:pcitures} bottom right), one can see that the light is coming from inside the source. The Sr-90 is encapsulated in glass and a very thin layer of steel \footnote{For a detailed view of the source, see Appendix REF!.}.  The light may be produced by scintillation of the glass capsule and then escapes the steel layer, which seems to be worn-out in some areas. Blocking this light is not trivial. The photons get through thin paper easily and thick materials would reduce the electron flux coming from the source. Also, the choice of the right material is very important, as plastics or white paper scintillate when excited increasing the overall background. The best compromise was found with a sheet from $\sim\SI{1}{mm}$ Neoprene enclosed in black paper. The photons coming from the source could not be completely blocked in this configuration though, as it can be seen in \ref{fig:backgroundComparison}. A discussion about the influence of this background on the measured spectra is made at the end of this section.
\begin{figure}[H]
  \centering
  \begin{minipage}[b]{0.49\textwidth}
    \includegraphics[width=\textwidth]{Figures/spectrum/vitrovex_entry}
  \end{minipage}
  \hfill
  \begin{minipage}[b]{0.49\textwidth}
    \includegraphics[width=\textwidth]{Figures/spectrum/vitrovex_exit}
  \end{minipage}
  \caption{Caca}
  \label{fig:slitsAperture}
\end{figure}

\par Before making the final measurements for each sample, the right slit aperture must be found for both, the entrance and exit slit of the monochromator. These slits play a major role in the performance of the monochromator, determining the instrumental bandwidth and the intensity of the output. As one is measuring a very low light intensity source, the slits should be as open as possible. However, the optical resolution worsens with wider slits. Therefore, it has to be verified if the measured spectra have discrete constituents that need a high resolution. Hence, the spectrum of each sample was measured with different entrance $a_{\rm{en}}$ and exit slit apertures $a_{\rm{ex}}$. The results of the Vitrovex sample can be seen in figure \ref{fig:slitsAperture}. Here, the spectra are normalised with their total area in order to be able to compare their shape. There is no noticeable difference between the different spectra, and it can be concluded that the scintillation spectrum does not have narrow components. Therefore the final and longer measurement is done with the maximal slit aperture of $a_{\rm{en}}= a_{\rm{ex}}= \SI{10}{mm}$, which yields more precise results. This is also the case for the other three materials (the results at different slits apertures can be found in Appendix REF!). 
\par blahblah final measurements. background discussion

\begin{figure}[H]
    \centering
    \includegraphics[width=\textwidth]{Figures/spectrum/allSpec}
    \caption{Caption}
    \label{fig:my_label}
\end{figure}

\subsection{Correction factors}

As already stated, the shape of the measured spectra is distorted by different factors from the experimental setup. These have to be measured and 
\subsubsection{Quantum efficiency from the measuring PMT}








\subsubsection{Transmittance from the samples}


\subsubsection{Efficiency of the monochromator grating}



