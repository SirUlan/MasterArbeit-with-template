\chapter{Material radioactivity}
\label{ch:GammaSpectroscopy}
Essential for this work is to know how much radioactivity is expected to be found in the optical modules. One method for measuring isotope activities is gamma spectroscopy. This can be done with a scintillator or a semiconductor detector. For measurements where the isotopes in the sample are not known, usually the latter is used, as it yields a much higher energy resolution and therefore the gamma energies can be better distinguished and classified. Hence, this chapter will focus only on spectroscopy using semiconductor detectors.
\par In 2016, measurements were done at the University of Alberta, Canada, at a low-level counting setup for, among others, a Vitrovex and a Benthos glass sample. The results of these measurements needed geometry correction factors that were calculated in this work. Also, for the sake of better statistics, more measurements were carried out in the scope of this thesis.
\par At the beginning of this chapter, a small introduction to Gamma spectroscopy is given (\ref{sec:gammaspec}). The second part (\ref{subsec:gammAlberta}) is devoted to the analysis of the measurements of Alberta and at the final part the measurements done in the scope of this work are explained and evaluated (\ref{subsec:gammMünster}). An extensive description of gamma spectroscopy and semiconductor detectors can be found in \cite{Leo1994} and \cite{knoll2010radiation}, and a more specific description for low-level counting in \cite{Zvara1994}.


\section{Gamma spectroscopy with germanium detector}
\label{sec:gammaspec}
\begin{wrapfigure}{O}{0pt}
\centering
\includegraphics[scale=1.0]{Figures/gammaspectroscopy/gamma045wrap.pdf}
\caption{Pulse-height spectrum of a monoenergetic gamma-ray source. Data taken from \cite{Zvara1994}.}
\label{fig:gammafuncresp}
\vspace{-10pt}
\end{wrapfigure}
With gamma spectroscopy it is possible to calculate the unknown concentration of radionuclides by evaluating the energies and intensities of gammas emitted by decays in a sample. These gammas interact with the detector medium releasing ionising electrons by the photoelectric effect, Compton scattering or pair production. In the case of a semiconductor detector, which is based on a p-n junction, the ionising electrons produce a number of electron-hole pairs proportional to the absorbed energy. These charge carriers are then collected and amplified, for then being read out as a voltage pulse. The amplitude of this signal is digitised by a multichannel analyzer. 
\par A typical pulse-height spectrum is shown in figure \ref{fig:gammafuncresp}. At the right side of this spectrum, one can see the full energy peak (FEP) produced when the gamma energy is totally absorbed. At lower energies of the FEP, one can find the Compton continuum, measured when the gamma scatters in the detector but then escapes without being fully absorbed. The calibration from channel to energy can be done by measuring a known radioactive source and relating the centre position of its FEPs to the gammas energy. In this manner, one can later estimate the gamma-ray energy of a sample by determining the position of the FEPs in the spectrum, which ultimately leads to the identification of the decaying isotope. The concentrations of the radionuclide can be then calculated by evaluating the area under its FEPs.
\par For a total number number of counts (net peak area) $S_{\rm{0}}$ from a FEP corresponding to the energy $E_{\rm{0}}$, the activity of the emitter isotope $A_{\rm{0}}$ can be calculated using 
\begin{equation}
\label{formula:activity}
    A_{\rm{0}} = \dfrac{S_{\rm{0}}}{t_{\rm{m}} \cdot \varepsilon (E_{\rm{0}}) \cdot I_{\rm{0}}},
\end{equation}
where $t_{m}$ is the duration of the measurement,  $\varepsilon (E_{\rm{0}})$ the absolute detection efficiency for gammas with energy $E_{\rm{0}}$ and $I_{\rm{0}}$ the gamma-ray intensity (probability of emission of the particular gamma-ray per decay). The efficiency $\varepsilon (E)$ depends on the geometry of the setup, and on the energy of the gamma quantum, as the interaction probability for the photoelectric effect is energy dependent. It can be calculated using Monte Carlo simulations, taking into account the geometries and materials of the detector and the source, and on the distance between them.
\begin{wrapfigure}{O}{0.3\textwidth}
\centering
\vspace{-10pt}
\includegraphics[width=0.3\textwidth]{Figures/gammaspectroscopy/gamma_045.pdf}
\caption{Sketch of the parameters for the calculation of net peak area.}
\label{fig:gammaintegration}
\vspace{-10pt}
\end{wrapfigure}
\par Since in a normal gamma spectrum there are many FEPs with different energies, the peaks are superimposed on a background formed by the Compton continuum steming from more energetic gamma-rays. Therefore, for the estimation of the net area peak, it has to be corrected for the compton background contribution. There are different approaches for this evaluation \cite{prospect}, but as long the spectrum does not exhibit poor counting statistics, there is no much difference between their results. In this thesis, it will be assumed a linear continuum under the FEP, which is described in \cite{prospect}. Here, the average background count per channel is calculated immediately before and after the peak. This value is then subtracted from the peak. If the FEP is found across N bins, 
and the background continuum before and after the FEP is spread over $n_1$ and $n_2$ channels, which have a total of $B_1$ and $B_2$ events, then the total counts in the FEP coming from the background $B$ is

\begin{equation}
B = \frac{N}{2}(\frac{B_1}{n_1}+\frac{B_2}{n_2})
\end{equation}
and hence the net area will be $S = T-B$, where T is the total number of counts under the FEP. Figure \ref{fig:gammaintegration} shows a sketch of this calculation. 
\par Not all the measured FEPs will originate from radionuclides in the sample. There is also a contribution from decays ocurring in the detector and materials surrounding the detector, and radiation from cosmic rays. Therefore it is important to reduce these external sources by shielding the detector with materials with high atomic number, like lead. Anyway, it is necessary to make a background measurement (that is, without any sample in the detector complex), since it is not possible to completely eliminate the external background. The gamma spectra have to be then corrected with this background measurement.
\par The levels of background radiation of the detector determines the sensitivity of the system. If a background measurement of time $t_0$ has $B$ counts at a specific FEP, then the uncertainty of this measurement will be $\sigma_{\rm{B}}=\sqrt{B}$, as it follows Poisson statistics. For a signal coming from a sample to be statistically valid, it has to produce at least $\sqrt{B}$ counts, which means there is a lower boundary $L(t_0,E_0)$ for the measurable activities, which depends on the detection system and the measured time. Following \autoref{formula:activity} this can be determined to be
\begin{equation}
    L(t_0,E_0) = \dfrac{\sigma_{\rm{B}}(E_0)}{t_{\rm{0}} \cdot \varepsilon (E_{\rm{0}}) \cdot I_{\rm{0}}} = \dfrac{\sqrt{B(E_0)}}{t_{\rm{0}} \cdot \varepsilon (E_{\rm{0}}) \cdot I_{\rm{0}}}
\end{equation}
for a FEP corresponding to an energy $E_0$. \par The background measurements of the setup of Alberta and Münster, are presented in figure \ref{fig:backgroundgamma}. It is noticeable that for most energies, the background of the system in Alberta is almost one order of magnitude lower. This makes it more suitable for finding gamma-rays of inferior intensities and isotopes with small activities.  Therefore, the isotope identification will be done from the measurements made in Canada, which are presented in the next section. Nevertheless, it will be shown that the setup of Münster is sensitive enough for measuring most of the isotopes in the sample, allowing to calculate activities with similar relative uncertainties.
\begin{figure}[H]
    \centering
    \includegraphics[scale=1.0]{Figures/gammaspectroscopy/Backgrounds-085.pdf}
    \caption{Background measurement of the detection system in the University of Alberta and the University of Münster.}
    \label{fig:backgroundgamma}
\end{figure}

\subsection{Identification of isotopes present in the samples}
\label{subsec:gammAlberta}

The Vitrovex glass sample measured in Alberta has a cylindrical shape and weights $\SI{575\pm5}{g}$, which was provided by the company. The sample of Benthos glass is a fragment from a half vessel of an IceCube DOM, and weights  $\SI{341(5)}{g}$. The measurement of the first sample lasted $1.8$ days and of the latter $3.1$ days\footnote{Respectively $\SI{159741}{s}$ and $\SI{264998}{s}$.}. The results from the Vitrovex sample can be seen in figure \ref{fig:LandscapeFigure}. For the identification of most of the peaks, the software Prospect Genie 2000 was used, which has a gamma-ray energy library. It can be seen that all gamma-rays can be identified coming from only 12 different isotopes. These, except for \ch{^{40}K}, are part of one of the three natural decay chains: \ch{^{235}U} , \ch{^{238}U} and $\ch{^{232}Th}$. The radionuclides from these series with their most intensive gamma-emissions can be found in table \ref{tab:allisotopes}.

\begin{table}[H]
\centering
\caption{Most intense gamma-ray energy of each isotope from the $\ch{^{238}U}$ ,$\ch{^{235}U}$ and $\ch{^{232}Th}$ chain. In bold and italic are the isotopes that were measured in both, the spectrum of the Benthos and Vitrovex sample. \\
n.e. : Isotope does not emitte gamma-rays.\\
$\dagger$: Regarding its intensity, the isotope should have been measured, if the chain is in secular equilibrium.}
\label{tab:allisotopes}
\begin{tabular}{@{}lll|lll|lll@{}}
\toprule
\multicolumn{3}{c|}{$\ch{^{238}U}$ Chain} & \multicolumn{3}{c|}{$\ch{^{235}U}$ Chain} & \multicolumn{3}{c}{$\ch{^{232}Th}$ Chain} \\ \midrule
 & \begin{tabular}[c]{@{}l@{}}Energy\\ (keV)\end{tabular} & \begin{tabular}[c]{@{}l@{}}Intensity\\ (\%)\end{tabular} &  & \begin{tabular}[c]{@{}l@{}}Energy\\ (keV)\end{tabular} & \begin{tabular}[c]{@{}l@{}}Intensity\\ (\%)\end{tabular} &  & \begin{tabular}[c]{@{}l@{}}Energy\\ (keV)\end{tabular} & \begin{tabular}[c]{@{}l@{}}Intensity\\ (\%)\end{tabular} \\ \midrule
$\ch{^{238}U}$ & n.e. &  & $ \textbf{\textit{\ch{^{235}U}}}$ & 185.7 & 57 & $\ch{^{232}Th}$ & 63.81 & 0.263 \\
$ \textbf{\textit{\ch{^{234}Th}}}$ & 63.29 & 3.7 & $\ch{^{231}Th}$ & 89.95 & 6.6 $\dagger$ & $\ch{^{228}Ra}$ & n.e &  \\
$ \textbf{\textit{\ch{^{234}Pa}}}$ & 1001.03 & 0.84 & $\ch{^{231}Pa}$ & 300.06 & 2.41 & $ \textbf{\textit{\ch{^{228}Ac}}}$ & 911.2 & 25.8 \\
$\ch{^{234}U}$ & 53.2 & 0.12 & $\ch{^{227}Ac}$ & 99.6 & 0.006 & $ \textbf{\textit{\ch{^{228}Th}}}$ & 84.37 & 1.19 \\
$\ch{^{230}Th}$ & 67.7 & 0.38 & $\ch{^{227}Th}$ & 235.96 & 12.9 $\dagger$ & $\ch{^{224}Ra}$ & 240.99 & 4.1 $\dagger$ \\
$\ch{^{226}Ra}$ & 186.2 & 3.64 & $\ch{^{223}Ra}$ & 269.5 & 13.9 $\dagger$ & $\ch{^{220}Rn}$ & 549.73 & 0.11 \\
$\ch{^{222}Rn}$ & 510 & 0.08 & $\ch{^{219}Rn}$ & 271.2 & 10.8 $\dagger$ & $\ch{^{216}Po}$ & 804.9 & 0.002 \\
$\ch{^{218}Po}$ & n.e. &  & $\ch{^{215}Po}$ & n.e &  & $ \textbf{\textit{\ch{^{212}Pb}}}$ & 238.63 & 43.6 \\
$ \textbf{\textit{\ch{^{214}Pb}}}$ & 351.93 & 35.6 & $\ch{^{211}Pb}$ & 404.85 & 3.78 & $ \textbf{\textit{\ch{^{212}Bi}}}$ & 727.33 & 6.67 \\
$\textbf{\textit{\ch{^{214}Bi}}}$ & 609.32 & 45.5 & $\ch{^{211}Bi}$ & n.e. &  & $\ch{^{212}Po}$ & n.e. &  \\
$\ch{^{214}Po}$ & 799.7 & 0.01 & $\ch{^{207}Tl}$ & 897.77 & 0.263 & $ \textbf{\textit{\ch{^{208}Tl}}}$ & 583.19 & 85 \\
$\textbf{\textit{\ch{^{210}Pb}}}$ & 46.54 & 4.25 & $\ch{^{207}Pb}$ & n.e. &  & $\ch{^{208}Pb}$ & n.e. &  \\
$\ch{^{210}Bi}$ & n.e. &  &  &  &  &  &  &  \\
$\textbf{\textit{\ch{^{210}Po}}}$ & 803.06 & 0.001 &  &  &  &  &  &  \\
$\ch{^{206}Pb}$ & n.e. &  &  &  &  &  &  &  \\ \bottomrule
\end{tabular}
\end{table}

\begin{sidewaysfigure}[]
    \includegraphics[scale=1.0]{Figures/gammaspectroscopy/VitrovexCanada.pdf}
    \caption{Gamma spectrum of the measurement of the Vitrovex glass sample and the background done at the University of Alberta.}
    \label{fig:LandscapeFigure}
\end{sidewaysfigure}



For the estimation of the absolute detection efficiency, a Geant4 simulation provided by Pawel Mekarski from the University of Alberta was used. The detector geometry is already defined in this code, while the geometry of the samples was incorporated into the program in the scope of this work \footnote{Picutres of the samples and their geometry in Geant4 can be found in appendix REF!.}. The efficiency is calculated by simulating $N_T$ gamma-rays stemming isotropically from the sample and saving the energy deposited in the active volume of the germanium crystal. The result of this is a histogram similar to the response function seen in figure \ref{fig:gammafuncresp}. Then, integrating the FEP of this histogram, one gets the number of gamma-rays $N_D$ that were totally absorbed in the detector, which ultimately leads to the efficiency $\varepsilon = \dfrac{N_D}{N_T}$.
\par Once the efficiency is simulated and calculated for every gamma-ray energy of the identified isotopes, it is possible to determine the decay rate of the radionuclides. For this, only the most intense gamma-rays were used, which did not overlap with other lines. In table \ref{table:actCanada} the resulting mass-specific activity $\frac{A}{m}$ are provided for each isotope. Here the total activity of the sample calculated with \autoref{formula:activity} is divided by its mass $m$ for better comparison between specimen of the same material.
\begin{table}[H]
\centering
\caption{Mass-specific activities from the samples measured in the University of Alberta. For the case of the $\ch{^{238}U}$- and $\ch{^{232}Th}$ series, the average of the activities of their isotopes is also given.}
\label{table:actCanada}
\begin{tabular}{@{}lll@{}}
\toprule
                      & \multicolumn{2}{l}{Mass-specific activity (Bq/kg)} \\ \midrule
                      & Vitrovex                 & Benthos                 \\ \midrule
$\ch{^{214}Bi}$       & $4.24\pm0.14$            & $3.41\pm0.16$           \\
$\ch{^{214}Pb}$       & $5.19\pm0.15$            & $3.85\pm0.12$           \\
$\ch{^{234}Th}$       & $4.49\pm3.3$             & $1.81\pm2.6$            \\ \midrule
$\ch{^{238}U}$-Chain  & $4.69\pm0.10$            & $3.67\pm0.10$           \\ \midrule
$\ch{^{228}Ac}$       & $1.42\pm0.24$            & $0.84\pm0.21$           \\
$\ch{^{212}Pb}$       & $1.04\pm0.20$            & $0.72\pm0.12$           \\
$\ch{^{208}Tl}$       & $0.97\pm0.14$            & $0.78\pm0.17$           \\ \midrule
$\ch{^{232}Th}$-Chain & $1.07\pm0.10$            & $0.76\pm0.09$           \\ \midrule
$\ch{^{235}U}$-Chain  & $0.62\pm0.16$            & $0.30\pm0.12$           \\ \midrule
$\ch{^{40}K}$         & $66.2\pm1.2$             & $5.3\pm0.6$             \\ \bottomrule
\end{tabular}
\end{table}

\par In nature, the three decay chains are approximately in secular equilibrium, as long as the soil has been undisturbed for a long time \cite{9781498769297}. This means, it is to be expected that all isotopes listed in table \ref{tab:allisotopes} should have the same decay rate within its decay series. Consequently, once the activity of an isotope from a chain is known, one can calculate the expected number of detected gamma-rays originated from a specific isotope, by solving for $S_0$ in the \autoref{formula:activity}. This was done for all undetected isotopes from the series. Four FEP from the $\ch{^{235}U}$- and one from $\ch{^{232}Th}$-chain, should have been detected in the measurement if the assumption of secular equilibrium were to be correct. Nevertheless, all their gamma-rays that were intense enough to be detected, had a similar energy of another FEP with higher activity. For example, the line from $\ch{^{223}Ra}$ and $\ch{^{219}Rn}$ with an energy of $\SI{269.5}{keV}$ and $\SI{271.2}{keV}$, from the $\ch{^{235}U}$-chain, lie in between a line from $\ch{^{228}Ac}$ with an energy of $\SI{270.2}{keV}$ from the $\ch{^{232}Th}$-series, which has a higher activity. Hence, both isotopes could have been detected, but can not be distinguished from other lines within the detector resolution. This is also the case for the $\SI{269.5}{keV}$-gamma-ray $\ch{^{224}Ra}$ and the  $\SI{89.95}{keV}$ from $\ch{^{231}Th}$ (see figure \ref{fig:LandscapeFigure}).

\par It is noteworthy that the activities of isotopes of a single series do not agree within their error, but are similar. Hence, the assumption of secular equilibrium is only an approximation. Nevertheless, it will be assumed that all isotopes of the series have the same activity, using the average of measured activities, which also can be found in table \ref{table:actCanada}. In the case of the $\ch{^{235}U}$-series, only one isotope in the chain was measured, namely $\ch{^{235}U}$, and therefore its activity is assumed for the rest of radionuclides of the series.

\par It is to notice that the Benthos sample has a far lower concentration of radioisotopes than the Vitrovex one, especially of $\ch{^{40}K}$, which is more than twelve times higher in the latter. This would imply that choosing the Benthos glass would yield lower dark rates from luminescence, assuming that both glasses have the same scintillation properties. This raises the question, wether the "cleanliness" of the glass samples is an intrinsic property of the brand, or if there is a high variance between specimens of the same material. To check up on this, new measurements were done in the scope of this work. Also, specimens of the two optical gel brands, Wacker and QSI, were produced and examined. The next section presents the results of these measurements. 




\subsection{Expanding statistics for the activities of radionuclides}
\label{subsec:gammMünster}


\begin{wrapfigure}{O}{0pt}
\centering
\vspace{-10pt}
\includegraphics[scale=1.0]{Figures/gammaspectroscopy/055efficiencywrap.pdf}
\caption{Absolute detection efficiency of the expected gamma-rays for both geometries, the cylindrical small samples (Vitrovex) and the half vessels (Vitrovex and Benthos).}
\label{fig:efficiency}
\vspace{-10pt}
\end{wrapfigure}

\par The Vitrovex sample tested in Canada was one of a set of three that were delivered from the company. These samples were available for measurements, together with two half pressure vessels of the same shape, one made from Benthos and one from Vitrovex glass. The latter is an old specimen of the early 2000s. As this samples have distinct geometries, the detector will have very different absolute detection efficiencies between these samples. While the smaller cylindrical samples can be placed near the germanium detector, the half vessel will use a larger volume, reducing the solid angle from where the gammas will be detected. The efficiency was therefore simulated preparatory to the measurement so that the measurement times could be estimated in such a way, that the measured spectra had similar statistical uncertainties. For this, a Geant4 code with the geometry of the detection system provided by Volker Hannen was integrated into the software used in the last section. The results can be found in figure \ref{fig:efficiency}.
\par  The detection efficiency for the cylindrical samples is, as expected, almost one order of magnitude larger than the one for the half vessels. However, the mass of these samples differ also by one order of magnitude, thus a similar measuring duration should deliver comparable statistical uncertainties. The evaluation of the peaks was the same as in the last section. The calculated activities are presented in table \ref{tab:actMue} and the spectra can be found in Appendix \textcolor{red}{Blah!}. 
\par First of all, it can be noticed that the measurements for both small Vitrovex samples yielded a similar result to the one from Canada. The biggest divergence between these samples were the results for the isotopes from the $\ch{^{232}Th}$-Chain and $\ch{^{40}K}$ with around $20$-$30 \%$ difference. Greater deviations shows the Vitrovex vessel, which has almost twice the amount of radionuclides from the $\ch{^{238}U}$- and $\ch{^{232}Th}$-series and none measurable $\ch{^{40}K}$. This is also the case for the Benthos vessel, which presents a higher amount of the natural series, but a far lower $\ch{^{40}K}$ activity, than the sample from Canada.
\begin{table}[H]
\centering
\caption{Mass-specific activities from the samples measured in this work. For the case of the $\ch{^{238}U}$- and $\ch{^{232}Th}$ series, the average of the activities of their isotopes is also given. The last two rows present the measuring time and the weight of the samples respectively. VV stands for Vitrovex and BT for Benthos.}
\label{tab:actMue}
\begin{tabular}{@{}llllll@{}}
\toprule
                               &                   & \multicolumn{4}{l}{Mass-specific activity (Bq/kg)}                                                                                                  \\ \midrule
                               & VV 1              & VV 2              & VV vessel              & BT vessel              & \footnotesize{Wacker \& QSI gel}                                                   \\ \midrule
$\ch{^{214}Bi}$                & $4.01\pm0.16$     & $4.29\pm0.17$     & $8.14\pm0.20$         & $5.14\pm0.17$           & \textless0.15                                                                 \\
$\ch{^{214}Pb}$                & $4.82\pm0.12$     & $4.83\pm0.14$     & $8.83\pm0.18$         & $5.29\pm0.16$           & \textless0.11                                                                 \\
$\ch{^{234}Th}$                & $5.2\pm0.9$       & $4.8\pm1.2$       & $5.1\pm0.8$           & $4.7\pm0.7$             & \textless0.76                                                                 \\ \midrule
$\ch{^{238}U}$-Chain           & $4.53\pm0.10$     & $4.61\pm0.19$     & $8.42\pm0.13$         & $5.20\pm0.12$           & \textless 0.11                                                                \\ \midrule
$\ch{^{228}Ac}$                & $1.31\pm0.21$     & $1.34\pm0.22$     & $2.37\pm0.24$         & $1.71\pm0.22$           & \textless0.26                                                                 \\
$\ch{^{212}Pb}$                & $1.42\pm0.10$     & $1.34\pm0.11$     & $2.03\pm0.12$         & $1.04\pm0.11$           & \textless0.10                                                                 \\
$\ch{^{208}Tl}$                & $1.38\pm0.20$     & $1.32\pm0.21$     & $2.06\pm0.28$         & $1.16\pm0.25$           & \textless0.15                                                                 \\ \midrule
$\ch{^{232}Th}$-Chain          & $1.39\pm0.09$     & $1.34\pm0.09$     & $2.27\pm0.10$         & $1.16\pm0.09$           & \textless 0.10                                                                \\ \midrule
$\ch{^{235}U}$-Chain           & $0.56\pm0.07$     & $0.61\pm0.07$     & $0.75\pm0.08$         & $0.61\pm0.09$           & \textless0.05                                                                 \\ \midrule
$\ch{^{40}K}$                  & $53.6\pm1.7$      & $57.5\pm1.8$      & \textless0.99         & $1.0\pm1.4$             & \textless1.32                                                                 \\ \bottomrule
\begin{tabular}[c]{@{}l@{}} \footnotesize{Measurement}\\ \footnotesize{\hspace{0.3cm} time (s)}\end{tabular} & \footnotesize{256419}  & \footnotesize{262815}  & \footnotesize{265731}       & \footnotesize{269667}       & \begin{tabular}[c]{@{}l@{}}\footnotesize{251533}\\ \footnotesize{315285}\end{tabular}  \\
\footnotesize{Sample mass}                  & \footnotesize{ $\SI{641\pm5}{g}$} & \footnotesize{ $\SI{604\pm5}{g}$ }& \footnotesize{ $\SI{4.37\pm0.01}{kg}$} & \footnotesize{ $\SI{4.43\pm0.01}{kg}$} & \begin{tabular}[c]{@{}l@{}}\footnotesize{ $\SI{649\pm5}{g}$}\\ \footnotesize{ $\SI{651\pm5}{g}$}\end{tabular}\\ \bottomrule
\end{tabular}
\end{table}

\par The fact that the three small samples from Vitrovex have similar activities, but deviate a lot from the Vessel, may imply that the amount of radioactivity is dependent on the production badge, as probably the same source for the raw material is used for each one. Also, $\ch{K_2 O}$ is often added in different production steps of the glass, but is not an imperative, since it can be replaced with $\ch{Na_2 O}$. This could explain that some samples have high activities of $\ch{^{40}K}$ and other no measurable amounts. For a better understanding of this deviations, the productions steps from samples should be known by direct communication with the manufacturers. This is especially the case for the Vitrovex vessels, as the three small samples, which were produced in late 2015, have higher $\ch{^{40}K}$ activities than the older Vitrovex vessel. If this difference comes merely from a production step, it could be avoidable. Also, the new Vitrovex vessels designed for the mDOM should be examined with gamma spectroscopy. This could not be done in the scope of this work, as the detection system was too small for them. But it would be appropriate, as the production of the smaller Vitrovex samples may be different to the one of the vessel.
\par The spectra for the gel samples did not yield any measurable activity. This does not exclude them from the studies of this thesis, as they could still have luminescence properties. Thus, a decay originating from the vessel could deposit some energy in the gel layer, which could then emit scintillation photons. The next chapters will focus on this possibility and the scintillation properties of each material.















