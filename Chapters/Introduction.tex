
\chapter{Introduction}
\vspace{6pt}
    { \setstretch{1.05}
    
     
    
Since the beginnings of humankind, the fascination with the universe has been the main driver for learning about nature and its laws. The motion of the stars and their patterns influenced peoples life enormously for thousands of years, keeping the track of seasons and time. The science of astronomy took a gigantic leap forward with the invention of the optical telescope in the 1600s century, which enabled the systematic study of the universe.

\par In the last century, astronomy across the electromagnetic spectrum - ranging from gamma-ray satellites to radio observatories - has established itself as a solid tool for studying objects of our universe. However, with the discovery of cosmic rays and the development of the Standard Model, we became aware that not only photons can be a source of seemingly unlimited knowledge. In this context, with the recent detection at LIGO and VIRGO of the gravitational wave GW170814, the product of two merging black holes \cite{Abbott:2017oio}, the era of multi-messenger astronomy has officially begun. 

\par Nevertheless, one of the first milestones of multi-messenger astronomy can be traced back to exactly two decades ago: the detection of several neutrinos at the Kamiokande-II, IMB and Baksan neutrino observatories from SN1987A - a supernova so close to earth that it was easily visible to the naked eye \cite{Schaeffer:1990bt}.

\par Neutrinos are ideal messengers of galactic events since they are produced in many different processes and they can travel almost freely through the universe. However, although their low interaction probability constitutes their biggest advantage as cosmic messengers, it also makes their detection a very difficult task. Nevertheless, scientists have taken on the challenge and various large-volume neutrino detectors have been built across the earth. IceCube, currently the biggest among them, is the first cubic kilometre sized neutrino observatory and it was designed to detect high energy neutrinos from astrophysical sources. It is located deep in the glacial ice at the South Pole, where several strings with optical sensors are deployed - the ``eyes'' of the detector.
\par IceCube is in full operation since 2011 and it has already discovered a flux of high-energy neutrinos of cosmic origin with energies up to $\mathcal{O}$(PeV) \cite{Kopper2013, Aartsen:2016ngq}, making it the first experiment to prove the feasibility of neutrino astronomy. This provides the motivation for new IceCube extensions that will dramatically boost IceCube's performance. In this context, new concepts are being considered for the optical sensors that will be deployed in these upgrades, taking advantage of the advances in technology and the knowledge gained through the years of IceCube's operation. One novel concept is the multi-PMT Digital Optical Module, or short mDOM. This module includes an array of small-size photomultipliers inside a pressure vessel, instead of housing a single larger one, like in the current IceCube modules. 
\newpage
\par As the deep ice at the South Pole is almost free of optical activity, the light produced by the modules themselves represents the dominant background source. Therefore, a very important aspect of the development of the mDOM is to understand this noise to estimate the overall background of the detector and its influence on the signal processing of real events. In this regard, two important light sources are Cherenkov and scintillation photons produced by radioactive decays inside the module's pressure vessel. The former, Cherenkov radiation, has already been studied in the framework of a Bachelor thesis considering $\ch{^{40}K}$ decays \cite{Tabea}, an important source of Cherenkov light inside the vessels. However, it did not completely explain the observed background. Investigations on the luminescence produced by these decays have been limited, although studies have already shown its influence on pressure spheres of the AMANDA experiment \cite{helbing, olav}. This thesis aims to contribute to the understanding and characterisation of these background processes and their impact on the mDOM performance.

}